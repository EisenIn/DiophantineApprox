\documentclass[12pt,a4paper]{article}

\usepackage{amsmath}
\usepackage{amssymb}
\usepackage{amsfonts}
\usepackage{amsthm}
\usepackage{graphics}
\usepackage{fullpage}
\usepackage{graphicx}
\usepackage{caption}
\usepackage{subcaption} 
\usepackage{enumerate}
\usepackage{polynom}
\usepackage[utf8]{inputenc} 
\usepackage{utf8math}
\usepackage{xfrac}

\date{}

\usepackage[boldsans]{concmath}

\theoremstyle{plain}
\newtheorem{theorem}{Th\'eor\`eme}
\newtheorem{Sol}{Solution}
\newtheorem*{Sol*}{Solution}
\theoremstyle{definition}
\newtheorem{Ex}{Exercise}
\newtheorem{lemma}[theorem]{Lemma}


\def \N {\mathbb N}
\def \Q {\mathbb Q}
\def \R {\mathbb R}
\def \Z {\mathbb Z}
\def \K {\mathbb K}
\def \C {\mathbb C}
\def \F {\mathbb F}
\def \id {{\rm id}\,}
\def \Ker {{\rm Ker}\,}
\def \Im {{\rm Im}\,}
\def \Vect {{\rm Vect}\,}

\newcommand{\df}{\mathrel{\mathop:}=}
\newcommand{\dx}{ \ \text{d} \, x}

\newcommand{\pscal}[1]{\langle {#1} \rangle}
\DeclareMathOperator{\spa}{span}
\newcommand{\nint}[1]{\ensuremath{\lfloor#1\rceil}}


%%%%%%%%%%%%%%%%%%%%%%%%%%%%%%%%%%%%%%%%%%%%%%%%%%%%%%%%%%%%%%%%%%%%%%%%%%%%%%%%%%
%%%%%%%%%%%%%%%%%%%%%%%%%%%%%%%%%%%%%%%%%%%%%%%%%%%%%%%%%%%%%%%%%%%%%%%%%%%%%%%%%%
%
%       ENABLE or DISABLE dislpay of solutions
% 
%%%%%%%%%%%%%%%%%%%%%%%%%%%%%%%%%%%%%%%%%%%%%%%%%%%%%%%%%%%%%%%%%%%%%%%%%%%%%%%%%%
%%%%%%%%%%%%%%%%%%%%%%%%%%%%%%%%%%%%%%%%%%%%%%%%%%%%%%%%%%%%%%%%%%%%%%%%%%%%%%%%%%
		\newif\ifsolutions
		
		% ENABLE or DISABLE display of solutions
		\solutionstrue
		%\solutionsfalse


%%%%%%%%%%%%%%%%%%%%%%%%%%%%%%%%%%%%%%%%%%%%%%%%%%%%%%%%%%%%%%%%%%%%%%%%%%%%%%%%%%
%		Dont touch much, just change the correct number and date :). Based on the setup above, the solutions will be automaticelly displayed or hidden.
%%%%%%%%%%%%%%%%%%%%%%%%%%%%%%%%%%%%%%%%%%%%%%%%%%%%%%%%%%%%%%%%%%%%%%%%%%%%%%%%%%
		\newcommand{\exercise}[2]{
			\begin{Ex} #1 \end{Ex}
			\ifsolutions  \begin{Sol*} #2 \end{Sol*} \bigskip \else \bigskip  \fi
		}

		

%%%%%%%%%%%%%%%%%%%%%%%%%%%%%%%%%%%%%%%%%%%%%%%%%%%%%%%%%%%%%%%%%%%%%%%%%%%%%%%%%%
%   Beginning of the document. Make sure to add the correct dates and numbers everywhere.
%%%%%%%%%%%%%%%%%%%%%%%%%%%%%%%%%%%%%%%%%%%%%%%%%%%%%%%%%%%%%%%%%%%%%%%%%%%%%%%%%%
\begin{document}

\begin{center}
{Prof. Friedrich Eisenbrand \hfill December 22nd 2023}
\end{center}
	
\hrule\vspace{\baselineskip}

\begin{center}
\textbf{Diophantine approximation}

Fall 2023

\bigskip

\textbf{Set 13}
\ifsolutions{\textbf{- Solutions}} \else{} \fi
\end{center}

\hrule\vspace{\baselineskip}


%%%%%%%%%%%%%%%%%%%%%%%%%%%%%%%%%%%%%%%%%%%%%%%%%%%%%%%%%%%%%%%%%%%%%%%%%%%%%%%%%%
%   Beginning of the exercises
%%%%%%%%%%%%%%%%%%%%%%%%%%%%%%%%%%%%%%%%%%%%%%%%%%%%%%%%%%%%%%%%%%%%%%%%%%%%%%%%%%

%%%%%%%%%%%%%%%%%%%%%%%%%%%%%%%%%%%%%%%%%%%%%%%%%%%%%%%%%%%%%%%%%%%%%%%%%%%%%%%%%%
%    Each exercise should look like 
% 		\exercise{ Exercise }{ Solution }
%%%%%%%%%%%%%%%%%%%%%%%%%%%%%%%%%%%%%%%%%%%%%%%%%%%%%%%%%%%%%%%%%%%%%%%%%%%%%%%%%%

\exercise{\label{ex:1}
	We call two real numbers $\alpha, \beta \in \R$ \emph{equivalent} if there exist some matrix
		\[ \begin{pmatrix} a & b \\ c & d \end{pmatrix} \in {\rm GL}_2(\Z) \]
	such that $\beta$ is the Möbius transformation of $\alpha$, that is
		\[ \beta = \begin{pmatrix} a & b \\ c & d \end{pmatrix} \circ α =    \frac{a \alpha + b}{c \alpha + d}. \]
	
	\begin{enumerate}[i)]
		\item Show that this relation is indeed an \emph{equivalence relation}.
		\item Let $\alpha = [a_0;a_1, a_2 \dots]$ and fix $r_n := [a_n; a_{n+1}, a_{n+2}, \dots]$ for each $n \geq 0$.
		Show that $\alpha$ is equivalent to $r_n$ for each $n \geq 0$.

		\item Deduce that if $\alpha$ and $\beta$ take the shapes
			\begin{gather*}
				\alpha = [a_1, \dots, a_i, c_1, c_2, \dots], \\
				\beta = [b_1, \dots, b_j, c_1, c_2, \dots],
			\end{gather*}
		for some integers $a_1, \dots, a_i$ ($i \geq 0$), $b_1, \dots, b_j$ ($j\geq0$), and $c_1, c_2, \dots$,
		then $\alpha$ and $\beta$ are equivalent.
	\end{enumerate}
}
{
	\begin{enumerate}
		\item The identity matrix $I_2$ sends $\alpha$ to itself, showing the reflexivity.
		The inverse integer matrix shows the symmetry.
		The multiplication of the matrices shows transitivity.

		\item It suffices to show that $\alpha$ is equivalent to $r_1$ to conclude.
		Since 
			\[ \alpha = a_0 + \frac{1}{r_1} = \frac{a_0 r_1 + 1}{r_1} = \begin{pmatrix} a_0 & 1 \\ 1 & 0 \end{pmatrix} \circ r_1, \]
		the two are equivalent.

		\item $\alpha$ and $\beta$ are both equivalent to $[c_1, c_2, \dots]$, and hence to each other by transitivity.
	\end{enumerate}
}



\exercise{
Let $α ∈ℝ ⧹ℚ$ be a quadratic irrationality of discriminant $Δ$ and let $U ∈ ℤ^{2 × 2}$ have determinant $\pm 1$. Show that $U \circ α$ is a quadratic irrationality of discriminant $Δ$. 

}{
	Let $A \alpha^2 + B\alpha + C = 0$ for some $A, B, C \in \Z, A \neq 0$.
	Then $\Delta = B^2 - 4 A C$.

	Next, for $U = \begin{pmatrix} a & b \\ c & d \end{pmatrix}$ such that $\alpha = U \circ \beta$, then
		\begin{gather*}
			A \left( \frac{a \beta + b}{c \beta + d} \right)^2 + B \left( \frac{a \beta + b}{c \beta + d} \right) + C = 0, \\
			\iff \\
			A ( a \beta + b)^2 + B (a \beta + b)(c \beta + d) + C (c \beta + d)^2 = 0, \\
			\iff \\
			 \beta^2 \left( A a^2 + B a c + C c^2 \right) + \beta \left( 2 A a b + Ba d + Bb c + 2 C c d \right) + \left(Ab^2 + B bd + Cd^2 \right) = 0.
			\iff \\
			\tilde{A} \beta^2 + \tilde{B} \beta + \tilde{C} = 0,
		\end{gather*}
	for appropriately defined $\tilde{A}, \tilde{B}, \tilde{C} \in \Z$.
	It remains only to show that $\tilde{B}^2 - 4 \tilde{A} \tilde{C} = \Delta$, which we leave for the reader.

	Note that $\det U = \pm 1$ is key here. In fact, one has generally $\tilde{B}^2 - 4 \tilde{A} \tilde{C} = (ad - bc)^2 \Delta$.
}

\exercise{Throughout this exercise, $α ∈ ℝ ⧹ℚ$ denotes a quadratic irrationality. 
  Recall that  $α$ is \emph{reduced}, if $α>1$ and $-1 < α'<0$ holds, where $α'$ is the conjugate of $α$.
  \begin{enumerate}[i)] 
  \item If  $α$ is reduced, then $1 / (α - ⌊α⌋)$ is reduced.
  \item Conclude the following for a quadratic irrationality $α ∈ ℝ$, where $α = [a_0 ; a_1, \dots, a_{n-1}, r_n]$ where $r_n$ is the $n$-th remainder of the continued fraction expansion of $α$ with $n≥1$. If $r_n$ is reduced, then so is $r_{n+1}$.
  \item Consider our standard notation $p_{n}/ q_n = [a_0;a_1,\dots,a_n]$. 
    Show that
    \begin{displaymath}
      α =
      \begin{pmatrix}
        p_{n-1} & p_{n-2} \\
        q_{n-1} & q_{n-2} 
      \end{pmatrix} \circ r_n
    \end{displaymath}
    and therefore
    \begin{displaymath}
      r_n ' = - \frac{q_{n-2} α' - p_{n-2}}{ q_{n-1} α' - p_{n-1}} = - \frac{q_{n-2}}{q_{n-1}} \left( \frac{α' - \frac{p_{n-2}}{ q_{n-2}}}{α' - \frac{p_{n-1} }{ q_{n-1}} }\right)
    \end{displaymath}
  \item Show that one has $-1 < r_n' <0$  for $n ∈ ℕ$  sufficiently large implying that $r_n$ is reduced for $n ∈ ℕ$  sufficiently large.
    \item Conclude that the continued fraction  expansion of $α $  is periodic. 
  \end{enumerate}
}{
	\begin{enumerate}
		\item Let $\beta = 1/(\alpha - \lfloor \alpha \rfloor)$. Clearly $\beta$ is a quadratic irrationality, and $\beta > 1$.
		Next, $\beta' = 1/(\alpha' - \lfloor \alpha \rfloor)$. One has $-1 < \beta < 0$ since
			\[ \alpha' < 0 < 1 \leq \lfloor \alpha \rfloor. \]
	
		\item This follows directly from i) since $r_n = a_n + \frac{1}{r_{n+1}}$ where $a_n = \lfloor r_n \rfloor$.

		\item This was shown by induction in set 3, with notation $x_n$ instead of $r_n$.
		Expressing $r_n$ w.r.t. $\alpha$ by inverting the matrix and taking conjugates on both sides yields the required relation for $r_n'$.

		Recall that the denominators verify $q_{n+1} = q_n a_{n+1} + q_{n-1} > q_n + 1$, and that the fractions $p_n / q_n$ approach $\alpha$ as $n$ grows. 
		Hence $r_n' < 0$ and $| r'_n | < 1$ for some $n$ large enough.

		\item Since, by the lecture, there are only finitely many reduced quadratic irrationalities with common discriminant $\Delta$, there must exist two integer $m \neq n$ sufficiently large such that $r_m = r_n$. 
		Hence $a_{m+k} = a_{n+k}$ for each $k\geq0$, and the expansion of $\alpha$ is periodic.
	\end{enumerate}

}

% The steps of the following exercise were taken from Cassels \emph{An introduction to Diophantine approximation} (1957).
% \exercise{
% 	Let $\alpha \in \R\setminus\Q$. 
% 	Recall that Dirichlet's approximation theorem gives infinitely $q\geq1, p\in\Z$ such that $ | q \alpha - p |  \leq \frac1q. $
% 	We aim to show the following tighter statement due to Adolf Hurwitz (1891).

% 	There exists inifinitely many $q\geq1, p\in\Z$ such that
% 		\[ | q \alpha - p| \leq \frac{1}{\sqrt{5}q}. \]
% 	This inequality is tight for the golden number $\alpha = \frac{1+\sqrt{5}}2$.

% 	Furthermore, if $\alpha$ is not equivalent to $\frac{1+\sqrt{5}}2$, there are infinitely many $q\geq1, p\in\Z$ such that
% 		\[ | q \alpha - p| \leq \frac{1}{\sqrt{8}q}. \]

% 	To this end, define $\{ q_n / p_n\}_{n\geq0}$ the best approximations of $\alpha = [a_0; a_1, \dots]$, and
% 		\[ A_n := q_n \cdot | q_n \alpha - p_n | \]
% 	for all $n \geq 0$.

% 	\begin{enumerate}[i)]
% 		\item Show that $q_{n-1} \cdot | q_n \alpha - p_n | + q_n \cdot | q_{n-1} \alpha - p_{n-1} | = 1$.

% 		\item Deduce that, for $\lambda = q_{n-1}/q_n$ and $\mu = q_{n+1}/q_n$,
% 			\begin{gather}
% 				\lambda^2 A_n - \lambda + A_{n-1} = 0, \label{1} \\
% 				\mu^2 A_n - \mu + A_{n+1} = 0, \label{2} \\
% 				\mu - \lambda = a_n \label{3}
% 			\end{gather}
% 		for all $n \geq 1$.

% 		\item Deduce from \eqref{2} - \eqref{1} that
% 			\begin{gather}
% 				a_n A_n (\lambda + \mu) = a_n + A_{n-1} - A_{n+1}. \label{4}
% 			\end{gather}

% 		\item Square equations \eqref{3} and \eqref{4} to find
% 			\begin{gather}
% 				2a_n^2 A_n^2 (\lambda^2 + \mu^2) = a_n^4 A_n^2 + \left(a_n - A_{n-1}-A_{n+1} \right)^2. \label{5}
% 			\end{gather}

% 		\item Deduce from \eqref{1} + \eqref{2} that
% 			\[ a_n^2 A_n^2 + 2A_n(A_{n-1} + A_{n+1}) = 1 - a_n^{-2} (A_{n-1} - A_{n+1})^2 \leq 1. \]
		
% 		\item Conclude that 
% 			\[ \min \{ A_{n-1}, A_n, A_{n+1} \}^2 \leq \frac{1}{a_n^2 + 4} \leq \frac15, \]
% 		and that
% 			\begin{gather}
% 				\min \{ A_{n-1}, A_n, A_{n+1} \} \leq \frac{1}{\sqrt{8}} \label{6}
% 			\end{gather}
% 		for infinitely many $n$ unless $a_n =1$ for all $n$ large enough.

% 		\item Use exercise \ref{ex:1} to show that \eqref{6} holds for infinitely many $n$ if $\alpha$ is not equivalent to $ \frac{1+\sqrt{5}}2$.
			
% 	\end{enumerate}
% }
% {}

\end{document}

%%% Local Variables:
%%% mode: latex
%%% TeX-master: t
%%% End:
