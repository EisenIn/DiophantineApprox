\documentclass[12pt,a4paper]{article}

\usepackage{amsmath}
\usepackage{amssymb}
\usepackage{amsfonts}
\usepackage{amsthm}
\usepackage{graphics}
\usepackage{fullpage}
\usepackage{graphicx}
\usepackage{caption}
\usepackage{subcaption} 
\usepackage{enumerate}
\usepackage{polynom}
\usepackage[utf8]{inputenc} 
\usepackage{utf8math}

\date{}

\usepackage[boldsans]{concmath}

\theoremstyle{plain}
\newtheorem{theorem}{Th\'eor\`eme}
\newtheorem{Sol}{Solution}
\newtheorem*{Sol*}{Solution}
\theoremstyle{definition}
\newtheorem{Ex}{Exercice}


\def \N {\mathbb N}
\def \Q {\mathbb Q}
\def \R {\mathbb R}
\def \Z {\mathbb Z}
\def \K {\mathbb K}
\def \C {\mathbb C}
\def \id {{\rm id}\,}
\def \Ker {{\rm Ker}\,}
\def \Im {{\rm Im}\,}
\def \Vect {{\rm Vect}\,}

\newcommand{\df}{\mathrel{\mathop:}=}
\newcommand{\dx}{ \ \text{d} \, x}

\newcommand{\pscal}[1]{\langle {#1} \rangle}
\DeclareMathOperator{\spa}{span}


%%%%%%%%%%%%%%%%%%%%%%%%%%%%%%%%%%%%%%%%%%%%%%%%%%%%%%%%%%%%%%%%%%%%%%%%%%%%%%%%%%
%%%%%%%%%%%%%%%%%%%%%%%%%%%%%%%%%%%%%%%%%%%%%%%%%%%%%%%%%%%%%%%%%%%%%%%%%%%%%%%%%%
%
%       ENABLE or DISABLE dislpay of solutions
% 
%%%%%%%%%%%%%%%%%%%%%%%%%%%%%%%%%%%%%%%%%%%%%%%%%%%%%%%%%%%%%%%%%%%%%%%%%%%%%%%%%%
%%%%%%%%%%%%%%%%%%%%%%%%%%%%%%%%%%%%%%%%%%%%%%%%%%%%%%%%%%%%%%%%%%%%%%%%%%%%%%%%%%
		\newif\ifsolutions
		
		% ENABLE or DISABLE display of solutions
		%\solutionstrue
		\solutionsfalse


%%%%%%%%%%%%%%%%%%%%%%%%%%%%%%%%%%%%%%%%%%%%%%%%%%%%%%%%%%%%%%%%%%%%%%%%%%%%%%%%%%
%		Dont touch much, just change the correct number and date :). Based on the setup above, the solutions will be automaticelly displayed or hidden.
%%%%%%%%%%%%%%%%%%%%%%%%%%%%%%%%%%%%%%%%%%%%%%%%%%%%%%%%%%%%%%%%%%%%%%%%%%%%%%%%%%
		\newcommand{\exercise}[2]{
			\begin{Ex} #1 \end{Ex}
			\ifsolutions  \begin{Sol*} #2 \end{Sol*} \bigskip \else \bigskip  \fi
		}

		

%%%%%%%%%%%%%%%%%%%%%%%%%%%%%%%%%%%%%%%%%%%%%%%%%%%%%%%%%%%%%%%%%%%%%%%%%%%%%%%%%%
%   Beginning of the document. Make sure to add the correct dates and numbers everywhere.
%%%%%%%%%%%%%%%%%%%%%%%%%%%%%%%%%%%%%%%%%%%%%%%%%%%%%%%%%%%%%%%%%%%%%%%%%%%%%%%%%%
\begin{document}

\begin{center}
{Prof. Friedrich Eisenbrand \hfill September 29th 2023}
\end{center}
	
\hrule\vspace{\baselineskip}

\begin{center}
\textbf{ Diophantine approximation}

Fall 2023

\bigskip

\textbf{Set 1}
\ifsolutions{\textbf{- Corrig\'e}} \else{} \fi
\end{center}

\hrule\vspace{\baselineskip}


%%%%%%%%%%%%%%%%%%%%%%%%%%%%%%%%%%%%%%%%%%%%%%%%%%%%%%%%%%%%%%%%%%%%%%%%%%%%%%%%%%
%   Beginning of the exercises
%%%%%%%%%%%%%%%%%%%%%%%%%%%%%%%%%%%%%%%%%%%%%%%%%%%%%%%%%%%%%%%%%%%%%%%%%%%%%%%%%%

%%%%%%%%%%%%%%%%%%%%%%%%%%%%%%%%%%%%%%%%%%%%%%%%%%%%%%%%%%%%%%%%%%%%%%%%%%%%%%%%%%
%    Each exercise should look like 
% 		\exercise{ Exercise }{ Solution }
%%%%%%%%%%%%%%%%%%%%%%%%%%%%%%%%%%%%%%%%%%%%%%%%%%%%%%%%%%%%%%%%%%%%%%%%%%%%%%%%%%
\exercise{
  \begin{enumerate}[a)]
  \item Show the identity $\left⌊ √{⌊x⌋}\right⌋ = \left⌊ √{x}\right⌋$ for $x ∈ℝ$. 
  \item Prove or disprove: $⌊x⌋ + ⌊y⌋ +⌊x+y⌋ ≤ 2⌊x⌋ + 2⌊y⌋$ for $x,y ∈ ℝ$.
  \item Prove the identity: $x^3 = 3 x ⌊x⌊x⌋⌋ + 3\{x\} \{x ⌊x⌋\} + \{x\}^3 - 3 ⌊x⌋ ⌊x ⌊x⌋⌋  + ⌊x⌋^3$ for $x ∈ ℝ$. 
  \end{enumerate}
  }

\exercise{ 
	Let $\alpha \in \R\setminus\Q$. Deduce from Dirichlet's theorem that there exist infinitely many $p \in \Z, q\in \N$ such that
		\[ | \alpha - \frac{p}{q} | < 1/q^2. \] 
} {}

\exercise{
	Show that, for any $n \in \N$ which is not a perfect square, $α = \sqrt{n} \in \R\setminus\Q$, and that one always has
		\[ | \alpha - \frac{p}{q} | > \frac{1}{(2 \alpha+1) q^2}. \]
}
{}

\exercise{
	Show that the sequence $\{\cos(n)^n\}_{n\geq1}$ does not converge to $0$.
      }

\exercise{
  Let $ 1 + ∑_{i=1}^∞ a_i 2^{-i}$, where  $a_i ∈ \{0,1\}$ for each $i$, be the binary representation of $\sqrt{2}$.  Show that there exists a constant $c ∈ ℕ_+$ such that, for all $i ∈ ℕ$, the subsequence $a_i,\dots, a_{c ⋅i}$ cannot consists entirely of ones. 
  
}

\end{document}

%%% Local Variables:
%%% mode: latex
%%% TeX-master: t
%%% End:
