\documentclass[12pt,a4paper]{article}

\usepackage{amsmath}
\usepackage{amssymb}
\usepackage{amsfonts}
\usepackage{amsthm}
\usepackage{graphics}
\usepackage{fullpage}
\usepackage{graphicx}
\usepackage{caption}
\usepackage{subcaption} 
\usepackage{enumerate}
\usepackage{polynom}
\usepackage[utf8]{inputenc} 
\usepackage{utf8math}

\date{}

\usepackage[boldsans]{concmath}

\theoremstyle{plain}
\newtheorem{theorem}{Th\'eor\`eme}
\newtheorem{Sol}{Solution}
\newtheorem*{Sol*}{Solution}
\theoremstyle{definition}
\newtheorem{Ex}{Exercice}


\def \N {\mathbb N}
\def \Q {\mathbb Q}
\def \R {\mathbb R}
\def \Z {\mathbb Z}
\def \K {\mathbb K}
\def \C {\mathbb C}
\def \id {{\rm id}\,}
\def \Ker {{\rm Ker}\,}
\def \Im {{\rm Im}\,}
\def \Vect {{\rm Vect}\,}

\newcommand{\df}{\mathrel{\mathop:}=}
\newcommand{\dx}{ \ \text{d} \, x}

\newcommand{\pscal}[1]{\langle {#1} \rangle}
\DeclareMathOperator{\spa}{span}


%%%%%%%%%%%%%%%%%%%%%%%%%%%%%%%%%%%%%%%%%%%%%%%%%%%%%%%%%%%%%%%%%%%%%%%%%%%%%%%%%%
%%%%%%%%%%%%%%%%%%%%%%%%%%%%%%%%%%%%%%%%%%%%%%%%%%%%%%%%%%%%%%%%%%%%%%%%%%%%%%%%%%
%
%       ENABLE or DISABLE dislpay of solutions
% 
%%%%%%%%%%%%%%%%%%%%%%%%%%%%%%%%%%%%%%%%%%%%%%%%%%%%%%%%%%%%%%%%%%%%%%%%%%%%%%%%%%
%%%%%%%%%%%%%%%%%%%%%%%%%%%%%%%%%%%%%%%%%%%%%%%%%%%%%%%%%%%%%%%%%%%%%%%%%%%%%%%%%%
		\newif\ifsolutions
		
		% ENABLE or DISABLE display of solutions
		\solutionstrue
		%\solutionsfalse


%%%%%%%%%%%%%%%%%%%%%%%%%%%%%%%%%%%%%%%%%%%%%%%%%%%%%%%%%%%%%%%%%%%%%%%%%%%%%%%%%%
%		Dont touch much, just change the correct number and date :). Based on the setup above, the solutions will be automaticelly displayed or hidden.
%%%%%%%%%%%%%%%%%%%%%%%%%%%%%%%%%%%%%%%%%%%%%%%%%%%%%%%%%%%%%%%%%%%%%%%%%%%%%%%%%%
		\newcommand{\exercise}[2]{
			\begin{Ex} #1 \end{Ex}
			\ifsolutions  \begin{Sol*} #2 \end{Sol*} \bigskip \else \bigskip  \fi
		}

		

%%%%%%%%%%%%%%%%%%%%%%%%%%%%%%%%%%%%%%%%%%%%%%%%%%%%%%%%%%%%%%%%%%%%%%%%%%%%%%%%%%
%   Beginning of the document. Make sure to add the correct dates and numbers everywhere.
%%%%%%%%%%%%%%%%%%%%%%%%%%%%%%%%%%%%%%%%%%%%%%%%%%%%%%%%%%%%%%%%%%%%%%%%%%%%%%%%%%
\begin{document}

\begin{center}
{Prof. Friedrich Eisenbrand \hfill September 29th 2023}
\end{center}
	
\hrule\vspace{\baselineskip}

\begin{center}
\textbf{ Diophantine approximation}

Fall 2023

\bigskip

\textbf{Set 1}
\ifsolutions{\textbf{- Corrig\'e}} \else{} \fi
\end{center}

\hrule\vspace{\baselineskip}


%%%%%%%%%%%%%%%%%%%%%%%%%%%%%%%%%%%%%%%%%%%%%%%%%%%%%%%%%%%%%%%%%%%%%%%%%%%%%%%%%%
%   Beginning of the exercises
%%%%%%%%%%%%%%%%%%%%%%%%%%%%%%%%%%%%%%%%%%%%%%%%%%%%%%%%%%%%%%%%%%%%%%%%%%%%%%%%%%

%%%%%%%%%%%%%%%%%%%%%%%%%%%%%%%%%%%%%%%%%%%%%%%%%%%%%%%%%%%%%%%%%%%%%%%%%%%%%%%%%%
%    Each exercise should look like 
% 		\exercise{ Exercise }{ Solution }
%%%%%%%%%%%%%%%%%%%%%%%%%%%%%%%%%%%%%%%%%%%%%%%%%%%%%%%%%%%%%%%%%%%%%%%%%%%%%%%%%%
\exercise{
  \begin{enumerate}[a)]
  \item Show the identity $\left⌊ √{⌊x⌋}\right⌋ = \left⌊ √{x}\right⌋$ for $x ∈ℝ$. 
  \item Prove or disprove: $⌊x⌋ + ⌊y⌋ +⌊x+y⌋ ≤ 2⌊x⌋ + 2⌊y⌋$ for $x,y ∈ ℝ$.
  \item Prove the identity: $x^3 = 3 x ⌊x⌊x⌋⌋ + 3\{x\} \{x ⌊x⌋\} + \{x\}^3 - 3 ⌊x⌋ ⌊x ⌊x⌋⌋  + ⌊x⌋^3$ for $x ∈ ℝ$. 
  \end{enumerate}
  }

\exercise{ 
	Let $\alpha \in \R\setminus\Q$. Deduce from Dirichlet's theorem that there exist infinitely many $p \in \Z, q\in \N$ such that
		\[ | \alpha - \frac{p}{q} | < 1/q^2. \] 
} {}

\exercise{
	Show that, for any $n \in \N$ which is not a perfect square, $α = \sqrt{n} \in \R\setminus\Q$, and that one always has
		\[ | \alpha - \frac{p}{q} | > \frac{1}{(2 \alpha+1) q^2}. \]
}
{}

\exercise{
	Show that the sequence $\{\cos(n)^n\}_{n\geq1}$ does not converge to $0$.
      }
{
	The goal is to show that there exist infinitely many $n$ such that $\cos(n)$ is sufficiently close to $1$ in order to break the influence of the $n$-th power.
	
	Notice first that
		\[ | 1 - \cos^n(n) | = | 1 - \cos(n) | \cdot |1 + \cos(n) + \cos^ 2(n) + \dots + \cos^{n-1}(n)| \leq  n  | 1 - \cos(n) |, \]
	which means that it is enough to bound $| 1 - \cos(n) |$ by eg. $\frac{1}{2n}$, for infinitely many $n$.

	To this end, consider Dirichlet's theorem and its consequence given by exercise 2.
	Pick $\alpha = 2\pi$, such that there exist infinitely many $p, q$ verifying
		\[ | 2\pi q - p | < \frac{1}{q}. \]
	While $p$ is close to a multiple of $2\pi$, $\cos(p)$ must be close to $1$.
	Indeed, the mean value theorem gives
		\[ |\cos(p) - 1 | = |\cos(p) - \cos(2\pi q)| \leq |\sin(u)| \cdot |p - 2\pi q| \leq \frac{1}{q} |\sin(u)|, \]
	for some real $u$ between $p$ and $2\pi q$.

	The same argument yields
		\[ |\sin(u)| = |\sin(u) - \sin(2\pi q)| \leq |u - 2\pi q| \leq |p - 2\pi q| \leq \frac{1}{q}. \]

	Noticing that $p/q$ is close to $2 \pi$, say $p/q \leq 7$, we get
		\[  | 1 - \cos^p(p) | \leq p  | 1 - \cos(p) | \leq \frac{p}{q^2} \leq \frac7q, \]
	which concludes.
}

\exercise{
  Let $ 1 + ∑_{i=1}^∞ a_i 2^{-i}$, where  $a_i ∈ \{0,1\}$ for each $i$, be the binary representation of $\sqrt{2}$.  Show that there exists a constant $c ∈ ℕ_+$ such that, for all $i ∈ ℕ$, the subsequence $a_i,\dots, a_{c ⋅i}$ cannot consists entirely of ones.  
}{
	Suppose, by contradiction, that for all constants $c \in \N$, there exists an $i \in \N$ such that the subsequence $a_i, \dots, a_{c \cdot i}$ consists entirely of ones.
	\[ \sqrt{2} = 1.\ [...] \ \underbrace{ 1 \cdots \cdots 1 }_{i, \dots, c\cdot i} \ [...] \ \]
	The goal is to show that replacing the bits after $a_i$ and replacing them by $2^{-(i-1)} = 2^{-i + 1}$ gives a good rational approximation of $\sqrt{2}$.

	To that end, we let $p/q = \sum_{0 \leq k \leq i-1} a_k 2^{-k} + 2^{-i + 1}$, and find that $q = 2^{i-1}$, $p=1 + \sum_{0 \leq k \leq i-1} a_k 2^{i-1-k}$ work.
	We will hence achieve a contradiction if the approximation $p/q$ of $\sqrt{2}$ is much better than $1/q$.

	Straightforward computations yield
	\begin{align*}
		| \sqrt{2} - p/q | = 2^{-i + 1} - \sum_{k \geq i} a_k 2^{-k} &\leq 2^{-i + 1} - \sum_{k=i}^{c \cdot i} 2^{-k}, \\
													&= 2^{-i + 1} - (2^{-i + 1} - 2^{-c \cdot i}), \\
													&= \frac{1}{(2q)^c}.
	\end{align*}
	By exercise 3, taking $c=2$ already yields a contradiction.
}

\exercise{
	Compute the first $4$ convergents of $\sqrt{2}$. Show that they are its first best rational approximations.
}{
	We begin by taking the integer part $\lfloor \sqrt{2} \rfloor = 1$, which yields the first approximation $p_0 / q_0 = 1/1$.

	Next, we let $\alpha$ be such that $\sqrt{2} = 1 + \frac{1}{\alpha}$, and continue the process with $\alpha$.
	Routing computation yields $\alpha = \sqrt{2} + 1$, which has integer part $2$.
	One thus has
		\[ p_1 / q_1 = 1 + \frac{1}{2} = \frac{3}{2}. \]
	
	Redefine $\alpha$ once again such that
		\[ \sqrt{2} + 1= 2 + \frac{1}{\alpha}, \]
	which once again yields $\alpha = \sqrt{2}+1$.

	The next approximation is thus
		\[ p_2/q_2 = 1 + \frac{1}{2 + \frac{1}{2}} = \frac{7}{5}. \]
	Alternatively, we easily identify that $a_n=2$ for all $n \geq 1$, which gives the following table, using the recursion formulas seen in class.

	\begin{center}
	\begin{tabular}{ |c|c|c|c|c|c|c|c| } 
	 \hline
	 $n$ & -2 & -1 & 0 & 1 & 2 & 3 & 4 \\ 
	 \hline
	 $a_n$ &  &  & 1 & 2 & 2 & 2 & 2 \\ 
	 \hline
	 $p_n$ & 0 & 1 & 1 & 3 & 7 & 17 & 41 \\ 
	 \hline
	 $q_n$ & 1 & 0 & 1 & 2 & 5 & 12 & 29 \\ 
	 \hline
	\end{tabular}
	\end{center}

}

\end{document}

%%% Local Variables:
%%% mode: latex
%%% TeX-master: t
%%% End:
