\documentclass[12pt,a4paper]{article}

\usepackage{amsmath}
\usepackage{amssymb}
\usepackage{amsfonts}
\usepackage{amsthm}
\usepackage{graphics}
\usepackage{fullpage}
\usepackage{graphicx}
\usepackage{caption}
\usepackage{subcaption} 
\usepackage{enumerate}
\usepackage{polynom}
\usepackage[utf8]{inputenc} 
\usepackage{utf8math}
\usepackage{xfrac}

\date{}

\usepackage[boldsans]{concmath}

\theoremstyle{plain}
\newtheorem{theorem}{Th\'eor\`eme}
\newtheorem{Sol}{Solution}
\newtheorem*{Sol*}{Solution}
\theoremstyle{definition}
\newtheorem{Ex}{Exercice}


\def \N {\mathbb N}
\def \Q {\mathbb Q}
\def \R {\mathbb R}
\def \Z {\mathbb Z}
\def \K {\mathbb K}
\def \C {\mathbb C}
\def \id {{\rm id}\,}
\def \Ker {{\rm Ker}\,}
\def \Im {{\rm Im}\,}
\def \Vect {{\rm Vect}\,}

\newcommand{\df}{\mathrel{\mathop:}=}
\newcommand{\dx}{ \ \text{d} \, x}

\newcommand{\pscal}[1]{\langle {#1} \rangle}
\DeclareMathOperator{\spa}{span}


%%%%%%%%%%%%%%%%%%%%%%%%%%%%%%%%%%%%%%%%%%%%%%%%%%%%%%%%%%%%%%%%%%%%%%%%%%%%%%%%%%
%%%%%%%%%%%%%%%%%%%%%%%%%%%%%%%%%%%%%%%%%%%%%%%%%%%%%%%%%%%%%%%%%%%%%%%%%%%%%%%%%%
%
%       ENABLE or DISABLE dislpay of solutions
% 
%%%%%%%%%%%%%%%%%%%%%%%%%%%%%%%%%%%%%%%%%%%%%%%%%%%%%%%%%%%%%%%%%%%%%%%%%%%%%%%%%%
%%%%%%%%%%%%%%%%%%%%%%%%%%%%%%%%%%%%%%%%%%%%%%%%%%%%%%%%%%%%%%%%%%%%%%%%%%%%%%%%%%
		\newif\ifsolutions
		
		% ENABLE or DISABLE display of solutions
		\solutionstrue
		%\solutionsfalse


%%%%%%%%%%%%%%%%%%%%%%%%%%%%%%%%%%%%%%%%%%%%%%%%%%%%%%%%%%%%%%%%%%%%%%%%%%%%%%%%%%
%		Dont touch much, just change the correct number and date :). Based on the setup above, the solutions will be automaticelly displayed or hidden.
%%%%%%%%%%%%%%%%%%%%%%%%%%%%%%%%%%%%%%%%%%%%%%%%%%%%%%%%%%%%%%%%%%%%%%%%%%%%%%%%%%
		\newcommand{\exercise}[2]{
			\begin{Ex} #1 \end{Ex}
			\ifsolutions  \begin{Sol*} #2 \end{Sol*} \bigskip \else \bigskip  \fi
		}

		

%%%%%%%%%%%%%%%%%%%%%%%%%%%%%%%%%%%%%%%%%%%%%%%%%%%%%%%%%%%%%%%%%%%%%%%%%%%%%%%%%%
%   Beginning of the document. Make sure to add the correct dates and numbers everywhere.
%%%%%%%%%%%%%%%%%%%%%%%%%%%%%%%%%%%%%%%%%%%%%%%%%%%%%%%%%%%%%%%%%%%%%%%%%%%%%%%%%%
\begin{document}

\begin{center}
{Prof. Friedrich Eisenbrand \hfill October 13th 2023}
\end{center}
	
\hrule\vspace{\baselineskip}

\begin{center}
\textbf{ Diophantine approximation}

Fall 2023

\bigskip

\textbf{Set 3}
\ifsolutions{\textbf{- Solutions}} \else{} \fi
\end{center}

\hrule\vspace{\baselineskip}


%%%%%%%%%%%%%%%%%%%%%%%%%%%%%%%%%%%%%%%%%%%%%%%%%%%%%%%%%%%%%%%%%%%%%%%%%%%%%%%%%%
%   Beginning of the exercises
%%%%%%%%%%%%%%%%%%%%%%%%%%%%%%%%%%%%%%%%%%%%%%%%%%%%%%%%%%%%%%%%%%%%%%%%%%%%%%%%%%

%%%%%%%%%%%%%%%%%%%%%%%%%%%%%%%%%%%%%%%%%%%%%%%%%%%%%%%%%%%%%%%%%%%%%%%%%%%%%%%%%%
%    Each exercise should look like 
% 		\exercise{ Exercise }{ Solution }
%%%%%%%%%%%%%%%%%%%%%%%%%%%%%%%%%%%%%%%%%%%%%%%%%%%%%%%%%%%%%%%%%%%%%%%%%%%%%%%%%%
\exercise{
	Let $d \in \N$ be a non square integer. Show that the equation
		\[ x^2 - dy^2 = N \]
	admits infinitely many integer solutions $(x,y)$ for some integer $N < 1 + 2 \sqrt{d}$.
}

\exercise{
	Let $\alpha \in \R\setminus\Q$. Show that if integers $p, q$ verify
		\[ | q \alpha - p | < \frac{1}{2q}, \]
	then $p/q$ is a best approximation of $\alpha$.
}
{
	Consider a fraction $r/s$ such that
		\[ | s \alpha - r | < |q \alpha - p | < \frac{1}{2q}. \]
	Then one has
		\[ | r/s - p/q | \leq | \alpha - r/s | + | \alpha - p/q | < \frac1{2qs} + \frac1{2q^2}.\]
	On the other hand, the difference of convergents is equal to
		\[| r/s - p/q | = | \frac{rq - ps}{sq} | \geq \frac{1}{sq}, \]
	provided that $rq - ps$ is nonzero. In fact, if that was the case, then the fractions $p/q$ and $r/s$ would be equal.

	Combining everything yields
		\[ \frac{1}{sq} \leq \frac1{2qs} + \frac1{2q^2}, \]
	which means that $q \leq s$.
}


\exercise{
	Consider the expansion $\alpha = [a_0; a_1, \dots]$ of some $\alpha \in \R\setminus\Q$, with convergents $\{p_{k}/q_{k}\}_k$ given by the truncated expansion \footnote{Recall we also defined $p_{-1} := 1$, and $q_{-1} := 0$.}.

	Define $x_k := [a_k; a_{k+1} \dots]$ for each $k\geq0$. Show that the sequence $\{x_k\}_k$ verifies
		\[ x_k = a_k + \frac{1}{x_{k+1}}, \]
	and that one has, for all $k\geq0$, 
		\[ \alpha = \frac{x_{k+1}p_k + p_{k-1}}{x_{k+1}q_k + q_{k-1}}. \]
}

\exercise{
	Consider the expansion $\alpha = [a_0; a_1, \dots]$ of some $\alpha \in \R\setminus\Q$.

	We say that the expansion of $a$ is \emph{eventually periodic} if there exist distinct integers $m, n$, such that
		\[ a_{m+k} = a_{n+k} \]
	for all $k\geq0$.

	Show, using the previous exercise, that if the expansion of $\alpha$ is eventually periodic, then $\alpha$ must be a quadratic irrationality (that is, the root of a degree 2 polynomial with integer coefficients).
}
{
	In the notations of exercise 3, we have by assumption that $x_m = x_n$ for some distinct integers $m,n$.

	Exercise 3 implies that
		\[ \alpha = \frac{x_{m}p_{m-1} + p_{m-2}}{x_{m}q_{m-1} + q_{m-2}} =  \frac{x_{m}p_{n-1} + p_{n-2}}{x_{m}q_{n-1} + q_{n-2}}. \]
	 Manipulating the expression yields a degree two polynomial to which $x_m$ is solution.

	The number $\alpha$ belongs to the field $\Q(x_m)$ which has degree at most 2. In fact it cannot have degree 1, since $\alpha$ was assumed irrational.
	As such, $\alpha$ is a quadratic irrationality.

}


%\exercise{
%	The following exercise will show the converse of exercise 4.
%	That is, all quadratic irrationalities $\alpha \in \R\setminus\Q$ admit eventually periodic expansions.
%	
%	Suppose to that end that there exist some $a, b \in \Q$ such that
%		\[ \alpha^2 + a \alpha + c = 0. \]
%
%	\begin{enumerate}
%		\item Argue that it is enough to show that, in the notation of exercise 3, $x_n = x_m$ for two distinct integers $n, m$.
%		\item Using exercise 3, show that the quantity $x_{n+1}$ verifies, for each $n\geq0$,
%			\[ Ax_{n+1}^2 + Bx_{n+1} + C = 0, \]
%		where
%			\begin{align*}
%			 A &= p_n^2 + a p_n q_n + b q_n^2,
%			\qquad C = p_{n-1}^2 + b q_{n_1}^2 + a p_{n-1}q_{n-1},\\
%			 B &= 2p_n p_{n-1} + 2b q_n q_{n-1} + a(p_n q_{n-1} + p_{n-1}q_{n}).
%			\end{align*}
%		\item Show that $|A|, |C| \leq |a| + 2|\alpha|+1$.
%		\item Show that $|B| \leq 6|\alpha| + 3|a| + 2$. (crude bounds are enough here)
%		\item Conclude that there exist two distinct integers $n, m$ such that $x_n = x_m$.
%	\end{enumerate}
%}


\exercise{
  We first recall the definition of best approximation. A fraction $p/q$  with $q ∈ ℕ_+$ and $p ∈ℤ$ is a best approximation of $α∈ℝ$ if for each $r/s$ with $s ∈ ℕ_+$, $s∈ ℤ$ one has: If $r/s ≠ p/q$ and $s ≤ q$, then
  \begin{displaymath}
    | α - p/q | < |α - r/s|.   
  \end{displaymath}

  Suppose that $α$ is not a fraction of the form $p/q$ with $q≤2 ⋅N^2$. 
  Show that there exists an $ε_0>0$ such that the set of best approximations with denominator bound $N$ of $α$ and those of  $α +ε$ are the same for each $0<ε < ε_0$. 
}{}

\exercise{Compute the best approximation of $α = 21/13$ with denominator bounded by $10$.}{}

\end{document}

%%% Local Variables:
%%% mode: latex
%%% TeX-master: t
%%% End:
