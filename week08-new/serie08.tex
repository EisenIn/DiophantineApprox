\documentclass[12pt,a4paper]{article}

\usepackage{amsmath}
\usepackage{amssymb}
\usepackage{amsfonts}
\usepackage{amsthm}
\usepackage{graphics}
\usepackage{fullpage}
\usepackage{graphicx}
\usepackage{caption}
\usepackage{subcaption} 
\usepackage{enumerate}
\usepackage{polynom}
\usepackage[utf8]{inputenc} 
\usepackage{utf8math}
\usepackage{xfrac}

\date{}

\usepackage[boldsans]{concmath}

\theoremstyle{plain}
\newtheorem{theorem}{Th\'eor\`eme}
\newtheorem{Sol}{Solution}
\newtheorem*{Sol*}{Solution}
\theoremstyle{definition}
\newtheorem{Ex}{Exercise}
\newtheorem{lemma}[theorem]{Lemma}


\def \N {\mathbb N}
\def \Q {\mathbb Q}
\def \R {\mathbb R}
\def \Z {\mathbb Z}
\def \K {\mathbb K}
\def \C {\mathbb C}
\def \F {\mathbb F}
\def \id {{\rm id}\,}
\def \Ker {{\rm Ker}\,}
\def \Im {{\rm Im}\,}
\def \Vect {{\rm Vect}\,}

\newcommand{\df}{\mathrel{\mathop:}=}
\newcommand{\dx}{ \ \text{d} \, x}

\newcommand{\pscal}[1]{\langle {#1} \rangle}
\DeclareMathOperator{\spa}{span}


%%%%%%%%%%%%%%%%%%%%%%%%%%%%%%%%%%%%%%%%%%%%%%%%%%%%%%%%%%%%%%%%%%%%%%%%%%%%%%%%%%
%%%%%%%%%%%%%%%%%%%%%%%%%%%%%%%%%%%%%%%%%%%%%%%%%%%%%%%%%%%%%%%%%%%%%%%%%%%%%%%%%%
%
%       ENABLE or DISABLE dislpay of solutions
% 
%%%%%%%%%%%%%%%%%%%%%%%%%%%%%%%%%%%%%%%%%%%%%%%%%%%%%%%%%%%%%%%%%%%%%%%%%%%%%%%%%%
%%%%%%%%%%%%%%%%%%%%%%%%%%%%%%%%%%%%%%%%%%%%%%%%%%%%%%%%%%%%%%%%%%%%%%%%%%%%%%%%%%
		\newif\ifsolutions
		
		% ENABLE or DISABLE display of solutions
		\solutionstrue
		%\solutionsfalse


%%%%%%%%%%%%%%%%%%%%%%%%%%%%%%%%%%%%%%%%%%%%%%%%%%%%%%%%%%%%%%%%%%%%%%%%%%%%%%%%%%
%		Dont touch much, just change the correct number and date :). Based on the setup above, the solutions will be automaticelly displayed or hidden.
%%%%%%%%%%%%%%%%%%%%%%%%%%%%%%%%%%%%%%%%%%%%%%%%%%%%%%%%%%%%%%%%%%%%%%%%%%%%%%%%%%
		\newcommand{\exercise}[2]{
			\begin{Ex} #1 \end{Ex}
			\ifsolutions  \begin{Sol*} #2 \end{Sol*} \bigskip \else \bigskip  \fi
		}

		

%%%%%%%%%%%%%%%%%%%%%%%%%%%%%%%%%%%%%%%%%%%%%%%%%%%%%%%%%%%%%%%%%%%%%%%%%%%%%%%%%%
%   Beginning of the document. Make sure to add the correct dates and numbers everywhere.
%%%%%%%%%%%%%%%%%%%%%%%%%%%%%%%%%%%%%%%%%%%%%%%%%%%%%%%%%%%%%%%%%%%%%%%%%%%%%%%%%%
\begin{document}

\begin{center}
{Prof. Friedrich Eisenbrand \hfill November 17th 2023}
\end{center}
	
\hrule\vspace{\baselineskip}

\begin{center}
\textbf{Diophantine approximation}

Fall 2023

\bigskip

\textbf{Set 8}
\ifsolutions{\textbf{- Solutions}} \else{} \fi
\end{center}

\hrule\vspace{\baselineskip}


%%%%%%%%%%%%%%%%%%%%%%%%%%%%%%%%%%%%%%%%%%%%%%%%%%%%%%%%%%%%%%%%%%%%%%%%%%%%%%%%%%
%   Beginning of the exercises
%%%%%%%%%%%%%%%%%%%%%%%%%%%%%%%%%%%%%%%%%%%%%%%%%%%%%%%%%%%%%%%%%%%%%%%%%%%%%%%%%%

%%%%%%%%%%%%%%%%%%%%%%%%%%%%%%%%%%%%%%%%%%%%%%%%%%%%%%%%%%%%%%%%%%%%%%%%%%%%%%%%%%
%    Each exercise should look like 
% 		\exercise{ Exercise }{ Solution }
%%%%%%%%%%%%%%%%%%%%%%%%%%%%%%%%%%%%%%%%%%%%%%%%%%%%%%%%%%%%%%%%%%%%%%%%%%%%%%%%%%

\exercise{Let $b_1,\dots,b_n ∈ℝ^n$ be linearly independent  and let $b_1^*,\dots,b_n^* ∈ ℝ^n$ be the Gram-Schmidt orthogonalization of $b_1,\dots,b_n$ where
  \begin{equation}
    \label{eq:1}
    b_j^* = b_j - ∑_{i=1}^{j-1}μ_{ij} b_i^*. 
  \end{equation}
  Let $c_1^*,\dots,c_n^*$ be the Gram-Schmidt orthogonalization of
  \begin{displaymath}
    b_1,\dots,b_{i-1},b_{i+1},b_i,b_{i+2},\dots,b_n.
  \end{displaymath}
  Show the following:
  \begin{enumerate}[i)]
  \item The $μ_{ij}$ in \eqref{eq:1} are unique.
  \item $c_k^* = b_k^*$ whenever $k ∉\{i,i+1\}$.
  \item $c_i^* = b_{i+1}^* + μ_{i,i+1} b_i^*$.
  \item $∏_{i=1}^n \|b_i^*\| = ∏_{i=1}^n \|c_i^*\|$. 
  \item $\|b_i^*\| ⋅\|b_{i+1}^*\| = \|c_i^*\| ⋅\|c_{i+1}^*\|$.
	\item $\| c_i^* \| \geq \| b_{i+1}^* \|$ and $\| c_{i+1}^* \| \leq \| b_{i}^* \|$.
	\item If $\frac{\| b_i \|}{\| b_i^* \|} > \frac{\| b_{i+1} \|}{\| b_{i+1}^* \|}$, then  $\frac{\| b_{i+1} \|}{\| c_i^* \|} < \frac{\| b_{i} \|}{\| c_{i+1}^* \|}$.
	We may thus reorder the basis such that $\frac{\| b_i \|}{\| b_i^* \|} \leq \frac{\| b_{i+1} \|}{\| b_{i+1}^* \|}$ for each $i=1, \dots, n$.
  \end{enumerate}

  }{
	The norms below are the $l^2$ norm unless specifed otherwise.
	\begin{enumerate}[i)]
		\item The coefficients of $\mu_{ij}$ of \eqref{eq:1} are that of the projection of $b_j$ onto the space ${\rm span}\{ b_1^*, \dots, b_{j-1}^* \}$. Note that this defines $b_j - b_j^*$ and hence $b_j^*$. 

		\item Clearly $c_k^* = b_k^*$ for $k < i$. For $k>i+1$, one defines $c_k - c_k^*$ by projecting $c_k$ onto ${\rm span}\{ c_1^*, \dots, c_{k-1}^* \}$. 
		By the definition of the $c_i$ and properties of the Gram-Schmidt orthogonalization, $b_k - c_k^*$ is the projection of $b_k$ onto ${\rm span}\{ b_1^*, \dots, b_{k-1}^* \}$.
		Therefore $c_k^* = b_k^*$ since they are defined in the same fashion.

		\item Note that the vector
			\[ v = b_{i+1}^* + μ_{i,i+1} b_i^* \]
		verifies that $v \perp b_1^*, \dots, v \perp b_{i-1}^*$.
		Furthermore, one has
			\[ (b_{i+1} - v)^T b_{k}^* = 0 \]
		for each $k = i, i+2, i+3, \dots, n$ (note that only $k=i$ requires computation).

		As such, the vector $b_{i+1} - v$ is the orthogonal projection of $b_{i+1}$ onto ${\rm span}\{ b_1^*, \dots, b_{i-1}^* \}$, which means that $v = c_{i}^*$.

		\item The Gram-Schmidt representation $B = B^* \cdot U$ where $U$ is upper triangular with $1$ on the diagonal shows that
			\[ \prod_{i=1}^n \| b_i^* \|_2 = |\det B^*| = | \det B|. \]
		The invariance of $|\det|$ under the swapping of columns concludes.

		\item Combining the last two points, we must have $\|b_i^*\| ⋅\|b_{i+1}^*\| = \|c_i^*\| ⋅\|c_{i+1}^*\|$.

		\item By point iii) and Pythagoras' theorem, we have
			\[ \| c_i^* \|^2_2 = \| b_{i+1}^* \|^2_2 + \mu_{i,i+1}^2 \| b_i^* \|^2_2 \geq  \| b_{i+1}^* \|^2_2. \]
		The relation v) gives the other inequality.

		\item If $\frac{\| b_i \|}{\| b_i^* \|} > \frac{\| b_{i+1} \|}{\| b_{i+1}^* \|}$, then one has
			\[ \frac{\| b_{i} \|}{\| c_{i+1}^* \|} > \frac{\| b_{i+1}\|}{\| b_{i+1}^* \|} \frac{\|b_i^* \|}{\| c_{i+1}^* \|} 
										= \frac{\| b_{i+1}\|}{\| b_{i+1}^* \|} \frac{\|c_{i}^* \|}{\| b_{i+1}^* \|} 
										> \frac{\| b_{i+1}\|}{\| b_{i+1}^* \|}, \]
		where we used v) and then vi).

	\end{enumerate}

}

  \exercise{
    Let $B ∈ℤ^{n×n}$ non-singular with Gram-Schmidt orthogonalization $B = B^*⋅ R$ and let $B^{(i)} ∈ℤ^{m ×i}$ be the matrix composed of the first $i$ columns of $B$.  Recall the definition of the \emph{potential} of $B$
    \begin{displaymath}
      Φ(B) = ∏_{i=1}^n \|b_i^*\|^{2(n-i+1)}.
    \end{displaymath}
    Show the following:
    \begin{enumerate}
    \item $ Φ(B) = ∏_{i=1}^n \det\left({B^{(i)}}^T B^{(i)}\right)$
    \item $ Φ(B) ∈ ℕ_+$  
    \end{enumerate}
  }{
	Note that $B^{(i)} = B_*^{(i)} U$ for some upper triangular matrix $U$ with diagonal $1$, and with $ B_*^{(i)}   = (b_1^*, \dots, b_i^*)$.

	Hence $(B_*^{(i)} )^T B_*^{(i)}  = {\rm diag}(\| b_1^* \|_2^2, \dots, \| b_i^* \|_2^2)$, and the determinant $\det\left({B^{(i)}}^T B^{(i)}\right)$ is found to be equal to $\prod_{i=1}^i \| b_i^* \|_2^2$.



}


  \exercise{Let $n≥2$. 
    Show that the LLL-algorithm provides on input $Q ∈ℕ_+$ and $α_1,\dots,α_n ∈ℚ$ (with suitable lattice basis)  a tuple $(q,p_1,\dots,p_n)$ such that
      \begin{enumerate}[i)] 
      \item $1 ≤ q < 2^n Q^n$ and
      \item $| q α_i - p_i | < 2^n/Q$ for $i=1,\dots,n$. 
      \end{enumerate}

    }{
	Seen in class.
}
  
\exercise{
	Let $B = \{b_1,\dots,b_n\} \subset \R^n$ be a linearly independent basis of $\R^n$. 
	Let $b_1^*,\dots,b_n^* ∈ ℝ^n$ be the Gram-Schmidt orthogonalization of $B$.
	Define the \emph{orthogonality defect} $\gamma(B)$ as
		\[ \gamma(B) := \prod_{i=1}^n \frac{\| b_i \|}{\| b_i^* \|}. \]
	Let $v \in \Lambda(B)$ be a shortest vector of the lattice generated by the elements of $B$.
	\begin{enumerate}[i)]
		\item Show that if $\gamma(B)=1$, then $B$ is an orthogonal basis.
		\item In the case that  $B$ is an orthogonal basis, show that $b_k$ is a shortest vector, where $k \in \arg\min_i \{ \| b_i \| \}$.
		\item Let $x$ be the coordinates of $v$ in the basis $B$. Show that
			\[ \| x \|_\infty \leq \gamma(B) \]
		using Cramer's rule and Hadamard's inequality.
		\item Deduce that the \emph{Shortest Vector Problem} can be solved in time $3^{\mathcal{O}(n)}\gamma(B)^n$ and polynomial space.
		
	\end{enumerate}
}
{
	The norms below are the $l^2$ norm unless specifed otherwise.

	\begin{enumerate}[i)]
		\item Note that one always has $\| b_i^* \| \leq \| b_i \|$ by equation \eqref{eq:1}.
		The relation $\gamma(B) = 1$ can only happen when $\| b_i^* \| = \| b_i \|$ for each $i=1, \dots, n$.
		In fact, equation \eqref{eq:1} also implies that $\| b_i^* \| = \| b_i \|$ if and only if $b_i^* = b_i$.

		\item If $B$ is orthogonal, then for any nonzero integer coordinates $x$, a vector
			\[ v = Bx = \sum_{i=1}^n b_i x_i \]
		has $l^2$ norm
			\[ \| v \|_2^2 = \sum_{i=1}^n x_i^2 \| b_i\|_2^2 \geq \| b_k \|_2^2, \]
		where $k$ belongs to the support of $x$.

		\item Let $v = Bx$ be a shortest vector of $B \cdot \Z^n$.
		Cramer's rule implies that
			\[ x_i = \frac{\det B^{(i)}}{\det B}, \]
		where $B^{(i)}$ has $i$-th column replaced by $v$.
		
		We have already shown that $|\det B| = \prod_{i=1}^n \| b_i^* \|_2$, so it only remains to show that
			\[ | \det B^{(i)} | \leq \prod_{i=1}^n  \| b_i \|_2. \]
		This is in fact implied by Hadamard's inequality, using that $\| v \| \leq \| b_i \|$ since $v$ is a shortest vector of $\Lambda$.

		\item We may enumerate all integer vectors in the $l^\infty$ ball of radius $\gamma(B)$. 
		There are at most $(2 \gamma(B) + 1)^n \leq 3^n \gamma(B)^n$ such candidates for $x$.
		Computing the image of the coordinates and comparing vectors takes quadratic time, swallowed by the $3^{\mathcal{O}(n)}$.
	\end{enumerate}


}

\exercise{
	Let $B = \{b_1,\dots,b_n\} \subset \R^n$ be a linearly independent basis of $\R^n$. 
	Let $B^* = \{b_1^*,\dots,b_n^*\} \subset \R^n$ be the Gram-Schmidt orthogonalization of $B$.
	Let $v \in \Lambda(B)$ be a shortest vector of the lattice generated by the elements of $B$.
	Define $c_i =  \frac{\| b_i \|}{\| b_i^* \|}$ for each $i = 1, \dots, n$, such that $\gamma(B) = \prod_{i=1}^n c_i$.
	\begin{enumerate}[i)]
		\item Let $v = Bx = B^*y$, where $x \in \Z^{n} \setminus \{0\}$, and $y \in \R^{n} \setminus \{0\}$.
		Show that
			\[ | y_i | \leq c_i \]
		for each $i=1, \dots, n$.
		\item Show that $ Rx = y$ for some $R \in \R^{n \times n}$ upper triangular with all diagonal elements equal to $1$.
		\item Deduce that 
			\[ | x_n | \leq c_n, \]
		and hence that there are at most $2c_n + 1$ possibilities for $x_n$.
		\item Show that, for all $i=1, \dots, n-1$, there are at most $2 c_i + 1$ possibilities for $x_i$ once the integers $x_{i+1}, \dots, x_n$ are fixed.
		\item Conclude that the \emph{Shortest Vector Problem} can be solved in time $n^{\mathcal{O}(n)} \gamma(B)$ and polynomial space.
	\end{enumerate}
}
{

	\begin{enumerate}[i)]
		\item The coordinates $y$ of $v$ in the orthogonal basis $B^*$ are given by
			\[ y_i = \frac{\langle v, b_i^* \rangle}{\langle b_i^*, b_i^* \rangle} \quad \text{ for each $i$}. \]
		The Cauchy-Schwarz inequality gives
			\[ | y_i | \leq \frac{\| v \|}{\| b_i^* \|} \leq c_i, \]
		as $v$ is a shortest vector.
		
		\item The matrix $R$ given by the Gram-Schmidt orthogonalization verifies $B = B^* R$, and hence gives $Rx = y.$

		\item The relation $Rx = y$ gives $x_n = y_n$, since $R$ is upper triangular with diagonal $1$.
		
		\item Fixing the integers $x_{i+1}, \dots, x_n$, the relation $Rx = y$ implies that 
			\[ x_i = y_i + F, \]
		where $F$ is some fixed quantity.
		The bound $| y_i | \leq c_i$ implies that $F - c_i \leq x_i \leq c_i + F$, which gives $2c_i + 1$ candidates for $x_i$.

		\item We enumerate the coordinates $x$ of a shortest vector from $x_n$ to $x_1$.
		 Note however that computing the lowerbound for the interval $x_i$ belongs to depending on $x_{i+1}, \dots, x_n$ takes linear time once the inverse $R^{-1}$ is computed (replacing $y$ by $-c$).

		Computing the image of each coordinate also takes linear time at each step, since we only modify one coordinate at a time. Accounting also for linear comparison, we get a runtime of order $\prod_{i=1}^n n (2c_i +1) \leq n^{\mathcal{O}(n)} \gamma(B)$.
		
	\end{enumerate}


}

\end{document}

%%% Local Variables:
%%% mode: latex
%%% TeX-master: t
%%% End:
