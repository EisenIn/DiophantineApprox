\documentclass[12pt,a4paper]{article}

\usepackage{amsmath}
\usepackage{amssymb}
\usepackage{amsfonts}
\usepackage{amsthm}
\usepackage{graphics}
\usepackage{fullpage}
\usepackage{graphicx}
\usepackage{caption}
\usepackage{subcaption} 
\usepackage{enumerate}
\usepackage{polynom}
\usepackage[utf8]{inputenc} 
\usepackage{utf8math}
\usepackage{xfrac}

\date{}

\usepackage[boldsans]{concmath}

\theoremstyle{plain}
\newtheorem{theorem}{Th\'eor\`eme}
\newtheorem{Sol}{Solution}
\newtheorem*{Sol*}{Solution}
\theoremstyle{definition}
\newtheorem{Ex}{Exercise}
\newtheorem{lemma}[theorem]{Lemma}


\def \N {\mathbb N}
\def \Q {\mathbb Q}
\def \R {\mathbb R}
\def \Z {\mathbb Z}
\def \K {\mathbb K}
\def \C {\mathbb C}
\def \F {\mathbb F}
\def \id {{\rm id}\,}
\def \Ker {{\rm Ker}\,}
\def \Im {{\rm Im}\,}
\def \Vect {{\rm Vect}\,}

\newcommand{\df}{\mathrel{\mathop:}=}
\newcommand{\dx}{ \ \text{d} \, x}

\newcommand{\pscal}[1]{\langle {#1} \rangle}
\DeclareMathOperator{\spa}{span}


%%%%%%%%%%%%%%%%%%%%%%%%%%%%%%%%%%%%%%%%%%%%%%%%%%%%%%%%%%%%%%%%%%%%%%%%%%%%%%%%%%
%%%%%%%%%%%%%%%%%%%%%%%%%%%%%%%%%%%%%%%%%%%%%%%%%%%%%%%%%%%%%%%%%%%%%%%%%%%%%%%%%%
%
%       ENABLE or DISABLE dislpay of solutions
% 
%%%%%%%%%%%%%%%%%%%%%%%%%%%%%%%%%%%%%%%%%%%%%%%%%%%%%%%%%%%%%%%%%%%%%%%%%%%%%%%%%%
%%%%%%%%%%%%%%%%%%%%%%%%%%%%%%%%%%%%%%%%%%%%%%%%%%%%%%%%%%%%%%%%%%%%%%%%%%%%%%%%%%
		\newif\ifsolutions
		
		% ENABLE or DISABLE display of solutions
		%\solutionstrue
		\solutionsfalse


%%%%%%%%%%%%%%%%%%%%%%%%%%%%%%%%%%%%%%%%%%%%%%%%%%%%%%%%%%%%%%%%%%%%%%%%%%%%%%%%%%
%		Dont touch much, just change the correct number and date :). Based on the setup above, the solutions will be automaticelly displayed or hidden.
%%%%%%%%%%%%%%%%%%%%%%%%%%%%%%%%%%%%%%%%%%%%%%%%%%%%%%%%%%%%%%%%%%%%%%%%%%%%%%%%%%
		\newcommand{\exercise}[2]{
			\begin{Ex} #1 \end{Ex}
			\ifsolutions  \begin{Sol*} #2 \end{Sol*} \bigskip \else \bigskip  \fi
		}

		

%%%%%%%%%%%%%%%%%%%%%%%%%%%%%%%%%%%%%%%%%%%%%%%%%%%%%%%%%%%%%%%%%%%%%%%%%%%%%%%%%%
%   Beginning of the document. Make sure to add the correct dates and numbers everywhere.
%%%%%%%%%%%%%%%%%%%%%%%%%%%%%%%%%%%%%%%%%%%%%%%%%%%%%%%%%%%%%%%%%%%%%%%%%%%%%%%%%%
\begin{document}

\begin{center}
{Prof. Friedrich Eisenbrand \hfill November 17th 2023}
\end{center}
	
\hrule\vspace{\baselineskip}

\begin{center}
\textbf{Diophantine approximation}

Fall 2023

\bigskip

\textbf{Set 8}
\ifsolutions{\textbf{- Solutions}} \else{} \fi
\end{center}

\hrule\vspace{\baselineskip}


%%%%%%%%%%%%%%%%%%%%%%%%%%%%%%%%%%%%%%%%%%%%%%%%%%%%%%%%%%%%%%%%%%%%%%%%%%%%%%%%%%
%   Beginning of the exercises
%%%%%%%%%%%%%%%%%%%%%%%%%%%%%%%%%%%%%%%%%%%%%%%%%%%%%%%%%%%%%%%%%%%%%%%%%%%%%%%%%%

%%%%%%%%%%%%%%%%%%%%%%%%%%%%%%%%%%%%%%%%%%%%%%%%%%%%%%%%%%%%%%%%%%%%%%%%%%%%%%%%%%
%    Each exercise should look like 
% 		\exercise{ Exercise }{ Solution }
%%%%%%%%%%%%%%%%%%%%%%%%%%%%%%%%%%%%%%%%%%%%%%%%%%%%%%%%%%%%%%%%%%%%%%%%%%%%%%%%%%

\exercise{Let $b_1,\dots,b_n ∈ℝ^n$ be linearly independent  and let $b_1^*,\dots,b_n^* ∈ ℝ^n$ be the Gram-Schmidt orthogonalization of $b_1,\dots,b_n$ where
  \begin{equation}
    \label{eq:1}
    b_j^* = b_j - ∑_{i=1}^{j-1}μ_{ij} b_i^*. 
  \end{equation}
  Let $c_1^*,\dots,c_n^*$ be the Gram-Schmidt orthogonalization of
  \begin{displaymath}
    b_1,\dots,b_{i-1},b_{i+1},b_i,b_{i+2},\dots,b_n.
  \end{displaymath}
  Show the following:
  \begin{enumerate}[i)]
  \item The $μ_{ij}$ in \eqref{eq:1} are unique.
  \item $c_k^* = b_k^*$ whenever $k ∉\{i,i+1\}$.
  \item $c_i^* = b_{i+1}^* + μ_{i,i+1} b_i^*$.
  \item $∏_{i=1}^n \|b_i^*\| = ∏_{i=1}^n \|c_i^*\|$. 
  \item $\|b_i^*\| ⋅\|b_{i+1}^*\| = \|c_i^*\| ⋅\|c_{i+1}^*\|$.
	\item $\| c_i^* \| \geq \| b_{i+1}^* \|$ and $\| c_{i+1}^* \| \leq \| b_{i}^* \|$.
	\item If $\frac{\| b_i \|}{\| b_i^* \|} > \frac{\| b_{i+1} \|}{\| b_{i+1}^* \|}$, then  $\frac{\| b_{i+1} \|}{\| c_i^* \|} < \frac{\| b_{i} \|}{\| c_{i+1}^* \|}$.
	We may thus reorder the basis such that $\frac{\| b_i \|}{\| b_i^* \|} \leq \frac{\| b_{i+1} \|}{\| b_{i+1}^* \|}$ for each $i=1, \dots, n$.
  \end{enumerate}

  }{}

  \exercise{
    Let $B ∈ℤ^{n×n}$ non-singular with Gram-Schmidt orthogonalization $B = B^*⋅ R$ and let $B^{(i)} ∈ℤ^{m ×i}$ be the matrix composed of the first $i$ columns of $B$.  Recall the definition of the \emph{potential} of $B$
    \begin{displaymath}
      Φ(B) = ∏_{i=1}^n \|b_i^*\|^{2(n-i+1)}.
    \end{displaymath}
    Show the following:
    \begin{enumerate}
    \item $ Φ(B) = ∏_{i=1}^n \det\left({B^{(i)}}^T B^{(i)}\right)$
    \item $ Φ(B) ∈ ℕ_+$  
    \end{enumerate}
  }{}


  \exercise{Let $n≥2$. 
    Show that the LLL-algorithm provides on input $Q ∈ℕ_+$ and $α_1,\dots,α_n ∈ℚ$ (with suitable lattice basis)  a tuple $(q,p_1,\dots,p_n)$ such that
      \begin{enumerate}[i)] 
      \item $1 ≤ q < 2^n Q^n$ and
      \item $| q α_i - p_i | < 2^n/Q$ for $i=1,\dots,n$. 
      \end{enumerate}

    }{}
  
\exercise{
	Let $B = \{b_1,\dots,b_n\} \subset \R^n$ be a linearly independent basis of $\R^n$. 
	Let $b_1^*,\dots,b_n^* ∈ ℝ^n$ be the Gram-Schmidt orthogonalization of $B$.
	Define the \emph{orthogonality defect} $\gamma(B)$ as
		\[ \gamma(B) := \prod_{i=1}^n \frac{\| b_i \|}{\| b_i^* \|}. \]
	Let $v \in \Lambda(B)$ be a shortest vector of the lattice generated by the elements of $B$.
	\begin{enumerate}[i)]
		\item Show that if $\gamma(B)=1$, then $B$ is an orthogonal basis.
		\item In the case that  $B$ is an orthogonal basis, show that $b_k$ is a shortest vector, where $k \in \arg\min_i \{ \| b_i \| \}$.
		\item Let $x$ be the coordinates of $v$ in the basis $B$. Show that
			\[ \| x \|_\infty \leq \gamma(B) \]
		using Cramer's rule and Hadamard's inequality.
		\item Deduce that the \emph{Shortest Vector Problem} can be solved in time $\mathcal{O}\left(3^n\gamma(B)^n\right)$ and polynomial space.
		
	\end{enumerate}
}
{}

\exercise{
	Let $B = \{b_1,\dots,b_n\} \subset \R^n$ be a linearly independent basis of $\R^n$. 
	Let $B^* = \{b_1^*,\dots,b_n^*\} \subset \R^n$ be the Gram-Schmidt orthogonalization of $B$.
	Let $v \in \Lambda(B)$ be a shortest vector of the lattice generated by the elements of $B$.
	Define $c_i =  \frac{\| b_i \|}{\| b_i^* \|}$ for each $i = 1, \dots, n$, such that $\gamma(B) = \prod_{i=1}^n c_i$.
	\begin{enumerate}[i)]
		\item Let $v = Bx = B^*y$, where $x \in \Z^{n} \setminus \{0\}$, and $y \in \R^{n} \setminus \{0\}$.
		Show that
			\[ | y_i | \leq c_i \]
		for each $i=1, \dots, n$.
		\item Show that $ Rx = y$ for some $R \in \R^{n \times n}$ upper triangular with all diagonal elements equal to $1$.
		\item Deduce that 
			\[ | x_n | \leq c_n, \]
		and hence that there are at most $2c_n + 1$ possibilities for $x_n$.
		\item Show that, for all $i=1, \dots, n-1$, there are at most $2 c_i + 1$ possibilities for $x_i$ once the integers $x_{i+1}, \dots, x_n$ are fixed.
		\item Conclude that the \emph{Shortest Vector Problem} can be solved in time $\mathcal{O}\left(3^n \gamma(B)\right)$ and polynomial space.
	\end{enumerate}
}
{}

\end{document}

%%% Local Variables:
%%% mode: latex
%%% TeX-master: t
%%% End:
