\documentclass[12pt,a4paper]{article}

\usepackage{amsmath}
\usepackage{amssymb}
\usepackage{amsfonts}
\usepackage{amsthm}
\usepackage{graphics}
\usepackage{fullpage}
\usepackage{graphicx}
\usepackage{caption}
\usepackage{subcaption} 
\usepackage{enumerate}
\usepackage{polynom}
\usepackage[utf8]{inputenc} 
\usepackage{utf8math}
\usepackage{xfrac}

\date{}

\usepackage[boldsans]{concmath}

\theoremstyle{plain}
\newtheorem{theorem}{Th\'eor\`eme}
\newtheorem{Sol}{Solution}
\newtheorem*{Sol*}{Solution}
\theoremstyle{definition}
\newtheorem{Ex}{Exercise}
\newtheorem{lemma}[theorem]{Lemma}


\def \N {\mathbb N}
\def \Q {\mathbb Q}
\def \R {\mathbb R}
\def \Z {\mathbb Z}
\def \K {\mathbb K}
\def \C {\mathbb C}
\def \id {{\rm id}\,}
\def \Ker {{\rm Ker}\,}
\def \Im {{\rm Im}\,}
\def \Vect {{\rm Vect}\,}

\newcommand{\df}{\mathrel{\mathop:}=}
\newcommand{\dx}{ \ \text{d} \, x}

\newcommand{\pscal}[1]{\langle {#1} \rangle}
\DeclareMathOperator{\spa}{span}


%%%%%%%%%%%%%%%%%%%%%%%%%%%%%%%%%%%%%%%%%%%%%%%%%%%%%%%%%%%%%%%%%%%%%%%%%%%%%%%%%%
%%%%%%%%%%%%%%%%%%%%%%%%%%%%%%%%%%%%%%%%%%%%%%%%%%%%%%%%%%%%%%%%%%%%%%%%%%%%%%%%%%
%
%       ENABLE or DISABLE dislpay of solutions
% 
%%%%%%%%%%%%%%%%%%%%%%%%%%%%%%%%%%%%%%%%%%%%%%%%%%%%%%%%%%%%%%%%%%%%%%%%%%%%%%%%%%
%%%%%%%%%%%%%%%%%%%%%%%%%%%%%%%%%%%%%%%%%%%%%%%%%%%%%%%%%%%%%%%%%%%%%%%%%%%%%%%%%%
		\newif\ifsolutions
		
		% ENABLE or DISABLE display of solutions
		%\solutionstrue
		\solutionsfalse


%%%%%%%%%%%%%%%%%%%%%%%%%%%%%%%%%%%%%%%%%%%%%%%%%%%%%%%%%%%%%%%%%%%%%%%%%%%%%%%%%%
%		Dont touch much, just change the correct number and date :). Based on the setup above, the solutions will be automaticelly displayed or hidden.
%%%%%%%%%%%%%%%%%%%%%%%%%%%%%%%%%%%%%%%%%%%%%%%%%%%%%%%%%%%%%%%%%%%%%%%%%%%%%%%%%%
		\newcommand{\exercise}[2]{
			\begin{Ex} #1 \end{Ex}
			\ifsolutions  \begin{Sol*} #2 \end{Sol*} \bigskip \else \bigskip  \fi
		}

		

%%%%%%%%%%%%%%%%%%%%%%%%%%%%%%%%%%%%%%%%%%%%%%%%%%%%%%%%%%%%%%%%%%%%%%%%%%%%%%%%%%
%   Beginning of the document. Make sure to add the correct dates and numbers everywhere.
%%%%%%%%%%%%%%%%%%%%%%%%%%%%%%%%%%%%%%%%%%%%%%%%%%%%%%%%%%%%%%%%%%%%%%%%%%%%%%%%%%
\begin{document}

\begin{center}
{Prof. Friedrich Eisenbrand \hfill November 3rd 2023}
\end{center}
	
\hrule\vspace{\baselineskip}

\begin{center}
\textbf{Diophantine approximation}

Fall 2023

\bigskip

\textbf{Set 6}
\ifsolutions{\textbf{- Solutions}} \else{} \fi
\end{center}

\hrule\vspace{\baselineskip}


%%%%%%%%%%%%%%%%%%%%%%%%%%%%%%%%%%%%%%%%%%%%%%%%%%%%%%%%%%%%%%%%%%%%%%%%%%%%%%%%%%
%   Beginning of the exercises
%%%%%%%%%%%%%%%%%%%%%%%%%%%%%%%%%%%%%%%%%%%%%%%%%%%%%%%%%%%%%%%%%%%%%%%%%%%%%%%%%%

%%%%%%%%%%%%%%%%%%%%%%%%%%%%%%%%%%%%%%%%%%%%%%%%%%%%%%%%%%%%%%%%%%%%%%%%%%%%%%%%%%
%    Each exercise should look like 
% 		\exercise{ Exercise }{ Solution }
%%%%%%%%%%%%%%%%%%%%%%%%%%%%%%%%%%%%%%%%%%%%%%%%%%%%%%%%%%%%%%%%%%%%%%%%%%%%%%%%%%

\exercise{
	Let $\mathcal{C} \subset \R^n$ be a \emph{compact}, convex set symmetric about the origin such that
		\[ {\rm vol}(\mathcal{C}) \geq 2^n.\]
	Show that $\mathcal{C}$ must containt a non-zero integer point.
}
{}

\exercise{
	Let $\mathcal{C} \subset \R^n$ be a convex set symmetric about the origin, and $A \in \R^{n \times n}$.
	Show that the set
		\[ A \cdot \mathcal{C} = \{ A u \ | \ u \in \mathcal{C} \} \]
	is also convex, symmetric about the origin.
}
{}

\exercise{
	  This exercise shows that the bound on $q≤ Q^n$  in Dirichlet's theorem is essentially tight.
	To this end, let $Q ∈ ℕ_+$ be an arbitrary integer.
	A tuple $(q,p_1,\dots,p_n) ∈ ℤ^{n+1}$ is said to \emph{cover} $α ∈ ℝ^n$ if $| q⋅α_i - p_i | ≤ 1/Q$ holds for each $i$. 
	We call $q$ the \emph{cost} of the tuple $(q,p_1,\dots,p_n) ∈ ℤ^{n+1}$. 
	  \begin{enumerate}
	  \item Let $S = \{ α ∈ ℝ^n : α \text{ covered by } (q,p_1,\dots,p_n)\}$. Show that ${\rm vol}(S) = \left( 2 / ( q ⋅ Q)\right)^n$.
	  \item Let  $α ∈ ℝ^n$ satisfy  $\|α\|_∞ ≤1$. If $α$ is covered by a tuple of cost $q$, then  $α$ is covered by a tuple  $(q,p_1,\dots,p_n) ∈ ℤ^{n+1}$  satisfying  $|p_i| ≤ q$ for each $i$. 
	  \item Use a covering argument to show that there exists $α ∈ [-1,1]^n$ that is not covered by any tuple $(q,p_1,\dots,p_n) ∈ ℤ^{n+1}$  of cost  $q< (Q/2)^n$.
	\item Derive a lower bound on ${\rm vol} (K)$ where 
	  \begin{displaymath}
	    K = \{α ∈ [-1,1]^n : α \text{ not covered with cost } ≤ (Q/4)^n\}. 
	  \end{displaymath} 

	  \end{enumerate}
}
{}

The following lemma is taken from Drmota, Tichy.  \emph{Sequences, discrepancies and applications}. Springer, 2006. Section 1.4.
\begin{lemma}[Khintchine's transference principle]\label{lem:1}
	Let $\alpha = (\alpha_1, \dots, \alpha_n) \in \R^n$ such that $1, \alpha_1, \dots, \alpha_n$ are linearly independent over $\Q$.
	The following properties are equivalent.
	\begin{enumerate}[i)]
		\item For every $c_1 > 0$, there are infinitely many integers $q \geq 1, p \in \Z^n$ with
			\[ \| q \alpha - p \|_\infty \leq \frac{c_1}{q^{1/n}}. \]
		\item For every $c_2 > 0$, there are infinitely many integers $x \in \Z^n, y \in \Z$ with
			\[ | x^T \alpha - y | \leq \frac{c_2}{ \| x \|_\infty^n}. \]
	\end{enumerate}
\end{lemma}

\exercise{
	The goal of this exercise is to show Khintchine's transference principle (lemma \ref{lem:1}) between the two versions of Dirichlet's theorem.

	First assume that i) is true, and consider integers $q \geq 1, p \in \Z^n$ with
		\[ \| q \alpha - p \|_\infty \leq \frac{c}{q^{1/n}}, \]
	and define the lattice 
		\[ \Lambda := \begin{pmatrix} q^{\frac{n+1}n} & q^{1/n} p^T \\ 0 & I_n \end{pmatrix} \cdot \Z^{n+1}.\]

	\begin{enumerate}
		\item Show that there exist integers $x \in \Z^n, y \in \Z$ with
			\[ \| x \|_\infty \leq q^{1/n}, \qquad \text{ and } \qquad qy = x^Tp.\]
		\item Deduce that, for these integers,
			\[  \| x \|_\infty \cdot | x^T \alpha - y | \leq n \cdot c_1. \]
		\item Conclude that ii) must hold using the linear independence of $\alpha$ over $\Q$.
	\end{enumerate}

	Next, assume that ii) is true, and let $x \in \Z^n, y \in \Z$ with
		\[ | x^T \alpha - y | \leq \frac{c_2}{ \| x \|_\infty^n}, \]
	and define the lattice 
		\[ \Lambda := \begin{pmatrix} y & & x^T \\ (Qc)\cdot\alpha_1 & & \\ \vdots & & (Qc) \cdot I_{n-1} \\ (Qc)\cdot\alpha_{n-1} & & \\ \frac{1}{Q^n c^{n-1}} & & 0^T \end{pmatrix} \cdot \Z^{n+1},\]
	where $Q := \| x \|_\infty = x_n$ without loss of generality, and $c > 1$ is a constant that will be chosen later.

	\begin{enumerate}
		\item Show that there exist integers $q \geq 1, p \in \Z^n$ verifying
			\[ | q \alpha_i - p_i | \leq \frac{1}{cQ}, \ \forall i=1, \dots, n-1, \qquad \text{ and } \qquad qy = x^T\alpha, \qquad \text{ and } \qquad q \leq Q^n c^{n-1}. \]
		\item Deduce that 
			\[ |q|^{1/n} \cdot | q \alpha_i - p_i | \leq \frac{1}{c^{1/n}}, \ \forall i=1, \dots, n-1 \qquad \text{ and } \qquad Q | q \alpha_n - p_n| \leq c_2 c^{n-1} + (n-1)\frac1c.\]
		\item Pick $c = c_2^{-1/n}$ and conclude that i) must hold with $c_1 = n \cdot c_2^{1/n^2}.$
	\end{enumerate}
}
{}

\exercise{
	Let $\alpha = (\alpha_1, \dots, \alpha_n) \in \R^n$. We say that $\alpha$ is \emph{badly approximable} if there exists a constant $c > 0$ such that
		\[ \| q \alpha - p \|_\infty > \frac{c}{q^{1/n}}, \]
	for all integers $q \in \Z, p \in \Z^n$.

	\begin{enumerate}[i)]
		\item Show that if $\alpha \in \R^n$ is badly approximable, then $1, \alpha_1, \dots, \alpha_n$ are linearly independent over $\Q$.
		\item Suppose that $\alpha \in \R$ is an algebraic integer of degree $n+1$. That is, there exist integers $a_0, \dots, a_{n} \in \Z$ such that
			\[ \alpha^{n+1} + a_{n} \alpha^{n} + \dots +  a_{0} = 0. \]
		Define $\alpha_1, \dots, \alpha_{n+1}$ the $n+1$ conjugate roots of the above polynomial.
		Show that the vector $(\alpha_1, \dots, \alpha_n) \in \R^n$ is badly approximable using lemma \ref{lem:1}ii).

		\emph{Hint: algebraic integers form a ring. Hence the minimal polynomial of $x^T \alpha$ is also an algebraic integer for $x \in Z^n$.}
	\end{enumerate}
}
{}

\end{document}

%%% Local Variables:
%%% mode: latex
%%% TeX-master: t
%%% End:
