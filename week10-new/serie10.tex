\documentclass[12pt,a4paper]{article}

\usepackage{amsmath}
\usepackage{amssymb}
\usepackage{amsfonts}
\usepackage{amsthm}
\usepackage{graphics}
\usepackage{fullpage}
\usepackage{graphicx}
\usepackage{caption}
\usepackage{subcaption} 
\usepackage{enumerate}
\usepackage{polynom}
\usepackage[utf8]{inputenc} 
\usepackage{utf8math}
\usepackage{xfrac}

\date{}

\usepackage[boldsans]{concmath}

\theoremstyle{plain}
\newtheorem{theorem}{Th\'eor\`eme}
\newtheorem{Sol}{Solution}
\newtheorem*{Sol*}{Solution}
\theoremstyle{definition}
\newtheorem{Ex}{Exercise}
\newtheorem{lemma}[theorem]{Lemma}


\def \N {\mathbb N}
\def \Q {\mathbb Q}
\def \R {\mathbb R}
\def \Z {\mathbb Z}
\def \K {\mathbb K}
\def \C {\mathbb C}
\def \F {\mathbb F}
\def \id {{\rm id}\,}
\def \Ker {{\rm Ker}\,}
\def \Im {{\rm Im}\,}
\def \Vect {{\rm Vect}\,}

\newcommand{\df}{\mathrel{\mathop:}=}
\newcommand{\dx}{ \ \text{d} \, x}

\newcommand{\pscal}[1]{\langle {#1} \rangle}
\DeclareMathOperator{\spa}{span}
\newcommand{\nint}[1]{\ensuremath{\lfloor#1\rceil}}


%%%%%%%%%%%%%%%%%%%%%%%%%%%%%%%%%%%%%%%%%%%%%%%%%%%%%%%%%%%%%%%%%%%%%%%%%%%%%%%%%%
%%%%%%%%%%%%%%%%%%%%%%%%%%%%%%%%%%%%%%%%%%%%%%%%%%%%%%%%%%%%%%%%%%%%%%%%%%%%%%%%%%
%
%       ENABLE or DISABLE dislpay of solutions
% 
%%%%%%%%%%%%%%%%%%%%%%%%%%%%%%%%%%%%%%%%%%%%%%%%%%%%%%%%%%%%%%%%%%%%%%%%%%%%%%%%%%
%%%%%%%%%%%%%%%%%%%%%%%%%%%%%%%%%%%%%%%%%%%%%%%%%%%%%%%%%%%%%%%%%%%%%%%%%%%%%%%%%%
		\newif\ifsolutions
		
		% ENABLE or DISABLE display of solutions
		%\solutionstrue
		\solutionsfalse


%%%%%%%%%%%%%%%%%%%%%%%%%%%%%%%%%%%%%%%%%%%%%%%%%%%%%%%%%%%%%%%%%%%%%%%%%%%%%%%%%%
%		Dont touch much, just change the correct number and date :). Based on the setup above, the solutions will be automaticelly displayed or hidden.
%%%%%%%%%%%%%%%%%%%%%%%%%%%%%%%%%%%%%%%%%%%%%%%%%%%%%%%%%%%%%%%%%%%%%%%%%%%%%%%%%%
		\newcommand{\exercise}[2]{
			\begin{Ex} #1 \end{Ex}
			\ifsolutions  \begin{Sol*} #2 \end{Sol*} \bigskip \else \bigskip  \fi
		}

		

%%%%%%%%%%%%%%%%%%%%%%%%%%%%%%%%%%%%%%%%%%%%%%%%%%%%%%%%%%%%%%%%%%%%%%%%%%%%%%%%%%
%   Beginning of the document. Make sure to add the correct dates and numbers everywhere.
%%%%%%%%%%%%%%%%%%%%%%%%%%%%%%%%%%%%%%%%%%%%%%%%%%%%%%%%%%%%%%%%%%%%%%%%%%%%%%%%%%
\begin{document}

\begin{center}
{Prof. Friedrich Eisenbrand \hfill December 1st 2023}
\end{center}
	
\hrule\vspace{\baselineskip}

\begin{center}
\textbf{Diophantine approximation}

Fall 2023

\bigskip

\textbf{Set 10}
\ifsolutions{\textbf{- Solutions}} \else{} \fi
\end{center}

\hrule\vspace{\baselineskip}


%%%%%%%%%%%%%%%%%%%%%%%%%%%%%%%%%%%%%%%%%%%%%%%%%%%%%%%%%%%%%%%%%%%%%%%%%%%%%%%%%%
%   Beginning of the exercises
%%%%%%%%%%%%%%%%%%%%%%%%%%%%%%%%%%%%%%%%%%%%%%%%%%%%%%%%%%%%%%%%%%%%%%%%%%%%%%%%%%

%%%%%%%%%%%%%%%%%%%%%%%%%%%%%%%%%%%%%%%%%%%%%%%%%%%%%%%%%%%%%%%%%%%%%%%%%%%%%%%%%%
%    Each exercise should look like 
% 		\exercise{ Exercise }{ Solution }
%%%%%%%%%%%%%%%%%%%%%%%%%%%%%%%%%%%%%%%%%%%%%%%%%%%%%%%%%%%%%%%%%%%%%%%%%%%%%%%%%%

\exercise{
	Let $\alpha \in \R^n$, and fix $N=2^n$, the number of sign vectors $\{\pm1\}^n$.

	Show, using the pigeonhole principle, that there exists a vector $y \in \{ 0, \pm 1\}^n \setminus \{ 0\}$ such that
		\[ | y^T \alpha | \leq \frac{\| \alpha \|_1 }{N - 1}. \]
}{}

\exercise{
	Let $\alpha = (\alpha_1, \dots, \alpha_n) \in \R^n$ be \emph{badly approximable}, that is there exists a constant $c > 0$ such that
		\[ \| q \alpha - p \|_\infty > \frac{c}{q^{1/n}}, \]
	for all integers $q \geq 1, p \in \Z^n$.

	The goal of this exercise is to evaluate the signed sum $| y^T \alpha |$ from below, where $y \in \{ \pm 1 \}^n$.

	Consider the lattice $\Lambda$, dependent on a real quantity $Q > 0$,
		\[ \Lambda \df \begin{pmatrix} \frac{1}{Q^{n+1}} & 0^T \\
								\alpha & I_n \end{pmatrix} \cdot \Z^{n+1}, \]
	with successive minima $\lambda_1, \dots, \lambda_{n+1}$ with respect to the $l^\infty$ norm.
	
	\begin{enumerate}[i)]
		\item 
		Show that, for all $\epsilon > 0$, one has
			\[ \frac{1}{Q^{1 + \epsilon}} \leq \lambda_1 \leq \frac{1}{Q}, \]
		for all $Q \geq c^{-1/\epsilon}$ large enough. Crude bounds are enough here.
		
		\item 
		Let $\epsilon = \frac1{2n}$. Deduce from Minkowski's second theorem that $\lambda_{n+1}$ is vanishing in $Q$, and that one may find $q \geq 1$, $p \in \Z^n$ such that
			\[ \| q \alpha - p \|_\infty^ \leq \frac{1}{2n}, \qquad \text{ and } \qquad y^T p \neq 0, \qquad \text{ and } \qquad q \leq \frac{(n+1)^{n^2 + 3n}}{c^{2n^2}}. \]
		
		\item
		Conclude that 
			\[ | y^T \alpha | \geq \frac{1}{2q} \geq \frac{c^{2n^2}}{2n^{n^2 + 3n}} . \]
	\end{enumerate}

}
{}

\exercise{
	Let $\alpha \in \R^n$ be a subspace-defining normal which splits the hypercube.
	That is, $\alpha^T x > 0$ or $\alpha^T x < 0$ for each $x \in \{0,1\}^n$.

	\begin{enumerate}[i)]
		\item Find the right lattice on which the LLL algorithm outputs integers $q \geq 1, p \in \Z^n$ such that
		\begin{gather*}
			q \leq 2^{n^2 + n \log(n) + 2n}, \\
			\| q \alpha - p \|_\infty \leq \frac{1}{2n}.
		\end{gather*}
		Deduce that, whenever $p^T x \neq 0$, one has ${\rm sign}(\alpha^T x) = {\rm sign}(p^Tx)$.
		
		\item
		In the case that $p^T x \neq 0$, find the constant $M$ such that the vector
			\[ v = M\cdot p + \begin{pmatrix}0 \\ v' \end{pmatrix} \]
		still verifies ${\rm sign}(\alpha^T x) = {\rm sign}(v^Tx)$ for any $\| v' \|_\infty \leq 2^{n^2}$.
	\end{enumerate}
}
{}

\end{document}

%%% Local Variables:
%%% mode: latex
%%% TeX-master: t
%%% End:
