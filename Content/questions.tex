\documentclass[a4paper,11pt,american]{article} 
 


\usepackage[utf8]{inputenc} 
\usepackage{utf8math}

%\usepackage[euler-digits,euler-hat-accent]{eulervm}
\usepackage[boldsans]{concmath}

\usepackage{mathrsfs}
\usepackage[american]{babel}
\usepackage{mathtools} % includes amsmath
\usepackage{amssymb}

\usepackage{amscd}
\usepackage{todonotes}

\usepackage{amsthm}

%\usepackage{multirow}
\usepackage{enumerate}

%\usepackage{tikz}
\usepackage{framed}
%\usepackage[colorlinks]{hyperref}
%\usepackage{showlabels}  


\newcommand{\smat}[1]{ \big(\begin{smallmatrix} #1 \end{smallmatrix}\big)}

\newcommand{\E}{\mathbb{E}}
\newcommand{\N}{\mathbb{N}}
\newcommand{\Q}{\mathbb{Q}}
\newcommand{\R}{\mathbb{R}}
\newcommand{\Z}{\mathbb{Z}}
\newcommand{\C}{\mathbb{C}}
\newcommand{\K}{\mathbb{K}}
\newcommand{\x}{\mathbf{x}}
\newcommand{\y}{\mathbf{y}}
\newcommand{\X}{\mathscr{X}}
\newcommand{\cA}{\mathcal{A}}
\newcommand{\cD}{\mathcal{D}}
\newcommand{\cO}{\mathcal{O}}
\newcommand{\cI}{\mathcal{I}}
\newcommand{\cP}{\mathcal{P}}
\newcommand{\cV}{\mathcal{V}}
\newcommand{\pscal}[1]{\langle {#1} \rangle}
\newcommand{\wt}[1]{\widetilde{#1}}
\newcommand{\car}{\mathrm{Char}}



\providecommand{\one}{\mathbf{1}}
\DeclareMathOperator{\vol}{vol}
\DeclareMathOperator{\rank}{rang}
\DeclareMathOperator{\noy}{noyau}
\DeclareMathOperator{\cone}{cone}
\DeclareMathOperator{\tcone}{tcone}
\DeclareMathOperator{\SV}{SV}
\DeclareMathOperator{\conv}{conv}
\DeclareMathOperator{\spec}{spec}
\DeclareMathOperator{\cof}{cof}
\DeclareMathOperator{\diam}{diam}
\DeclareMathOperator{\sign}{sgn}
\DeclareMathOperator{\poly}{poly}
\DeclareMathOperator{\Ker}{Ker}
\DeclareMathOperator{\Var}{Var}
\DeclareMathOperator{\Id}{Id}
\DeclareMathOperator{\tr}{tr}
\DeclareMathOperator{\relint}{relint}
\DeclareMathOperator{\spa}{span}
\DeclareMathOperator{\Tr}{Tr}
\DeclareMathOperator{\diag}{diag}
\DeclareMathOperator{\pspace}{\mathcal{PSPACE}}
\DeclarePairedDelimiter\mnorm{\lvert\lvert\lvert}{\rvert\rvert\rvert} % matrix norm



\theoremstyle{plain}
\newtheorem{theorem}{Theorem}
\newtheorem{lemma}[theorem]{Lemma}
\newtheorem{proposition}[theorem]{Proposition}
\newtheorem{claim}[theorem]{Claim}
\newtheorem{corollary}[theorem]{Corollary}

\theoremstyle{definition}
\newtheorem{definition}{Definition}
\newtheorem*{notation}{Notation}
\newtheorem{example}{Exemple}
\newtheorem{remark}[theorem]{Remark}
\newtheorem{problem}{Problem}

\newtheorem{exercise}{Exercise}
\newtheorem{algorithm}{Algorithm}

\setlength{\parindent}{0pt}  % No indents 

\newcommand{\iunit}{\mathrm{i}}
\newcommand{\CC}{{\mathbb C}}
\newcommand{\EE}{{\mathbb E}}
\newcommand{\FF}{{\mathbb F}}
\newcommand{\KK}{{\mathbb K}}
\newcommand{\NN}{{\mathbb N}}
\newcommand{\QQ}{{\mathbb Q}}
\newcommand{\RR}{{\mathbb R}}
\newcommand{\ZZ}{{\mathbb Z}}

\newcommand{\calB}{{\mathcal B}}


%opening

\title{Diophantine Approximation \\ Catalog of Questions}
\author{Friedrich Eisenbrand}

\begin{document}


\maketitle


\begin{enumerate}
\item Explain the notion of \emph{best approximation} of $α ∈ℝ$ with denominator bound $Q$.
\item Provide a proof of Dirichlet's theorem in dimension $1$ and explain in which sense the corresponding approximation is better than simple rounding. 
\item What is the continued fraction expansion of an irrational number $α ∈ ℝ \ ℚ$?
\item Prove the formula $[a_0,a_1,\dots,a_n] = p_n / q_n$  with 
  \begin{displaymath}
    \begin{pmatrix}
    p_n & p_{n-1} \\
    q_n & q_{n-1}
  \end{pmatrix} =
  \begin{pmatrix}
    a_0 & 1\\
    1 & 0
  \end{pmatrix} \cdots
  \begin{pmatrix}
    a_n & 1 \\
    1 & 0
  \end{pmatrix}
  \end{displaymath}
\item How can one find a best approximation with denominator bound $Q$ via continued fractions? How well can you bound $n$ in the corresponding expansion $[a_0,a_1,\dots,a_n]$? Compute the best approximation to $21/11$ with denominator bound $5$.  
\item When is a quadratic irrationality $α ∈ ℝ \setminus  ℚ$ \emph{reduced}? Why is the continued fraction expansion of a quadratic irrationality periodic?
\item What is Liouville's Theorem? Provide a proof.
\item How can one solve the Pell equation?
\item What is a lattice and what is Minkowski's first theorem. Provide a proof.
\item What are the successive minima of a lattice? Provide a proof of Minkowski's second theorem for $ℓ_2$. 
\item What is   Dirichlet's theorem on simultaneous Diophantine approximation? Provide a direct proof and a proof based on Minkowski's theorem. Why is the bound $q ≤Q^n$ essentially tight? (Set~6, Exercise~3)
\item Consider a half-space $ a^T x ≤ 0$  that divides the sets of $\pm1$ vectors into two subsets $S_1$ and $S_2⊆ \{\pm 1\}^n$. Explain how to replace $a$ by an integer vector $v$ with $\|v\|_∞  ≤ n^{O(n^2)}$ with the same division into $S_1$ and $S_2$ with Dirichlet's theorem. (Principle of Frank and Tardos).
\item What is the \emph{sums of square roots} problem? Which lower bounds can you derive on an expression
  \begin{displaymath}
    E = ∑_{i=1}^n y_i \sqrt{p_i}, 
  \end{displaymath}
  where $y_i ∈ \{\pm 1\}$ for $i=1,\dots,n$ and $p_1,\dots,p_n$ are distinct primes.
\item When is the basis of a lattice LLL reduced? How well does the first basis vector of an LLL-reduced lattice approximate the shortest vector w.r.t. $ℓ_2$?
\item What is the LLL algorithm and why does it terminate in time polynomial in the input encoding of the basis? 
\end{enumerate}


\bibliographystyle{plain}
\bibliography{references}
\end{document}


%%% Local Variables:
%%% mode: latex
%%% TeX-master: t
%%% End:


