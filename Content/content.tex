\documentclass[a4paper,11pt,american]{article} 
 


\usepackage[utf8]{inputenc} 
\usepackage{utf8math}

%\usepackage[euler-digits,euler-hat-accent]{eulervm}
\usepackage[boldsans]{concmath}

\usepackage{mathrsfs}
\usepackage[american]{babel}
\usepackage{mathtools} % includes amsmath
\usepackage{amssymb}

\usepackage{amscd}
\usepackage{todonotes}

\usepackage{amsthm}

%\usepackage{multirow}
\usepackage{enumerate}

%\usepackage{tikz}
\usepackage{framed}
%\usepackage[colorlinks]{hyperref}
%\usepackage{showlabels}  


\newcommand{\smat}[1]{ \big(\begin{smallmatrix} #1 \end{smallmatrix}\big)}

\newcommand{\E}{\mathbb{E}}
\newcommand{\N}{\mathbb{N}}
\newcommand{\Q}{\mathbb{Q}}
\newcommand{\R}{\mathbb{R}}
\newcommand{\Z}{\mathbb{Z}}
\newcommand{\C}{\mathbb{C}}
\newcommand{\K}{\mathbb{K}}
\newcommand{\x}{\mathbf{x}}
\newcommand{\y}{\mathbf{y}}
\newcommand{\X}{\mathscr{X}}
\newcommand{\cA}{\mathcal{A}}
\newcommand{\cD}{\mathcal{D}}
\newcommand{\cO}{\mathcal{O}}
\newcommand{\cI}{\mathcal{I}}
\newcommand{\cP}{\mathcal{P}}
\newcommand{\cV}{\mathcal{V}}
\newcommand{\pscal}[1]{\langle {#1} \rangle}
\newcommand{\wt}[1]{\widetilde{#1}}
\newcommand{\car}{\mathrm{Char}}



\providecommand{\one}{\mathbf{1}}
\DeclareMathOperator{\vol}{vol}
\DeclareMathOperator{\rank}{rang}
\DeclareMathOperator{\noy}{noyau}
\DeclareMathOperator{\cone}{cone}
\DeclareMathOperator{\tcone}{tcone}
\DeclareMathOperator{\SV}{SV}
\DeclareMathOperator{\conv}{conv}
\DeclareMathOperator{\spec}{spec}
\DeclareMathOperator{\cof}{cof}
\DeclareMathOperator{\diam}{diam}
\DeclareMathOperator{\sign}{sgn}
\DeclareMathOperator{\poly}{poly}
\DeclareMathOperator{\Ker}{Ker}
\DeclareMathOperator{\Var}{Var}
\DeclareMathOperator{\Id}{Id}
\DeclareMathOperator{\tr}{tr}
\DeclareMathOperator{\relint}{relint}
\DeclareMathOperator{\spa}{span}
\DeclareMathOperator{\Tr}{Tr}
\DeclareMathOperator{\diag}{diag}
\DeclareMathOperator{\pspace}{\mathcal{PSPACE}}
\DeclarePairedDelimiter\mnorm{\lvert\lvert\lvert}{\rvert\rvert\rvert} % matrix norm



\theoremstyle{plain}
\newtheorem{theorem}{Theorem}
\newtheorem{lemma}[theorem]{Lemma}
\newtheorem{proposition}[theorem]{Proposition}
\newtheorem{claim}[theorem]{Claim}
\newtheorem{corollary}[theorem]{Corollary}

\theoremstyle{definition}
\newtheorem{definition}{Definition}
\newtheorem*{notation}{Notation}
\newtheorem{example}{Exemple}
\newtheorem{remark}[theorem]{Remark}
\newtheorem{problem}{Problem}

\newtheorem{exercise}{Exercise}
\newtheorem{algorithm}{Algorithm}

\setlength{\parindent}{0pt}  % No indents 

\newcommand{\iunit}{\mathrm{i}}
\newcommand{\CC}{{\mathbb C}}
\newcommand{\EE}{{\mathbb E}}
\newcommand{\FF}{{\mathbb F}}
\newcommand{\KK}{{\mathbb K}}
\newcommand{\NN}{{\mathbb N}}
\newcommand{\QQ}{{\mathbb Q}}
\newcommand{\RR}{{\mathbb R}}
\newcommand{\ZZ}{{\mathbb Z}}

\newcommand{\calB}{{\mathcal B}}


%opening

\title{Diophantine Approximation}
\author{Friedrich Eisenbrand}

\begin{document}


\maketitle



\subsection*{22. September 2023}

Motivation for the neeed for approximation by rationals with small denominator: Calendar design.

Let $α∈ℝ$. A fraction $(p/q)$ with $p,q ∈ℤ$ and $q≥ 1$ is a \emph{best approximation} of $α$ if for each fraction $c/d ≠ p/q$  with $c,d ∈ℤ$ and $d≥ 1$ one has:
\begin{displaymath}
  0 < d ≤ q \text{ implies } \left|α - p/q \right| < \left|α - c/d \right|. 
\end{displaymath}

\emph{Rounding with denominator bound}: Given $α ∈ ℝ$ and $N ∈ ℕ_+$, find $(p/q)$ with $1 ≤ q ≤ n$ and  $| α - p/q|$ minimal.


\emph{Simple $N$-rounding}: Approximate $α$ with $ ⌊N ⋅ α ⌉ / N$. Guaranteed error $≤ 1/(2N)$.

\begin{theorem}[Dirichlet]
  \label{thr:1}
  Let $α ∈ ℝ$ and $N ∈ ℕ_+$. There exists a fraction $p/q$ with $p,q ∈ ℤ$ such that
  \begin{enumerate}[i)]
  \item $1 ≤ q ≤N$, and
  \item $|α - p/q| < 1/(q⋅N)$ 
  \end{enumerate}
\end{theorem}
  This is much better than $1/(2N)$ as soon as $q$ gets larger. The more you \emph{pay} for the denominator $q$, the better the approximation is.

  \begin{proof}[Proof of Dirichlet's Theorem] 
    For $0 ≤ q ≤ N$ consider the points $\{q ⋅ α\}$. These points lie in $[0,1)$ and there are $n+1$ many of them. This means that two are at distance $< 1/N$ from each other. Therefore, there exist $q_1,q_2 ∈ \{0,\dots,N\}$ such that $|\{q_1 ⋅ α\} -\{q_2 ⋅ α\} | < 1/N$. Since $\{q_i ⋅ α\} = q_i ⋅ α - x_i$  for some suitable $x_i ∈ℤ$ we have:
    \begin{displaymath}
          \left| (q_1 -q_2)⋅ α - (x_1-x_2) \right| < 1/N.
    \end{displaymath}
  \end{proof}
  This gives the theorem with $q = |q_1 -q_2| ∈ \{1,\dots,N\}$. 
  
Reading: \cite{eisenbrand2012pope,WikiDirichlet}  


\subsection*{29. September 2023}

Reading: The book of Khinchine~\cite[Section~2]{khinchin1997continued}.

Explain the geometric intuition behind the theory of continued fractions, as in \cite{eisenbrand2012pope}. 

Explain  continued fraction expansion of a number $α ∈ ℝ / ℚ$:

$α =  ⌊α ⌋ + \frac{1}{\frac{1}{α - ⌊α⌋}} $. Setting $β = \frac{1}{α - ⌊α⌋}$:
 
  \begin{displaymath}
    α = ⌊α⌋ + \cfrac{1}{ ⌊β⌋ + \cfrac{1}{⌊γ⌋ + \dots}} 
  \end{displaymath}

  Process does not end. If $α ∈ℚ$, process ends. Mention relation to Euclidean algorithm.

  Explain continued fraction expansion of $\sqrt{2}$.


  \begin{definition}
    For $a_0 ∈ ℝ$ and $a_1,\dots,a_n ∈ ℝ_+$ write
    \begin{equation}      
      [a_0;a_1,\dots,a_n] = a_0 + \cfrac{1}{a_1 + \cfrac{1}{a_2 + \cfrac{1}{\cdots  \cfrac{1}{a_n}}}}
      \label{eq:1}
    \end{equation}
  \end{definition}
Equation~\eqref{eq:1} is a fraction $p_n / q_n$ that depends on the sequence $a_0,\dots,a_n$.


\begin{theorem}
  \label{thr:2}
  Set $p_{-1} = 1, q_{-1} = 0, p_{-2} = 0, q_{-2} =1$. Then for $k≥1$ one has
  \begin{equation}
    \label{eq:2}
    \begin{array}{rcl}
    p_ k & = & p_{k-1} a_k + p_{k-2}\\
      q_ k & = & q_{k-1} a_k + q_{k-2}.
    \end{array}
  \end{equation}
  
\end{theorem}
This theorem gives a recursive formula to compute the value of~\eqref{eq:1}. In matrix notation one has
\begin{equation}
  \label{eq:3}
  \begin{pmatrix}
    p_k & p_{k-1} \\
    q_k & q_{k-1}
  \end{pmatrix} =
  \begin{pmatrix}
    a_0 & 1\\
    1 & 0
  \end{pmatrix} \cdots
  \begin{pmatrix}
    a_k & 1 \\
    1 & 0
  \end{pmatrix}
\end{equation}


Observation: If $a_i ∈ℕ_+$ for $i≥1$ and $a_0 ∈ℤ$, then $\gcd(p_k,q_k) = 1$.


\begin{theorem}
  \label{thr:3}
  For $k≥ -1$ one has
  \begin{displaymath}
    p_k q_{k-1} - q_k p_{k-1} = (-1)^{k+1} 
  \end{displaymath}
\end{theorem}

\begin{corollary}
  \label{co:1}
  \begin{equation}
    \label{eq:4}
      \frac{p_{k-1}}{q_{k-1}} - \frac{p_k}{q_k} = \frac{(-1)^k }{q_k q_{k-1}}.    
  \end{equation}

\end{corollary}

\begin{theorem}
  For all $k≥1$:
  \begin{displaymath}
    q_k p_{k-2} - p_k q_{k-2} = (-1)^ {k+1} a_k. 
  \end{displaymath}
\end{theorem}


\begin{corollary}
  \label{co:2}
  For all $k≥ 2$:

  \begin{displaymath}
    \frac{p_{k-2}}{q_{k-2}} -    \frac{p_k}{q_k} = \frac{(-1)^{k-1} a_k}{q_k q_{k-2}}. 
  \end{displaymath}
\end{corollary}

We always assume $a_0∈ℤ$ and $a_i ∈ ℕ_+$ for $i≥1$.





\subsubsection*{Convergence of convergents}

Then Corollary~\ref{co:2} implies that even convergents are increasing sequence and off convergents are decreasing. Together with Corollary~\ref{co:1} and the exponential growth of the sequence $p_k$ we have fast convergence of the convergents.



\subsection*{6. October 2023}
Solution to the best approximation problem.

\subsection*{13. October 2023}

We consider the Diophantine equation
\begin{equation}
  \label{eq:6}
  x^2 - d \,y^2  = 1, \, x,y ∈ℤ,
\end{equation}
where $d ∈ ℕ$. A trivial solution is $(x,y) = (\pm 1, 0)$.
\begin{exercise}
  \label{exe:1}
  Convince yourself of the fact that, if a non-trivial solution exists, then $d$ is not a perfect square. 
\end{exercise}

We also consider the field $Q(\sqrt{d}) = \{ x+ y \sqrt{d} : x,y ∈ℚ \}$ and  the ring $ℤ[ \sqrt{d}] = \{ x+ y \sqrt{d} : x,y ∈ℤ\}$.
\begin{exercise}
  \label{exe:2}
  Convince yourself that the above are indeed a field and a ring. 
\end{exercise}




The \emph{norm} of an element $x + y \sqrt{d} ∈ ℚ(\sqrt{d})$ is the rational number $(x + y \sqrt{d}) (x - y \sqrt{d} ) = x^2 -d y^2$. It is denoted by $N(x+y\sqrt{d})$. 
\begin{exercise}
  \label{exe:3}
  Show that the norm is multiplicative, i.e. that for $α,β ∈ℚ(\sqrt{d})$ one has $N(α⋅β) = N(α) ⋅N(β)$. 
\end{exercise}


If $(x,y) ∈ℤ^2$ is a solution to~\eqref{eq:6}, then $x + y \sqrt{d}$ is a \emph{unit} in $ℤ[ \sqrt{d}]$, since $(x + y \sqrt{d})(x - y \sqrt{d}) =1$. However, not each unit corresponds to a solution of \eqref{eq:6}. For example, $1 - \sqrt{2} ∈ℤ[\sqrt{2}]$ is  unit, but $1^2 - 2 ⋅ 1^2 = -1$. This is a \emph{unit of norm $-1$}.

There is a correspondence between units of norm $1$ and solutions of the Pell equation~\eqref{eq:6}. 



\begin{theorem}
  \label{thr:4}
  If $d ∈ ℕ_+$ is not a perfect square, then \eqref{eq:6} has a nontrivial solution. 
\end{theorem}


\begin{proof}
  We recall Dirichlet's Theorem~\ref{thr:1}. This implies that there exist infinitely many pairs $(p,q) ∈ ℕ × ℕ_+$ with $\gcd(p,q) = 1$ such that 
   \begin{displaymath}
    | p - q \sqrt{d}  | < 1/q. 
  \end{displaymath}

  For such $(p,q)$ one has, by the triangle inequality also  $ | p +  q \sqrt{d}  | ≤ 1/q + 2 q \sqrt{d} $. By multiplying both inequalities, one obtains
  \begin{displaymath}
    | p^2 - d q^2 | <  1/q^2 + 2 \sqrt{d} ≤ 1 + 2 \sqrt{d},
  \end{displaymath}
  which is a constant. Consequently, there exits a constant $r ∈ ℤ$
  and infinitely many pairs $(p,q) ∈(ℕ × ℕ_+)$ with
  \begin{equation}
    \label{eq:7}    
    p^2 - d q^2  = r.
  \end{equation}

  There are only $r^2$ many elements in $ℤ_r × ℤ_r$. This implies that
  there exists one such vector $(u,v) ∈ ℤ_r × ℤ_r$ and infinitely many
  pairs $(p,q) ∈(ℕ × ℕ_+)$ with $(p,q) ≡ (u,v) \pmod{r}$ that satisfy
  \eqref{eq:7}.


  Now let   $(p_1,q_1) ≠  (p_2,q_2) ∈(ℕ × ℕ_+)$ be two such pairs. We show that
  \begin{equation}
    \label{eq:8}   
  α =  (p_1 + \sqrt{d} q_1 ) (p_2 + \sqrt{d} q_2)^{-1} ∈ ℤ[\sqrt{d}]. 
  \end{equation}
  Since $N(α) = 1$, $α$ is a unit of norm one and if it is non-trivial, we are done, once we have established~\eqref{eq:8}.

  We have
  \begin{displaymath}
    α = \frac{p_1 p_2 - d q_1 q_2}{r} + \frac{-p_1q_2 + q_1p_2}{r} \sqrt{d}.  
  \end{displaymath} 
  Since $p_1 p_2 - d q_1 q_2 ≡ p_1^2 - d q_1^2 ≡ 0\pmod{r}$ and since $-p_1q_2 + q_1p_2 ≡ 0 \pmod{r}$, it follows that $α ∈ℤ[\sqrt{d}]$.  On the other hand, since $\gcd(p_1,q_1) =\gcd(p_2,q_2)=1$,   $p_1q_2 = q_1p_2$ implies $p_1 = p_2$ and $q_1 = q_2$ which is excluded. Thus $α$ is not rational and this yields a nontrivial solution of~\eqref{eq:6}, namely $x = (p_1 p_2 - d q_1 q_2)/r$ and $y = (-p_1q_2 + q_1p_2)/{r}$. 
\end{proof}

  \subsubsection*{Solving the Pell equation with continued fractions}


  If $d ∈ ℕ_+$ is not a perfect square, then $\sqrt{d}$ is irrational. One can compute the continued fraction expansion of $\sqrt{d}$. Recall that one has  \eqref{eq:4} from which we conclude
  \begin{displaymath}
    | \sqrt{d} - p_k / q_k | ≤ 1 (q_k q_{k+1}) ≤ 1/ q_k^2 
  \end{displaymath}
  and that the fraction $p_k/q_k$ are reduced for $k≥0$. This means that we can attractively compute solutions $(p_k^2 - d q_k^2) = r$ with $|r| ≤ 1 + 2 \sqrt{d}$. Once we have computed  $k ≥ ⌈ 2( 1  + 2\sqrt{d} ) +1 ⌉  ⋅ ⌈1 + 2 \sqrt{d}⌉^2 $ convergents,  we can identify two of them that yield $(p_1,q_1)$ and $(p_2,q_2)$ as in the proof of Theorem~\ref{thr:4} and construct a nontrivial solution of $x^2 - d y^2 = 1$. The running time of this naive algorithm is $O(d^{3/2})$  which is polynomial in the \emph{value } of  $d$ but \emph{exponential} in the \emph{binary encoding length} of $d$. It is also known that there exist $d$ such that each nontrivial solution of the Pell equation requires an exponential number of bits in $\ln (d)$, see~\cite{lenstra2002solving}. 


  \begin{example}
    \label{exe:4}

    We want to find a nontrivial solution of
    \begin{displaymath}
      x^2 - 3 y^2 = 1, \, x,y ∈ℤ. 
    \end{displaymath}
    The continued fraction expansion of $\sqrt{3} = [1; \overline{1,2}]$.    Starting with
    \begin{displaymath}
      \begin{pmatrix}
        p_{-1} & p_{-2}  \\
        q_{-1} & q_{-2}  
      \end{pmatrix}
      =
      \begin{pmatrix}
        1 & 0 \\
        0 & 1
      \end{pmatrix} 
    \end{displaymath}
    we compute
    \begin{displaymath}
      \begin{pmatrix}
        p_{0} & p_{-1}  \\
        q_{0} & q_{-1}  
      \end{pmatrix} =
      \begin{pmatrix}
        1 & 1  \\
        1 & 0  
      \end{pmatrix},
    \end{displaymath}
 \begin{displaymath}
      \begin{pmatrix}
        p_{1} & p_{0}  \\
        q_{1} & q_{0}  
      \end{pmatrix} =
      \begin{pmatrix}
        1 & 1  \\
        1 & 0  
      \end{pmatrix}
       \begin{pmatrix}
        1 & 1  \\
        1 & 0  
      \end{pmatrix}  =
      \begin{pmatrix}
        2 & 1  \\
        1 & 1  
      \end{pmatrix} 
    \end{displaymath}


    \begin{displaymath}
      \begin{pmatrix}
        p_{2} & p_{1}  \\
        q_{2} & q_{1}  
      \end{pmatrix} =
      \begin{pmatrix}
        2 & 1  \\
        1 & 1  
      \end{pmatrix}
       \begin{pmatrix}
        2 & 1  \\
        1 & 0  
      \end{pmatrix}  =
      \begin{pmatrix}
        5 & 2  \\
        3 & 1  
      \end{pmatrix} 
    \end{displaymath}
    and
    
    \begin{displaymath}
      \begin{pmatrix}
        p_{3} & p_{2}  \\
        q_{3} & q_{2}  
      \end{pmatrix} =
      \begin{pmatrix}
        5 & 2  \\
        3 & 1  
      \end{pmatrix}
       \begin{pmatrix}
        1 & 1  \\
        1 & 0  
      \end{pmatrix}  =
      \begin{pmatrix}
        7 & 5  \\
        4 & 3  
      \end{pmatrix}. 
    \end{displaymath}
    We find a non-trivial solution
    \begin{displaymath}
      (x,y) = (7,4)  ∈ℤ^2
    \end{displaymath}
    $7^2 - 3⋅ 4^2 = 1$. In this example, we did not search for convergents of same value and remainder pattern as in the proof. instead, we just computed some convergents until we found a solution. We have not argued that  this always works. 
  \end{example}


\bibliographystyle{alpha}
\bibliography{references}
\end{document}


%%% Local Variables:
%%% mode: latex
%%% TeX-master: t
%%% End:


