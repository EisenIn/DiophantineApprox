\documentclass[12pt,a4paper]{article}

\usepackage{amsmath}
\usepackage{amssymb}
\usepackage{amsfonts}
\usepackage{amsthm}
\usepackage{graphics}
\usepackage{fullpage}
\usepackage{graphicx}
\usepackage{caption}
\usepackage{subcaption} 
\usepackage{enumerate}
\usepackage{polynom}
\usepackage[utf8]{inputenc} 
\usepackage{utf8math}
\usepackage{xfrac}

\date{}

\usepackage[boldsans]{concmath}

\theoremstyle{plain}
\newtheorem{theorem}{Theorem}
\newtheorem{Sol}{Solution}
\newtheorem*{Sol*}{Solution}
\theoremstyle{definition}
\newtheorem{Ex}{Exercise}
\newtheorem{definition}{Definition}
\newtheorem{lemma}[theorem]{Lemma}


\def \N {\mathbb N}
\def \Q {\mathbb Q}
\def \R {\mathbb R}
\def \Z {\mathbb Z}
\def \K {\mathbb K}
\def \C {\mathbb C}
\def \F {\mathbb F}
\def \id {{\rm id}\,}
\def \Ker {{\rm Ker}\,}
\def \Im {{\rm Im}\,}
\def \Vect {{\rm Vect}\,}

\newcommand{\df}{\mathrel{\mathop:}=}
\newcommand{\dx}{ \ \text{d} \, x}

\newcommand{\pscal}[1]{\langle {#1} \rangle}
\DeclareMathOperator{\spa}{span}
\newcommand{\nint}[1]{\ensuremath{\lfloor#1\rceil}}


%%%%%%%%%%%%%%%%%%%%%%%%%%%%%%%%%%%%%%%%%%%%%%%%%%%%%%%%%%%%%%%%%%%%%%%%%%%%%%%%%%
%%%%%%%%%%%%%%%%%%%%%%%%%%%%%%%%%%%%%%%%%%%%%%%%%%%%%%%%%%%%%%%%%%%%%%%%%%%%%%%%%%
%
%       ENABLE or DISABLE dislpay of solutions
% 
%%%%%%%%%%%%%%%%%%%%%%%%%%%%%%%%%%%%%%%%%%%%%%%%%%%%%%%%%%%%%%%%%%%%%%%%%%%%%%%%%%
%%%%%%%%%%%%%%%%%%%%%%%%%%%%%%%%%%%%%%%%%%%%%%%%%%%%%%%%%%%%%%%%%%%%%%%%%%%%%%%%%%
		\newif\ifsolutions
		
		% ENABLE or DISABLE display of solutions
		%\solutionstrue
		\solutionsfalse


%%%%%%%%%%%%%%%%%%%%%%%%%%%%%%%%%%%%%%%%%%%%%%%%%%%%%%%%%%%%%%%%%%%%%%%%%%%%%%%%%%
%		Dont touch much, just change the correct number and date :). Based on the setup above, the solutions will be automaticelly displayed or hidden.
%%%%%%%%%%%%%%%%%%%%%%%%%%%%%%%%%%%%%%%%%%%%%%%%%%%%%%%%%%%%%%%%%%%%%%%%%%%%%%%%%%
		\newcommand{\exercise}[2]{
			\begin{Ex} #1 \end{Ex}
			\ifsolutions  \begin{Sol*} #2 \end{Sol*} \bigskip \else \bigskip  \fi
		}

		

%%%%%%%%%%%%%%%%%%%%%%%%%%%%%%%%%%%%%%%%%%%%%%%%%%%%%%%%%%%%%%%%%%%%%%%%%%%%%%%%%%
%   Beginning of the document. Make sure to add the correct dates and numbers everywhere.
%%%%%%%%%%%%%%%%%%%%%%%%%%%%%%%%%%%%%%%%%%%%%%%%%%%%%%%%%%%%%%%%%%%%%%%%%%%%%%%%%%
\begin{document}

\begin{center}
{Prof. Friedrich Eisenbrand \hfill December 8th 2023}
\end{center}
	
\hrule\vspace{\baselineskip}

\begin{center}
\textbf{Diophantine approximation}

Fall 2023

\bigskip

\textbf{Set 11}
\ifsolutions{\textbf{- Solutions}} \else{} \fi
\end{center}

\hrule\vspace{\baselineskip}


%%%%%%%%%%%%%%%%%%%%%%%%%%%%%%%%%%%%%%%%%%%%%%%%%%%%%%%%%%%%%%%%%%%%%%%%%%%%%%%%%%
%   Beginning of the exercises
%%%%%%%%%%%%%%%%%%%%%%%%%%%%%%%%%%%%%%%%%%%%%%%%%%%%%%%%%%%%%%%%%%%%%%%%%%%%%%%%%%

%%%%%%%%%%%%%%%%%%%%%%%%%%%%%%%%%%%%%%%%%%%%%%%%%%%%%%%%%%%%%%%%%%%%%%%%%%%%%%%%%%
%    Each exercise should look like 
% 		\exercise{ Exercise }{ Solution }
%%%%%%%%%%%%%%%%%%%%%%%%%%%%%%%%%%%%%%%%%%%%%%%%%%%%%%%%%%%%%%%%%%%%%%%%%%%%%%%%%%

\begin{definition}
	We call $L_1, \dots, L_n$ \emph{independent} linear forms in $n$ variables with real algebraic coefficients if, for some $\alpha_{11}, \dots, \alpha_{nn} \in \R$ algebraic,
		\[ L_i(x_1, \dots, x_n) = \alpha_{i1} x_1 + \dots \alpha_{in} x_n \qquad \text{ for each $i$},\]
	and such that
		\[ \det(L_1, \dots, L_n) := \det \left( \alpha_{ij} \right)_{i,j=1}^n \neq 0. \]
	
\end{definition}

\begin{theorem}[Schmidt, 1989]

	Let $n \geq 2$ be an integer.
	Let $L_1, \dots, L_n$ be linearly independent linear forms in $n$ variables with real algebraic coefficients.
	Let $\epsilon$ be a real number satisfying $0 < \epsilon < 1$.

	Then there are proper rational subspaces $S_1, \dots, S_t$ of $\Q^n$
	such that all nonzero integer solutions $x = (x_1, \dots, x_n)$ satisfying 
		\[ \left| L_1(x) \cdots L_n(x) \right| < \frac{ \left| \det(L_1, \dots, L_n) \right|  }{\| x\|_\infty^{\epsilon}} \]
	lie in the union of $S_1, \dots, S_t$.

\end{theorem}

\exercise{\label{ex:1}
	Let $n\geq1$ and $\alpha = (\alpha_1, \dots, \alpha_n) \in \R^n$ such that $1, \alpha_1, \dots, \alpha_n$ are linearly independent over $\Q$.

	We aim to show that for any $0 < \epsilon < \frac{1}{2n}$ the number of integers solutions $q\geq1, p\in\Z^n$ to
		\begin{gather}
			 \| q \alpha - p \|_\infty \leq \frac{1}{q^{1/n + \epsilon}} \label{eq:1}
		\end{gather}
	is finite.

	\begin{enumerate}[i)]
		\item Show, using the subspace theorem, that the integer solutions lie on finitely many subspaces of $\Q^{n+1}$.

		\item Consider a subspace $S \subset \Q^{n+1}$ and one of its \emph{unit} normal vector $v \in \Q^{n+1} \setminus\{0\}$.
			Show that, if $(q, p)$ belongs in $S$ and is a solution of \eqref{eq:1}, then
				\[ q \left| v^T \begin{pmatrix}1 \\ \alpha \end{pmatrix} \right| \leq 1, \]
			for $q$ large enough.
		
		\item Conclude that, in each subspace $S \subset \Q^{n+1}$, there are only finitely many solutions of \eqref{eq:1}.

	\end{enumerate}
}
{}


\exercise{
	Let $n\geq1$ and $\alpha = (\alpha_1, \dots, \alpha_n) \in \R^n$ such that $1, \alpha_1, \dots, \alpha_n$ are linearly \emph{dependent} over $\Q$.

	Fix an $0 < \epsilon < \frac{1}{n(n-1)}$.
	Show that there exist infinitely many integers $q\geq1, p\in\Z^n$ verfiying
		\begin{gather*}
			 \| q \alpha - p \|_\infty \leq \frac{1}{q^{1/n + \epsilon}}.
		\end{gather*}
}
{}

\exercise{

	Let $n\geq2$ and $\alpha = (\alpha_1, \dots, \alpha_n) \in \R^n$ such that $1, \alpha_1, \dots, \alpha_n$ are linearly independent over $\Q$. 
	Let $\epsilon$ satisfy $0 < \epsilon < \frac{1}{2n}$.

	Conclude from exercise \ref{ex:1} that there exists a constant $c > 0$ such that
		\[ \| q \alpha - p \|_\infty > \frac{c}{q^{1/n + \epsilon}}, \]
	for all integers $q \geq 1, p \in \Z^n$.

	\begin{enumerate}[i)]
		\item Deduce \emph{Roth's theorem}: for any $\alpha \in \R\setminus\Q$ and any $\epsilon>0$, there exists a constant $C=C(\epsilon)$ such that
			\[ | \alpha - p/q | \geq \frac{C}{q^{2^\epsilon}} \]
		for all $q\geq1, p\in\Z$.
		
		\item In the notation of exercise 2 of set 10, derive a similar bound on $\lambda_1$ of the type
			\[ \lambda_1 \geq \frac{1}{Q^{1+\delta}} \]
		for some $\delta$ which depends on $\epsilon$.

		\item Pick an adequate $\epsilon = \epsilon(n)$ such that 
			\[ \lambda_{n+1} \leq \frac{1}{Q^{1/2}}. \]
		
		\item Find a lower bound on $|y^T \alpha|$ as a function of $n$ and $C(n)$.

	\end{enumerate}
}
{}

\end{document}

%%% Local Variables:
%%% mode: latex
%%% TeX-master: t
%%% End:
