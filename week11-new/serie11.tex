\documentclass[12pt,a4paper]{article}

\usepackage{amsmath}
\usepackage{amssymb}
\usepackage{amsfonts}
\usepackage{amsthm}
\usepackage{graphics}
\usepackage{fullpage}
\usepackage{graphicx}
\usepackage{caption}
\usepackage{subcaption} 
\usepackage{enumerate}
\usepackage{polynom}
\usepackage[utf8]{inputenc} 
\usepackage{utf8math}
\usepackage{xfrac}

\date{}

\usepackage[boldsans]{concmath}

\theoremstyle{plain}
\newtheorem{theorem}{Theorem}
\newtheorem{Sol}{Solution}
\newtheorem*{Sol*}{Solution}
\theoremstyle{definition}
\newtheorem{Ex}{Exercise}
\newtheorem{definition}{Definition}
\newtheorem{lemma}[theorem]{Lemma}


\def \N {\mathbb N}
\def \Q {\mathbb Q}
\def \R {\mathbb R}
\def \Z {\mathbb Z}
\def \K {\mathbb K}
\def \C {\mathbb C}
\def \F {\mathbb F}
\def \id {{\rm id}\,}
\def \Ker {{\rm Ker}\,}
\def \Im {{\rm Im}\,}
\def \Vect {{\rm Vect}\,}

\newcommand{\df}{\mathrel{\mathop:}=}
\newcommand{\dx}{ \ \text{d} \, x}

\newcommand{\pscal}[1]{\langle {#1} \rangle}
\DeclareMathOperator{\spa}{span}
\newcommand{\nint}[1]{\ensuremath{\lfloor#1\rceil}}


\newcommand{\qp}{\begin{pmatrix} q \\ p \end{pmatrix}}
\newcommand{\onealpha}{\begin{pmatrix}1 \\ \alpha \end{pmatrix}}

%%%%%%%%%%%%%%%%%%%%%%%%%%%%%%%%%%%%%%%%%%%%%%%%%%%%%%%%%%%%%%%%%%%%%%%%%%%%%%%%%%
%%%%%%%%%%%%%%%%%%%%%%%%%%%%%%%%%%%%%%%%%%%%%%%%%%%%%%%%%%%%%%%%%%%%%%%%%%%%%%%%%%
%
%       ENABLE or DISABLE dislpay of solutions
% 
%%%%%%%%%%%%%%%%%%%%%%%%%%%%%%%%%%%%%%%%%%%%%%%%%%%%%%%%%%%%%%%%%%%%%%%%%%%%%%%%%%
%%%%%%%%%%%%%%%%%%%%%%%%%%%%%%%%%%%%%%%%%%%%%%%%%%%%%%%%%%%%%%%%%%%%%%%%%%%%%%%%%%
		\newif\ifsolutions
		
		% ENABLE or DISABLE display of solutions
		\solutionstrue
		%\solutionsfalse


%%%%%%%%%%%%%%%%%%%%%%%%%%%%%%%%%%%%%%%%%%%%%%%%%%%%%%%%%%%%%%%%%%%%%%%%%%%%%%%%%%
%		Dont touch much, just change the correct number and date :). Based on the setup above, the solutions will be automaticelly displayed or hidden.
%%%%%%%%%%%%%%%%%%%%%%%%%%%%%%%%%%%%%%%%%%%%%%%%%%%%%%%%%%%%%%%%%%%%%%%%%%%%%%%%%%
		\newcommand{\exercise}[2]{
			\begin{Ex} #1 \end{Ex}
			\ifsolutions  \begin{Sol*} #2 \end{Sol*} \bigskip \else \bigskip  \fi
		}

		

%%%%%%%%%%%%%%%%%%%%%%%%%%%%%%%%%%%%%%%%%%%%%%%%%%%%%%%%%%%%%%%%%%%%%%%%%%%%%%%%%%
%   Beginning of the document. Make sure to add the correct dates and numbers everywhere.
%%%%%%%%%%%%%%%%%%%%%%%%%%%%%%%%%%%%%%%%%%%%%%%%%%%%%%%%%%%%%%%%%%%%%%%%%%%%%%%%%%
\begin{document}

\begin{center}
{Prof. Friedrich Eisenbrand \hfill December 8th 2023}
\end{center}
	
\hrule\vspace{\baselineskip}

\begin{center}
\textbf{Diophantine approximation}

Fall 2023

\bigskip

\textbf{Set 11}
\ifsolutions{\textbf{- Solutions}} \else{} \fi
\end{center}

\hrule\vspace{\baselineskip}


%%%%%%%%%%%%%%%%%%%%%%%%%%%%%%%%%%%%%%%%%%%%%%%%%%%%%%%%%%%%%%%%%%%%%%%%%%%%%%%%%%
%   Beginning of the exercises
%%%%%%%%%%%%%%%%%%%%%%%%%%%%%%%%%%%%%%%%%%%%%%%%%%%%%%%%%%%%%%%%%%%%%%%%%%%%%%%%%%

%%%%%%%%%%%%%%%%%%%%%%%%%%%%%%%%%%%%%%%%%%%%%%%%%%%%%%%%%%%%%%%%%%%%%%%%%%%%%%%%%%
%    Each exercise should look like 
% 		\exercise{ Exercise }{ Solution }
%%%%%%%%%%%%%%%%%%%%%%%%%%%%%%%%%%%%%%%%%%%%%%%%%%%%%%%%%%%%%%%%%%%%%%%%%%%%%%%%%%

\begin{definition}
	We call $L_1, \dots, L_n$ \emph{independent} linear forms in $n$ variables with real algebraic coefficients if, for some $\alpha_{11}, \dots, \alpha_{nn} \in \R$ algebraic,
		\[ L_i(x_1, \dots, x_n) = \alpha_{i1} x_1 + \dots \alpha_{in} x_n \qquad \text{ for each $i$},\]
	and such that
		\[ \det(L_1, \dots, L_n) := \det \left( \alpha_{ij} \right)_{i,j=1}^n \neq 0. \]
	
\end{definition}

\begin{theorem}[Schmidt, 1989]

	Let $n \geq 2$ be an integer.
	Let $L_1, \dots, L_n$ be linearly independent linear forms in $n$ variables with real algebraic coefficients.
	Let $\epsilon > 0$.

	Then there are proper rational subspaces $S_1, \dots, S_t$ of $\Q^n$
	such that all nonzero integer solutions $x = (x_1, \dots, x_n)$ satisfying 
		\[ \left| L_1(x) \cdots L_n(x) \right| < \frac{ \left| \det(L_1, \dots, L_n) \right|  }{\| x\|_\infty^{\epsilon}} \]
	lie in the union of $S_1, \dots, S_t$.

\end{theorem}

\exercise{\label{ex:1}
	Let $n\geq1$ and $\alpha = (\alpha_1, \dots, \alpha_n) \in \R^n$ algebraic such that $1, \alpha_1, \dots, \alpha_n$ are linearly independent over $\Q$.

	We aim to show that for any $\epsilon > 0$ the number of integers solutions $q\geq1, p\in\Z^n$ to
		\begin{gather}
			 \| q \alpha - p \|_\infty \leq \frac{1}{q^{1/n + \epsilon}} \label{eq:1}
		\end{gather}
	is finite.

	\begin{enumerate}[i)]
		\item Show, using the subspace theorem, that the integer solutions lie on finitely many subspaces of $\Q^{n+1}$ (and possibly a ball around $0$).

		\item Consider a subspace $S \subset \Q^{n+1}$ and one of its \emph{unit} normal vector $v \in \Q^{n+1} \setminus\{0\}$.
			Show that, if $(q, p)$ belongs in $S$ and is a solution of \eqref{eq:1}, then
				\[ q \left| v^T \onealpha \right| \leq 1, \]
			for $q$ large enough.
		
		\item Conclude that, in each subspace $S \subset \Q^{n+1}$, there are only finitely many solutions of \eqref{eq:1}.

	\end{enumerate}
}
{
	\begin{enumerate}[i)]
		\item
		Define the linear forms $L_1, \dots, L_{n+1}$ depending on variables $x = (q, p_1, \dots, p_n)$ by
			\begin{align*}
			        & L_i(x) = q \beta_i - p_i \quad \text{ for } i = 1, \hdots n \\
			        & L_{n+1}(x) = q.
			\end{align*}
		Then for any vector $q \geq 1, p \in \Z^n$ satisfying $| q \beta_i - p_i | \leq \frac{1}{q^{1/n + \epsilon}}  \ \ \forall i=1, \dots, n$, 
			    \begin{gather*}
			    | L_1(x) \cdots L_{n+1}(x) | \leq \frac{1}{q^{2 \delta}}, \\
				\| p \|_\infty \leq C q,
			    \end{gather*}
		where $C = \max_i \{ | \beta_i |\}+1$, and  $\delta = \frac{n \epsilon}2 \in (0, 1)$.
	
		For such integer vectors with $\| x \|_\infty > C^{2}$, one thus has
			\[ | L_1(x) \cdots L_{n+1}(x) | \leq \frac{1}{\| x  \|_\infty^\delta}. \] 
	    
		Thus, by the subspace theorem, the integers $q \geq 1, p \in \Z^n$ satisfying the relation \eqref{eq:1} lie on the union of finitely many subspaces of $\Q^{n+1}$ and of the $l^\infty$ ball of radius $C^2$.

		\item 
		We know that $v^T \qp = 0$, as $v$ is normal to $S$.
		Hence,
			\begin{align*}
				| q v^T \onealpha | &= | q v^T \onealpha - v^T \qp |, \\
								& = | v^T \begin{pmatrix} 0 \\ q\alpha - p \end{pmatrix} |, \\
								&\leq\frac{n}{q^{1/n + \epsilon}} \leq 1, 
			\end{align*}
		where we used that $v$ is unit, and provided that $q$ is large enough.
		Do note that $v^T \onealpha \neq 0$ since $1, \alpha_1, \dots, \alpha_n$ were supposed linearly independent over $\Q$.

		\item 
		Point ii) implies that 
			\[ q \leq \left| v^T \onealpha \right|^{-1}, \]
		ie. that there are only finitely many $q$ in the subspace $S$.
	\end{enumerate}
}


\exercise{
	Let $n\geq1$ and $\alpha = (\alpha_1, \dots, \alpha_n) \in \R^n$ such that $1, \alpha_1, \dots, \alpha_n$ are linearly \emph{dependent} over $\Q$.

	Fix an $0 < \epsilon < \frac{1}{n(n-1)}$.
	Show that there exist infinitely many integers $q\geq1, p\in\Z^n$ verfiying
		\begin{gather*}
			 \| q \alpha - p \|_\infty \leq \frac{1}{q^{1/n + \epsilon}}.
		\end{gather*}
}
{
	Denote $\beta = (\alpha_2, \dots, \alpha_n) \in \R^{n-1}$.
	Without loss of generality, we may find an integer vector $v \in \Z^{n-1}$ such that
		\[ \alpha_1 = v^T \beta. \]

	We show that, by approximating $\beta$, we may approximate $\alpha_1$ as well.
	Dirichlet's theorem gives infinitely many $q \geq 1, \tilde{p} \in\Z^{n-1}$ such that
		\[ \| q \beta - p \|_\infty \leq \frac{1}{q^{1/(n-1)}}. \]
	We now note that $\epsilon$ was chosen such that
		\[ \frac{1}{q^{1/(n-1)}} < \frac{1}{q^{1/n + \epsilon}}, \]
	since $1/n + \epsilon < 1/(n-1)$.

	Let $p = v^T \tilde{p}$. Then
		\[ |q \alpha_1 - p| = |q v^T \beta - v^T \tilde{p}| \leq \frac{\| v \|_1}{q^{1/(n-1)}}. \]
	Finally, notice  that
		\[ \frac{\| v \|_1}{q^{1/(n-1)}} < \frac{1}{q^{1/n + \epsilon}} \]
	as soon as 
		\[ q \geq \| v \|_1^{\left( \frac{1}{n(n-1)} - \epsilon \right)^{-1}}, \]
	which must be true for infinitely many $q$.

	In conclusion, defining $p = (p_1, \tilde{p})$, we have found infinitely many $q \geq 1, p\in\Z^n$ such that
		\[ \| q \alpha - p \|_\infty \leq \frac{1}{q^{1/n + \epsilon}}. \]


}

\exercise{

	Let $n\geq2$ and $\alpha = (\alpha_1, \dots, \alpha_n) \in \R^n$ algebraic such that $1, \alpha_1, \dots, \alpha_n$ are linearly independent over $\Q$. 
	Let $\epsilon>0$.

	Conclude from exercise \ref{ex:1} that there exists a constant $c=c(\alpha, \epsilon) > 0$ such that
		\[ \| q \alpha - p \|_\infty > \frac{c}{q^{1/n + \epsilon}}, \]
	for all integers $q \geq 1, p \in \Z^n$.

	\begin{enumerate}[i)]
		\item Deduce \emph{Roth's theorem}: for any algebraic $\alpha \in \R\setminus\Q$ and any $\epsilon>0$, there exists a constant $C=C(\alpha, \epsilon)$ such that
			\[ | \alpha - p/q | \geq \frac{C}{q^{2+\epsilon}} \]
		for all $q\geq1, p\in\Z$.
		
		\item In the notation of exercise 2 of set 10, derive a similar bound on $\lambda_1$ of the type
			\[ \lambda_1 \geq \frac{1}{Q^{1+\delta}} \]
		for some $\delta$ which depends on $\epsilon$ ($\delta$ should tend to $0$ as $\epsilon$ does).

		\item Pick an adequate $\epsilon = \epsilon(n)$ such that 
			\[ \lambda_{n+1} \leq \frac{ (n+1)^{(n+1)/2} }{Q^{1/2}}. \]
		
		\item Find a lower bound on $|y^T \alpha|$ as a function of $n$ and $c(\alpha, n)$.
			This bound should be singly exponential in $n$.

	\end{enumerate}
}
{
	Exercise \ref{ex:1} says that there exist only finitely many integers $q \geq1, p\in\Z^n$ such that
		\[ \|q \alpha - p \|_\infty \leq \frac{1}{q^{1/n + \epsilon}}. \]
	In particular, the infimum
		\[ c(\alpha, \epsilon) = \inf_{q, p} \{  q^{1/n + \epsilon} \|q \alpha - p \|_\infty \} \]
	is positive.
	We may divide it by $2$ if we wish to get a strict inequality.

	\begin{enumerate}[i)]
		\item 
		Any $\alpha \in \R \setminus\Q$ is independent to $1$ over $\Q$.
		Applying exercise \ref{ex:1} with $n=1$ gives \emph{Roth's theorem} directly.

		\item
		Suppose that
			$ \lambda_1 < \frac{1}{Q^{1+\delta}} .$
		We will show that $Q < Q_0$ for $Q_0$ to be defined later.
	
		The assumption implies the existence of integers $(q, p) \in \N \times \Z^n$ such that
		\begin{align*}
			\frac{q}{Q^{n+1}} &\leq \frac{1}{Q^{1+\delta}}, \text{ and} \\
			\| q \alpha - p \|_\infty &\leq \frac{1}{Q^{1+\delta}}
		\end{align*}
		In turn, this gives a simultaneous approximation of the algebraic numbers $\alpha_1, \dots, \alpha_n$:
		\begin{align*}
			\| q \alpha - p \|_\infty \leq \frac{1}{Q^{1+\delta}},
		\end{align*}

		The definition of $c=c(\alpha, \epsilon)$ implies that, choosing $\delta = 2 n \epsilon$,
			\[  \frac{c}{Q^{1 + \delta/2}} \leq \frac{c}{q^{1/n + \epsilon}} \leq  \| q \alpha - p \|_\infty \leq \frac{1}{Q^{1+\delta}}, \]
		using that $q \leq Q^n$.

		This implies
			\[ Q \leq c^{-2/\delta} = c^{-1/(n \epsilon)} =: Q_0. \]

		\item 
		Minkowski's second theorem says that
			\[ \lambda_1^n \cdot \lambda_{n+1} \leq \prod_{i=1}^{n+1} \lambda_i \leq (n+1)^{(n+1)/2} \cdot \frac{1}{Q^{n+1}}. \]
		This implies that
			\[ \lambda_{n+1} \leq (n+1)^{(n+1)/2} \cdot \frac{1}{Q^{1-n\delta}} = (n+1)^{(n+1)/2} \cdot \frac{1}{Q^{1/2}}, \]
		choosing $n\delta = 1/2$, that is $\epsilon = 1/(4n^2)$.

		\item 
		Using the same reasoning as set 10, we wish to get a $Q$ such that
			\[ \lambda_{n+1} \leq \frac{1}{2n}. \]
		For this, $Q$ must verify $Q > Q_0 = c^{-1/(n\epsilon)} = c^{-4n}$ by question ii), and $Q > (2n)^2 (n+1)^{n+1}$, by question iii).

		We may therefore pick their product, $Q = (2n)^2 (n+1)^{n+1} c^{-4n}$.
		By linear independence, there exist integers $q \geq 1, p \in \Z^n$ such that
			\[ q \leq Q^n, \qquad \|q\alpha - p \|_\infty \leq \frac{1}{2n},  \text{ and } \qquad p^T y \neq 0. \]
		The triangle inequality concludes, since
			\[ | q y^T \alpha - y^T p | \leq \frac12. \]
		As $y^Tp$ is a nonzero integer, we must have that
			\[ q | y^T \alpha | \geq \frac12, \]
		which gives a lower bound on the signed sum, evaluted at
			\[ | y^T p | \geq \frac{1}{2q} \geq \frac{1}{2Q^n} = \frac{c^{4n^2}}{2 (2n)^{2n}  (n+1)^{n(n+1)}}. \]

	\end{enumerate}
}

\end{document}

%%% Local Variables:
%%% mode: latex
%%% TeX-master: t
%%% End:
