\documentclass[12pt,a4paper]{article}

\usepackage{amsmath}
\usepackage{amssymb}
\usepackage{amsfonts}
\usepackage{amsthm}
\usepackage{graphics}
\usepackage{fullpage}
\usepackage{graphicx}
\usepackage{caption}
\usepackage{subcaption} 
\usepackage{enumerate}
\usepackage{polynom}
\usepackage[utf8]{inputenc} 
\usepackage{utf8math}
\usepackage{xfrac}

\date{}

\usepackage[boldsans]{concmath}

\theoremstyle{plain}
\newtheorem{theorem}{Th\'eor\`eme}
\newtheorem{Sol}{Solution}
\newtheorem*{Sol*}{Solution}
\theoremstyle{definition}
\newtheorem{Ex}{Exercise}


\def \N {\mathbb N}
\def \Q {\mathbb Q}
\def \R {\mathbb R}
\def \Z {\mathbb Z}
\def \K {\mathbb K}
\def \C {\mathbb C}
\def \id {{\rm id}\,}
\def \Ker {{\rm Ker}\,}
\def \Im {{\rm Im}\,}
\def \Vect {{\rm Vect}\,}

\newcommand{\df}{\mathrel{\mathop:}=}
\newcommand{\dx}{ \ \text{d} \, x}

\newcommand{\pscal}[1]{\langle {#1} \rangle}
\DeclareMathOperator{\spa}{span}


%%%%%%%%%%%%%%%%%%%%%%%%%%%%%%%%%%%%%%%%%%%%%%%%%%%%%%%%%%%%%%%%%%%%%%%%%%%%%%%%%%
%%%%%%%%%%%%%%%%%%%%%%%%%%%%%%%%%%%%%%%%%%%%%%%%%%%%%%%%%%%%%%%%%%%%%%%%%%%%%%%%%%
%
%       ENABLE or DISABLE dislpay of solutions
% 
%%%%%%%%%%%%%%%%%%%%%%%%%%%%%%%%%%%%%%%%%%%%%%%%%%%%%%%%%%%%%%%%%%%%%%%%%%%%%%%%%%
%%%%%%%%%%%%%%%%%%%%%%%%%%%%%%%%%%%%%%%%%%%%%%%%%%%%%%%%%%%%%%%%%%%%%%%%%%%%%%%%%%
		\newif\ifsolutions
		
		% ENABLE or DISABLE display of solutions
		\solutionstrue
		%\solutionsfalse


%%%%%%%%%%%%%%%%%%%%%%%%%%%%%%%%%%%%%%%%%%%%%%%%%%%%%%%%%%%%%%%%%%%%%%%%%%%%%%%%%%
%		Dont touch much, just change the correct number and date :). Based on the setup above, the solutions will be automaticelly displayed or hidden.
%%%%%%%%%%%%%%%%%%%%%%%%%%%%%%%%%%%%%%%%%%%%%%%%%%%%%%%%%%%%%%%%%%%%%%%%%%%%%%%%%%
		\newcommand{\exercise}[2]{
			\begin{Ex} #1 \end{Ex}
			\ifsolutions  \begin{Sol*} #2 \end{Sol*} \bigskip \else \bigskip  \fi
		}

		

%%%%%%%%%%%%%%%%%%%%%%%%%%%%%%%%%%%%%%%%%%%%%%%%%%%%%%%%%%%%%%%%%%%%%%%%%%%%%%%%%%
%   Beginning of the document. Make sure to add the correct dates and numbers everywhere.
%%%%%%%%%%%%%%%%%%%%%%%%%%%%%%%%%%%%%%%%%%%%%%%%%%%%%%%%%%%%%%%%%%%%%%%%%%%%%%%%%%
\begin{document}

\begin{center}
{Prof. Friedrich Eisenbrand \hfill October 27th 2023}
\end{center}
	
\hrule\vspace{\baselineskip}

\begin{center}
\textbf{Diophantine approximation}

Fall 2023

\bigskip

\textbf{Set 5}
\ifsolutions{\textbf{- Solutions}} \else{} \fi
\end{center}

\hrule\vspace{\baselineskip}


%%%%%%%%%%%%%%%%%%%%%%%%%%%%%%%%%%%%%%%%%%%%%%%%%%%%%%%%%%%%%%%%%%%%%%%%%%%%%%%%%%
%   Beginning of the exercises
%%%%%%%%%%%%%%%%%%%%%%%%%%%%%%%%%%%%%%%%%%%%%%%%%%%%%%%%%%%%%%%%%%%%%%%%%%%%%%%%%%

%%%%%%%%%%%%%%%%%%%%%%%%%%%%%%%%%%%%%%%%%%%%%%%%%%%%%%%%%%%%%%%%%%%%%%%%%%%%%%%%%%
%    Each exercise should look like 
% 		\exercise{ Exercise }{ Solution }
%%%%%%%%%%%%%%%%%%%%%%%%%%%%%%%%%%%%%%%%%%%%%%%%%%%%%%%%%%%%%%%%%%%%%%%%%%%%%%%%%%

\exercise{
	Let $d$ be a non square integer.
	We study the equation 
	\begin{align}\label{eq:1} x^2 - dy^2 = k \end{align}
	for integer $k$.

	Recall that if $\alpha = x + y\sqrt{d} \in \Z[\sqrt{d}]$ has norm equal to $k$, then $(x,y)$ is a solution to \eqref{eq:1}.
	Further, for any element $\epsilon = u + v \sqrt{d} \in \N_+[\sqrt{d}]$ such that $N(\epsilon)=1$, then $\alpha \cdot \epsilon$ gives a solution to \eqref{eq:1} as well.

	\begin{enumerate}[i)]
		\item Assume that $x > 0$. Show that, up to inverting $\epsilon$, one has $v \cdot y < 0$, and that
			\[ \alpha \cdot \epsilon \in x \cdot \left(u+ \frac{vyd}{x} \right) + \sqrt{d} \cdot \Q. \]
		Deduce that
			\[ u+ \frac{vyd}{x} \leq u \left( 1 - \sqrt{1-\frac{k}{x^2}} \right).\]

		\item Prove that, whenever
			\[ x \geq \sqrt{\frac{k \cdot u}{2}}, \]
		one has
			\[ u+ \frac{vyd}{x} < 1. \]

		\item Conclude that the solutions of $\eqref{eq:1}$ each belong to an orbit $\{ \pm \alpha \epsilon^n \ | \ n \in \Z \}$, where $N(\epsilon)=1$, and $\alpha = x + y\sqrt{d}$ gives a solution to $\eqref{eq:1}$ which verifies $\sqrt{k} \leq x < \sqrt{\frac{k \cdot u}{2}}$.
		
		\item Express all integer solutions to the equation
			\[ x^2 - 5y^2 = 11. \]
	\end{enumerate}

}
{
	\begin{enumerate}[i)]
		\item 
		First, since $N(\epsilon)=1$, the inverse of $\epsilon$ is given by
			\[ \epsilon^{-1} = \overline{\epsilon} = u - v \sqrt{d}. \]
		Up to inverting $\epsilon$, one may thus suppose that $v \cdot y < 0$.

		Next, computing $(x+y\sqrt{d}) \cdot (u+v \sqrt{d})$ gives the required $\left(u+ \frac{vyd}{x} \right) + \sqrt{d} \cdot \Q$.

		Finally, one may use the relations
			\[ u^2 - dv^2 = 1, \qquad \text{ and } \qquad x^2 - dy^2 = k,\]
		in order to get
			\[ \sqrt{d} v = \pm \sqrt{u^2 - 1}, \qquad \text{ and } \qquad \sqrt{d} y = \pm \sqrt{x^2 - k}, \]
		which yields
			\[ \frac{vyd}{x} = - \frac1x \sqrt{u^2 - 1} \sqrt{x^2 - k} \leq \sqrt{1 - \frac{k}{x^2}}, \]
		and concludes.
		
		\item
		This is clear by plugging in the relation deduced in i).

		\item 
		First, we multiply $x$ by $\pm$ so that the assumption $x > 0$ is correct.
		Whenever $x \geq \sqrt{\frac{k \cdot u}{2}}$, the above results say that we may multiply by powers of $\epsilon^\pm$, decreasing the rational part.
		Notice further, that since $N(\alpha)=k$, we must have $x \geq \sqrt{k}$.

		\item
		A seen in set 4, we may take $\epsilon = 9 + 4\sqrt{5}$.
		We only have to check for solutions of the diophantine equations with $x$ values between $\sqrt{11} < 4$, and $\sqrt{\frac{11 \cdot 9}2} < 8$.
		The only solution is given by $\alpha = 4 + 1\sqrt{5}$.
	\end{enumerate}

}

\exercise{
 What is the distance of the line
through $(2,5)$ and $(20, 65)$ to the closest
point in $\mathbb{Z}^2$ not on this line?

}
{}

\exercise{
	 Show that there is no analogue to Pick's theorem in $\mathbb{R}^3$. In particular show that there are polyhedra with integer vertices and not other integer points that have arbitrarily large volume.

}
{
	These are given by the \emph{Reeve tetrahedra}.
	Its vertices are given by $0, (1,0,0), (0,1,0)$, and $(1,1,N)$, where $N \in \N$ is a large positive integer.
}

\exercise{ 
	Show that if $a$ and $b$ are two integers that can be written as a sum of two squares, then so can their product $ab$.	

	Furthermore, show that if an even integer $2n$ can be written as sum of two squares, then so can $n$.
}
{
	Firstly, a number is a sum of two squares if and only if it is the norm of a Gaussian integer (element of $\Z[i]$).
	Since $N(z \cdot w) = N(z) \cdot N(w)$, we conclude.

	Secondly, if $2n = a^2 + b^2$, then $a, b$ must have same parity.
	As such,
		\[ n = \left(\frac{a+b}2 \right)^2 + \left(\frac{a-b}2 \right)^2, \]
	and $n$ is the sum of two squares.
}

\exercise{
	We have see in class that every prime $p$ that is equal to $1$ modulo $4$ can be written as a sum of two squares, and that no integer that is equal to $3$ modulo $4$ can be written as a sum of two squares.

	Give a sufficient characterization of which natural numbers can be written as a sum of two squares given their prime factorization.
}
{
	Any positive integer $n$ with prime decomposition
		\[ n = 2^{n_2} \cdot \prod_{p \equiv 1 \mod{4}} p^{n_p} \cdot  \prod_{p \equiv 3 \mod{4}} p^{2n_p}, \]
	where $n_p \in \N_0$ for each prime $p$, may be written as the sum of two squares.

	To show that the characterization is necessary, one needs that whenever $ab$ is sum of two squares and $\gcd(a,b) = 1$, then both $a$ and $b$ are sum of two squares.
	The proof of this statement can be seen in an Algebraic Number Theory course.
}

\exercise{
	Let $P$ be a polygon with integer vertices in the plane. Show that if $P$ is not a triangle of area $1/2$, then we can find two integer points on the boundary of $P$ and a polygonal path (possibly just one segment) between them with integer vertices passing through the interior of $P$.

}
{}

\exercise{
	Let $u$ and $v$ be two vectors in the plane. Show that the area of the parallelogram
	$\{au+bv \mid ~0 \leq a, b \leq 1\}$ is equal to $|\det(u,v)|$.
  }
{}
\end{document}

%%% Local Variables:
%%% mode: latex
%%% TeX-master: t
%%% End:
