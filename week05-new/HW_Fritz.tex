\documentclass[12pt]{article}
\usepackage{graphicx,color}
\usepackage{epsfig}
\usepackage{amsmath}
\usepackage{amsthm, amssymb, latexsym}
\usepackage{amsfonts}

\textheight=20.3cm

\newtheorem{theorem}{Theorem}[section]
\newtheorem{lemma}[theorem]{Lemma}
\newtheorem{corollary}[theorem]{Corollary}
\newtheorem{definition}[theorem]{Definition}
\newtheorem{claim}[theorem]{Claim}

\renewcommand{\d}{\partial}

\newcommand{\grad}{\nabla}
\newcommand{\remove}[1]{}

\def \R {\mathrm{R}}
\def \C {\cal{R}}
\def \F {\cal{F}}
\def \conv {\mbox{conv}}
\def \la {\langle}
\def \ra {\rangle}
\renewcommand{\vec}[1]{{\overrightarrow{#1}}}

\begin{document}

\title{\bf Problem set}
\maketitle

\begin{enumerate}

\item What is the distance of the line
through $(2,5)$ and $(20, 65)$ to the closest
point in $\mathbb{Z}^2$ not on this line?

\item Show that there is no analogue to Pick's theorem in $\mathbb{R}^3$. In particular show that there are pyramids with integer vertices and not other integer points that have arbitrarily large volume.

\item Show that is $a$ and $b$ are two integers that can be written as a sum of two squares, then so is the integer $ab$.

\item We have see in class that every prime $p$ that is equal to $1$ modulo $4$ can be written as a sum of two squares, and that no integer that is equal to $2$ modulo $4$ can be written as a sum of two squares.

Give a full characterization of which natural numbers can be written as a sum of two squares given their prime factorization.

\item Let $P$ be a polygon with integer vertices in the plane. Show that if $P$ is not a triangle of area $1/2$, then we can find two integer points on the boundary of $P$ and a polygonal path (possibly just one segment) between them with integer vertices passing through the interior of $P$.

\item Let $u$ and $v$ be two vectors in the plane. Show that the area of the parallelogram
$\{au+bv \mid ~0 \leq a, b \leq 1\}$ is equal to $|det(u,v)|$.


\end{enumerate}


\noindent {\bf Week 5: Fritz' class: Minkowsky}

\begin{itemize}

\item State and prove Minkowsky's Theorem I in $\mathbb{R}^d$.

\item Show tighness.

\item Corollary: True also for lattices - Define. Linear transformations.

\item Define a convex set + examples.
\item $C$ is convex iff $a,b \in A$ implies
$\lambda a +(1-\lambda)b \in C$.

\item Minkowsky's theorem II: If $C$ is centerally symmetric and convex around $O$, with area $>4$, then it contains an integer point different than $O$.

\item Proof: consider $\frac{1}{2}C$.
We have $x, y \in \frac{1}{2}C$ such that 
$x-y \in \mathbb{Z}^2$ and is not $O$.
Notice: $x-y=\frac{1}{2}a-\frac{1}{2}b=
\frac{1}{2}a+\frac{1}{2}(-b) \in C$.

\item Example: The area of a triangle with integer vertices and no other integer point is 
$\frac{1}{2}$.

\item Proof: We can assume without loss of generality that $a,b,O$ are the three vertices. Then $a+b,a-b,b-a,-a-b$ are vertices of a parallelogram whose only other integer point is
the origin. We conclude that the area of the triangle $\Delta Oab$ is less than or equal to $\frac{1}{2}$. It now follows that it is equal to $\frac{1}{2}$.

Consequences:

\item Pick's theorem. By induction.

\item Farey sequences. Choose $n$.
Then write all the numbers in $(0,1)$ with denominator
smaller than or equal to $n$ in co-primes form
$p/q$.
Then $q_{i+1}p_{i}-q_{i}p_{i+1}=1$.
Give example with $n=6$.

\item Most classical example: A prime
$P$ can be written as a sum of two squares iff $P \equiv 1 (mod 4)$.

\item One direction is clear. Sum of two squares cannot be $3$ modulo $4$.

\item If $P \equiv 1 (mod 4)$, then 
$-1$ is a square. This is because 
$(p-1)!=-1 (mod p)$ for any prime $p$.
Next we arrange the numbers $1, \ldots, p-1$
in pairs of $a, -a$. number of pairs
$(p-1)/2$. it is even if $p \equiv 1 (mod 4)$,
showing that $-1$ is a square.
If $p \equiv 3 (mod 4)$, then $-1$ cannot be a square because a square is always $0,1$ mod $4$
and $p-1$ would be $2$ mod $4$.

\item Let $a^2=-1$ and consider the set 
$\{(x,y) \in \mathbb{R}^2 \mid 
(px+ay)^2+y^2<2p$.
The area of this set is $2\pi>4$.
Take an integer point $(x,y)$. 
Then it must be that $(px+ay)^2+y^2=p$ because
it is equal to $0$ mod $p$ and not equal to $0$.

\item It is enough to show this for primes.
This because if $a$ and $b$ can be written as sum of two squares, then so is $ab$.
Also $2=1^2+1^2$





\end{itemize}

\end{document}