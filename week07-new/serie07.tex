\documentclass[12pt,a4paper]{article}

\usepackage{amsmath}
\usepackage{amssymb}
\usepackage{amsfonts}
\usepackage{amsthm}
\usepackage{graphics}
\usepackage{fullpage}
\usepackage{graphicx}
\usepackage{caption}
\usepackage{subcaption} 
\usepackage{enumerate}
\usepackage{polynom}
\usepackage[utf8]{inputenc} 
\usepackage{utf8math}
\usepackage{xfrac}

\date{}

\usepackage[boldsans]{concmath}

\theoremstyle{plain}
\newtheorem{theorem}{Th\'eor\`eme}
\newtheorem{Sol}{Solution}
\newtheorem*{Sol*}{Solution}
\theoremstyle{definition}
\newtheorem{Ex}{Exercise}
\newtheorem{lemma}[theorem]{Lemma}


\def \N {\mathbb N}
\def \Q {\mathbb Q}
\def \R {\mathbb R}
\def \Z {\mathbb Z}
\def \K {\mathbb K}
\def \C {\mathbb C}
\def \id {{\rm id}\,}
\def \Ker {{\rm Ker}\,}
\def \Im {{\rm Im}\,}
\def \Vect {{\rm Vect}\,}

\newcommand{\df}{\mathrel{\mathop:}=}
\newcommand{\dx}{ \ \text{d} \, x}

\newcommand{\pscal}[1]{\langle {#1} \rangle}
\DeclareMathOperator{\spa}{span}


%%%%%%%%%%%%%%%%%%%%%%%%%%%%%%%%%%%%%%%%%%%%%%%%%%%%%%%%%%%%%%%%%%%%%%%%%%%%%%%%%%
%%%%%%%%%%%%%%%%%%%%%%%%%%%%%%%%%%%%%%%%%%%%%%%%%%%%%%%%%%%%%%%%%%%%%%%%%%%%%%%%%%
%
%       ENABLE or DISABLE dislpay of solutions
% 
%%%%%%%%%%%%%%%%%%%%%%%%%%%%%%%%%%%%%%%%%%%%%%%%%%%%%%%%%%%%%%%%%%%%%%%%%%%%%%%%%%
%%%%%%%%%%%%%%%%%%%%%%%%%%%%%%%%%%%%%%%%%%%%%%%%%%%%%%%%%%%%%%%%%%%%%%%%%%%%%%%%%%
		\newif\ifsolutions
		
		% ENABLE or DISABLE display of solutions
		%\solutionstrue
		\solutionsfalse


%%%%%%%%%%%%%%%%%%%%%%%%%%%%%%%%%%%%%%%%%%%%%%%%%%%%%%%%%%%%%%%%%%%%%%%%%%%%%%%%%%
%		Dont touch much, just change the correct number and date :). Based on the setup above, the solutions will be automaticelly displayed or hidden.
%%%%%%%%%%%%%%%%%%%%%%%%%%%%%%%%%%%%%%%%%%%%%%%%%%%%%%%%%%%%%%%%%%%%%%%%%%%%%%%%%%
		\newcommand{\exercise}[2]{
			\begin{Ex} #1 \end{Ex}
			\ifsolutions  \begin{Sol*} #2 \end{Sol*} \bigskip \else \bigskip  \fi
		}

		

%%%%%%%%%%%%%%%%%%%%%%%%%%%%%%%%%%%%%%%%%%%%%%%%%%%%%%%%%%%%%%%%%%%%%%%%%%%%%%%%%%
%   Beginning of the document. Make sure to add the correct dates and numbers everywhere.
%%%%%%%%%%%%%%%%%%%%%%%%%%%%%%%%%%%%%%%%%%%%%%%%%%%%%%%%%%%%%%%%%%%%%%%%%%%%%%%%%%
\begin{document}

\begin{center}
{Prof. Friedrich Eisenbrand \hfill November 10th 2023}
\end{center}
	
\hrule\vspace{\baselineskip}

\begin{center}
\textbf{Diophantine approximation}

Fall 2023

\bigskip

\textbf{Set 7}
\ifsolutions{\textbf{- Solutions}} \else{} \fi
\end{center}

\hrule\vspace{\baselineskip}


%%%%%%%%%%%%%%%%%%%%%%%%%%%%%%%%%%%%%%%%%%%%%%%%%%%%%%%%%%%%%%%%%%%%%%%%%%%%%%%%%%
%   Beginning of the exercises
%%%%%%%%%%%%%%%%%%%%%%%%%%%%%%%%%%%%%%%%%%%%%%%%%%%%%%%%%%%%%%%%%%%%%%%%%%%%%%%%%%

%%%%%%%%%%%%%%%%%%%%%%%%%%%%%%%%%%%%%%%%%%%%%%%%%%%%%%%%%%%%%%%%%%%%%%%%%%%%%%%%%%
%    Each exercise should look like 
% 		\exercise{ Exercise }{ Solution }
%%%%%%%%%%%%%%%%%%%%%%%%%%%%%%%%%%%%%%%%%%%%%%%%%%%%%%%%%%%%%%%%%%%%%%%%%%%%%%%%%%

\exercise{
	Prove the linear form version of Minkowski's theorem.
	
	Consider a real matrix $A = (a_{ij})_{i,j=1}^n$ and real numbers $c_1, \dots, c_n \geq 0$. The system of inequalities
	\begin{gather*}
		\left| \sum_{j=1}^n a_{1j} x_j \right| \leq c_1, \\
		\left| \sum_{j=1}^n a_{ij} x_j \right| < c_i, \ \forall i=2, \dots, n.
	\end{gather*}
	admits a non-zero integer solution if $c_1 \cdots c_n \geq | \det(A) |.$
}
{}

\end{document}

%%% Local Variables:
%%% mode: latex
%%% TeX-master: t
%%% End:
