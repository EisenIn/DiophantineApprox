\documentclass[12pt,a4paper]{article}

\usepackage{amsmath}
\usepackage{amssymb}
\usepackage{amsfonts}
\usepackage{amsthm}
\usepackage{graphics}
\usepackage{fullpage}
\usepackage{graphicx}
\usepackage{caption}
\usepackage{subcaption} 
\usepackage{enumerate}
\usepackage{polynom}
\usepackage[utf8]{inputenc} 
\usepackage{utf8math}
\usepackage{xfrac}

\date{}

\usepackage[boldsans]{concmath}

\theoremstyle{plain}
\newtheorem{theorem}{Th\'eor\`eme}
\newtheorem{Sol}{Solution}
\newtheorem*{Sol*}{Solution}
\theoremstyle{definition}
\newtheorem{Ex}{Exercise}
\newtheorem{lemma}[theorem]{Lemma}


\def \N {\mathbb N}
\def \Q {\mathbb Q}
\def \R {\mathbb R}
\def \Z {\mathbb Z}
\def \K {\mathbb K}
\def \C {\mathbb C}
\def \F {\mathbb F}
\def \id {{\rm id}\,}
\def \Ker {{\rm Ker}\,}
\def \Im {{\rm Im}\,}
\def \Vect {{\rm Vect}\,}

\newcommand{\df}{\mathrel{\mathop:}=}
\newcommand{\dx}{ \ \text{d} \, x}

\newcommand{\pscal}[1]{\langle {#1} \rangle}
\DeclareMathOperator{\spa}{span}


%%%%%%%%%%%%%%%%%%%%%%%%%%%%%%%%%%%%%%%%%%%%%%%%%%%%%%%%%%%%%%%%%%%%%%%%%%%%%%%%%%
%%%%%%%%%%%%%%%%%%%%%%%%%%%%%%%%%%%%%%%%%%%%%%%%%%%%%%%%%%%%%%%%%%%%%%%%%%%%%%%%%%
%
%       ENABLE or DISABLE dislpay of solutions
% 
%%%%%%%%%%%%%%%%%%%%%%%%%%%%%%%%%%%%%%%%%%%%%%%%%%%%%%%%%%%%%%%%%%%%%%%%%%%%%%%%%%
%%%%%%%%%%%%%%%%%%%%%%%%%%%%%%%%%%%%%%%%%%%%%%%%%%%%%%%%%%%%%%%%%%%%%%%%%%%%%%%%%%
		\newif\ifsolutions
		
		% ENABLE or DISABLE display of solutions
		\solutionstrue
		%\solutionsfalse


%%%%%%%%%%%%%%%%%%%%%%%%%%%%%%%%%%%%%%%%%%%%%%%%%%%%%%%%%%%%%%%%%%%%%%%%%%%%%%%%%%
%		Dont touch much, just change the correct number and date :). Based on the setup above, the solutions will be automaticelly displayed or hidden.
%%%%%%%%%%%%%%%%%%%%%%%%%%%%%%%%%%%%%%%%%%%%%%%%%%%%%%%%%%%%%%%%%%%%%%%%%%%%%%%%%%
		\newcommand{\exercise}[2]{
			\begin{Ex} #1 \end{Ex}
			\ifsolutions  \begin{Sol*} #2 \end{Sol*} \bigskip \else \bigskip  \fi
		}

		

%%%%%%%%%%%%%%%%%%%%%%%%%%%%%%%%%%%%%%%%%%%%%%%%%%%%%%%%%%%%%%%%%%%%%%%%%%%%%%%%%%
%   Beginning of the document. Make sure to add the correct dates and numbers everywhere.
%%%%%%%%%%%%%%%%%%%%%%%%%%%%%%%%%%%%%%%%%%%%%%%%%%%%%%%%%%%%%%%%%%%%%%%%%%%%%%%%%%
\begin{document}

\begin{center}
{Prof. Friedrich Eisenbrand \hfill November 10th 2023}
\end{center}
	
\hrule\vspace{\baselineskip}

\begin{center}
\textbf{Diophantine approximation}

Fall 2023

\bigskip

\textbf{Set 7}
\ifsolutions{\textbf{- Solutions}} \else{} \fi
\end{center}

\hrule\vspace{\baselineskip}


%%%%%%%%%%%%%%%%%%%%%%%%%%%%%%%%%%%%%%%%%%%%%%%%%%%%%%%%%%%%%%%%%%%%%%%%%%%%%%%%%%
%   Beginning of the exercises
%%%%%%%%%%%%%%%%%%%%%%%%%%%%%%%%%%%%%%%%%%%%%%%%%%%%%%%%%%%%%%%%%%%%%%%%%%%%%%%%%%

%%%%%%%%%%%%%%%%%%%%%%%%%%%%%%%%%%%%%%%%%%%%%%%%%%%%%%%%%%%%%%%%%%%%%%%%%%%%%%%%%%
%    Each exercise should look like 
% 		\exercise{ Exercise }{ Solution }
%%%%%%%%%%%%%%%%%%%%%%%%%%%%%%%%%%%%%%%%%%%%%%%%%%%%%%%%%%%%%%%%%%%%%%%%%%%%%%%%%%

\exercise{
	Prove the linear form version of Minkowski's theorem.
	
	Consider a real non-singular matrix $A = (a_{ij})_{i,j=1}^n$ and real numbers $c_1, \dots, c_n \geq 0$. The system of inequalities
	\begin{gather*}
		\left| \sum_{j=1}^n a_{1j} x_j \right| \leq c_1, \\
		\left| \sum_{j=1}^n a_{ij} x_j \right| < c_i, \ \forall i=2, \dots, n.
	\end{gather*}
	admits a non-zero integer solution if $c_1 \cdots c_n \geq | \det(A) |.$
}
{
	We modify each expression to get an inequality of the type
		\[ | B x | < 1 \qquad \iff \qquad \| Bx \|_\infty < 1, \]
	for which we find a nonzero integer $x$ under the condition that $\det(B) < 1$.

	For the condition to hold, we cannot have any zero $c_i$, since $A$ is non-singular. 
	Dividing each line by $c_i$, we define
		\[ B := \begin{pmatrix} & \frac{1}{c_1} a_1^T & \\ & \vdots & \\ & \frac{1}{c_n} a_n^T &  \end{pmatrix}. \]
	$B$ has determinant $\det B = \frac{\det A}{c_1 \cdots c_n} \leq 1$. 
	Notice that the inequality must be strict in order to apply Minkowski's theorem.
	To this end, we modify the first large inequality slightly, replacing $c_1$ by $\tilde{c}_1 = c_1 + \epsilon$,
	and $c_i$ ($i \geq 2$) by $\tilde{c}_i = c_i - \delta$, such at $\tilde{c}_1 \cdots \tilde{c}_n > | \det A |$.

	Minkowski's first theorem gives, for each $\epsilon > 0$ and some $\delta >0$, an integer vector $x = x(\epsilon)$ such that
	\begin{gather*}
		\left| \sum_{j=1}^n a_{1j} x_j \right| \leq c_1 + \epsilon, \\
		\left| \sum_{j=1}^n a_{ij} x_j \right| \leq c_i - \delta, \ \forall i=2, \dots, n.
	\end{gather*}
	Note now that all solutions of the above system are bounded by some constant which does not depend on $\epsilon$.
	Indeed, as
		\[ | Ax | \leq C, \]
	for $C = (c_1+1, c_2, \dots, c_n)^T$ for example, then
		\[ | x_i | = | (A^{-1} Ax)^T e_i | = | (Ax)^T (A^{-T} e_i) | \leq n \|C\|_\infty M, \]
	for $M$ bounding the coefficients of $A^{-1}$.

	As such, the number of \emph{integer} vectors $x$ verifying the modified system is finite, a number independent of $\epsilon$.
	There must therefore exist some nonzero $x$ integer which verifies the system for all $\epsilon>0$. In the limit, this vector concludes.

}

\exercise{
	The following exercise shows Lagrange's \emph{four-square theorem}. 
	That is, all integers $n \in \N$ may be written as the sum of four squares.
		
	\begin{enumerate}[i)]
		\item Show that it suffices to show the theorem for odd primes $p$.

		\item Show that there are exactly $\frac{p-1}{2}$ perfect squares in $\F_p^\times$.

		\item Deduce, by considering the tuples $(0, p-1), \dots, (\frac{p-1}2, \frac{p-1}2)$, that there exist non-negative integers $a, b \in \N_0$ such that 
			\[ a^2 + b^2 \equiv  -1 \mod {p}. \]

		\item Define the hypersphere
			\[ \mathcal{C} \df \left\{ (x,y,z,w) \in \R^4 \ | \ x^2 + y^2 + z^2 + w^2 < 2p \right\}, \]
		and the lattice
			\[ \Lambda \df \begin{pmatrix} p & 0 & a & b \\
										0 & p & -b & a \\
										0 & 0& 1 & 0 \\
										0 & 0 & 0 & 1 \end{pmatrix}. \]
		Conclude by showing that the shortest nonzero vector $v \in \Lambda \setminus \{ 0 \}$  verifies
			\[ \| v \|_2^2 < 2p, \qquad \text { and } \qquad  \| v \|_2^2  \equiv 0 \mod{p}. \]
	\end{enumerate}

}
{
	\begin{enumerate}[i)]
		\item Letting $\{1, i, j, k\}$ denote the basis of the quaternions, we have the following norm.
			\[ N(a + bi + cj + dk) = \prod_{\pm^4} (\pm a \pm bi \pm cj \pm dk) = a^2 + b^2 + c^2 + d^2. \]
		By associativity of the quaternions, and if $n, m$ may be written as sums of four squares, then
			\[ n = N(\alpha), \qquad m = N(\beta), \text{ and } \qquad n\cdot m = N(\alpha) N(\beta) = N(\alpha \cdot \beta), \]
		and $n \cdot m$ may also be written as sum of four squares.
		The multiplication of $\alpha$ by $\beta$ gives the integers in question.
		This concludes, since $2 = 1 + 1 + 0 + 0$ may trivially be written as the sum of four squares. 

		One may compute the integers for $n \cdot m$ and verify the result by hand using the relations $ij = k, jk = i, ki = j$ and $i^2 = j^2 = k^2 = -1$.


		\item The morphism of groups
			\begin{align*}
				\phi : \F_p &\rightarrow \F_p \\
					x &\mapsto x^2
			\end{align*}
		verifies $|\Im(\phi)| = \frac{p-1}{| \ker(\phi)|}$.
		The kernel of $\phi$ contains exactly two elements, since the polynomial $x^2 - 1$ splits as $(x+1)(x-1)$ in the field: the only square roots of $1$ are $\pm1$.

		\item The tuples are listed as $(a, b)$ such that
			\[ a + b = -1 \mod{p}. \]
		We must only verify that at least one of these tuples contains two squares.
		Note that $0$ is trivially a square, and that the last tuple contains twice the same number.
		Hence, if neither the first nor the last tuple contain two squares, the other $\frac{p-1}2 - 1$ ones are made of $\frac{p-1}2$ distinct squares. 
		 By pigeonhole, one tuple must therefore contain two squares.

		\item Note that ${\rm vol}(C) = \frac12 \pi^2 (\sqrt{2p})^4 = 2 \pi^2 p^2$, and that $2^4 \det \Lambda = 16p^2 < {\rm vol}(C)$.

		Minkowski's first theorem gives a vector $v \in \left( \Lambda \cap C \right) \setminus \{0\}$.
		Routine computation then yields
			\[ \| v \|_2^2 < 2p, \qquad \text { and } \qquad  \| v \|_2^2  \equiv 0 \mod{p}, \]
		which mean that $\| v \|_2^2  = p$, and which concludes.
		
	\end{enumerate}

}



\exercise{
	Consider the expansion $\alpha = [a_0; a_1, \dots]$ of some $\alpha \in \R\setminus\Q$, with convergents $\{p_{k}/q_{k}\}_k$ given by the truncated expansion.

	Define $x_k := [a_k; a_{k+1} \dots]$, and
	recall that one showed in set 3 that, for all $k\geq0$, 
		\[ \alpha = \frac{x_{k+1}p_k + p_{k-1}}{x_{k+1}q_k + q_{k-1}}. \]
	 We say that $\alpha$ is \emph{badly approximable} if there exists a constant $c > 0$ such that
		\[ | q \alpha - p | ~ > \frac{c}{q}, \]
	for all integers $q \geq 1, p \in \Z$.

	Show that $\alpha$ is \emph{badly approximable} if and only if the coefficients $\{ a_n \}_n$ are bounded.
}
{
	Routine computation gives 
		\[ \left| \alpha - \frac{p_k}{q_k} \right| = \left| \frac{q_k p_{k-1} - p_k q_{k-1}}{q_k \left( q_k x_{k+1} + q_{k-1} \right)}  \right| = \frac{1}{q_k \left( q_k x_{k+1} + q_{k-1} \right)}. \]
	As such, the convergents of $\alpha$ approximate it with quality
		\[ | q_k \alpha - p_k | \leq \frac{1}{q_k x_{k+1} + q_{k-1}}. \]
	On one hand, if $\alpha$ is badly approximable, then there exists a constant $c>0$ such that, for each $k \geq 0$,
		\[ \frac{c}{q_k} \leq \frac{1}{q_k x_{k+1} + q_{k-1}} \leq \frac{1}{q_k x_{k+1}}. \]
	Hence, the expansions $x_k := [a_k; a_{k+1} \dots]$ are bounded by $\frac{1}c$ for each $k$.
	As $a_k \leq x_k$, the coefficients $\{ a_n \}_n$ are bounded.

	On the other hand, if the coefficients $\{ a_n \}_n$ are bounded by $M$, then
		\[ q_k \left| q_k \alpha - p_k \right|= \frac{1}{\left( x_{k+1} + q_{k-1}/q_k \right)} \geq \frac{1}{M + 1}. \]
	To conclude, notice that for any $q_k \leq q < q_{k+1}, p \in \Z$ one thus has
		\[ | q \alpha - p |  \geq | q_k \alpha - p_k | \geq \frac{c}{q_k} \geq \frac{c}{q}. \]



}

\exercise{
  Let $b_1,\dots,b_n$ be an orthonormal basis of $ℝ^n$, $λ_1,\dots,λ_n ∈ ℝ_{>0}$  and consider
  \begin{displaymath}
    \mathcal{E} = \left\{ ∑_{i=1}^n x_i b_i : \left\| ∑_{i=1}^n x_i \frac{b_i}{λ_i} \right\|_2 \leq 1 \right\}. 
  \end{displaymath}
  Show that $\mathrm{vol}(\mathcal{E}) = \left( ∏_{i=1}^n λ_i \right) {\rm vol}(B(1,0))$, where
  $B(1,0) = \{ x∈ ℝ^n :\|x\|_2 ≤1\}$. 
}
{
	Write $\mathcal{E}$ as 
		\[  \mathcal{E} = \left\{ Bx: \left\| (B \Lambda^{-1})x \right\|_2 \leq 1 \right\}, \]
	where $B = (b_1, \dots, b_n)$ and $\Lambda = {\rm diag}(\lambda_1, \dots, \lambda_n)$.
		
	This is equivalent to
		\[  \mathcal{E} = \left\{ B\Lambda B^{-1} y: \left\| y \right\|_2 \leq 1 \right\} = \left( B \Lambda B^{-1} \right) \cdot B(1,0), \]
	which has volume ${\rm vol}(\mathcal{E}) = \det\left( B \Lambda B^{-1} \right) \cdot {\rm vol}B(1,0) = \left( \prod_{i=1}^n \lambda_i \right) {\rm vol}B(1,0).$


}




\end{document}

%%% Local Variables:
%%% mode: latex
%%% TeX-master: t
%%% End:
