\documentclass[12pt,a4paper]{article}

\usepackage{amsmath}
\usepackage{amssymb}
\usepackage{amsfonts}
\usepackage{amsthm}
\usepackage{graphics}
\usepackage{fullpage}
\usepackage{graphicx}
\usepackage{caption}
\usepackage{subcaption} 
\usepackage{enumerate}
\usepackage{polynom}
\usepackage[utf8]{inputenc} 
\usepackage{utf8math}
\usepackage{xfrac}

\date{}

\usepackage[boldsans]{concmath}

\theoremstyle{plain}
\newtheorem{theorem}{Th\'eor\`eme}
\newtheorem{Sol}{Solution}
\newtheorem*{Sol*}{Solution}
\theoremstyle{definition}
\newtheorem{Ex}{Exercise}
\newtheorem{lemma}[theorem]{Lemma}


\def \N {\mathbb N}
\def \Q {\mathbb Q}
\def \R {\mathbb R}
\def \Z {\mathbb Z}
\def \K {\mathbb K}
\def \C {\mathbb C}
\def \F {\mathbb F}
\def \id {{\rm id}\,}
\def \Ker {{\rm Ker}\,}
\def \Im {{\rm Im}\,}
\def \Vect {{\rm Vect}\,}

\newcommand{\df}{\mathrel{\mathop:}=}
\newcommand{\dx}{ \ \text{d} \, x}

\newcommand{\pscal}[1]{\langle {#1} \rangle}
\DeclareMathOperator{\spa}{span}


%%%%%%%%%%%%%%%%%%%%%%%%%%%%%%%%%%%%%%%%%%%%%%%%%%%%%%%%%%%%%%%%%%%%%%%%%%%%%%%%%%
%%%%%%%%%%%%%%%%%%%%%%%%%%%%%%%%%%%%%%%%%%%%%%%%%%%%%%%%%%%%%%%%%%%%%%%%%%%%%%%%%%
%
%       ENABLE or DISABLE dislpay of solutions
% 
%%%%%%%%%%%%%%%%%%%%%%%%%%%%%%%%%%%%%%%%%%%%%%%%%%%%%%%%%%%%%%%%%%%%%%%%%%%%%%%%%%
%%%%%%%%%%%%%%%%%%%%%%%%%%%%%%%%%%%%%%%%%%%%%%%%%%%%%%%%%%%%%%%%%%%%%%%%%%%%%%%%%%
		\newif\ifsolutions
		
		% ENABLE or DISABLE display of solutions
		%\solutionstrue
		\solutionsfalse


%%%%%%%%%%%%%%%%%%%%%%%%%%%%%%%%%%%%%%%%%%%%%%%%%%%%%%%%%%%%%%%%%%%%%%%%%%%%%%%%%%
%		Dont touch much, just change the correct number and date :). Based on the setup above, the solutions will be automaticelly displayed or hidden.
%%%%%%%%%%%%%%%%%%%%%%%%%%%%%%%%%%%%%%%%%%%%%%%%%%%%%%%%%%%%%%%%%%%%%%%%%%%%%%%%%%
		\newcommand{\exercise}[2]{
			\begin{Ex} #1 \end{Ex}
			\ifsolutions  \begin{Sol*} #2 \end{Sol*} \bigskip \else \bigskip  \fi
		}

		

%%%%%%%%%%%%%%%%%%%%%%%%%%%%%%%%%%%%%%%%%%%%%%%%%%%%%%%%%%%%%%%%%%%%%%%%%%%%%%%%%%
%   Beginning of the document. Make sure to add the correct dates and numbers everywhere.
%%%%%%%%%%%%%%%%%%%%%%%%%%%%%%%%%%%%%%%%%%%%%%%%%%%%%%%%%%%%%%%%%%%%%%%%%%%%%%%%%%
\begin{document}

\begin{center}
{Prof. Friedrich Eisenbrand \hfill November 10th 2023}
\end{center}
	
\hrule\vspace{\baselineskip}

\begin{center}
\textbf{Diophantine approximation}

Fall 2023

\bigskip

\textbf{Set 7}
\ifsolutions{\textbf{- Solutions}} \else{} \fi
\end{center}

\hrule\vspace{\baselineskip}


%%%%%%%%%%%%%%%%%%%%%%%%%%%%%%%%%%%%%%%%%%%%%%%%%%%%%%%%%%%%%%%%%%%%%%%%%%%%%%%%%%
%   Beginning of the exercises
%%%%%%%%%%%%%%%%%%%%%%%%%%%%%%%%%%%%%%%%%%%%%%%%%%%%%%%%%%%%%%%%%%%%%%%%%%%%%%%%%%

%%%%%%%%%%%%%%%%%%%%%%%%%%%%%%%%%%%%%%%%%%%%%%%%%%%%%%%%%%%%%%%%%%%%%%%%%%%%%%%%%%
%    Each exercise should look like 
% 		\exercise{ Exercise }{ Solution }
%%%%%%%%%%%%%%%%%%%%%%%%%%%%%%%%%%%%%%%%%%%%%%%%%%%%%%%%%%%%%%%%%%%%%%%%%%%%%%%%%%

\exercise{
	Prove the linear form version of Minkowski's theorem.
	
	Consider a real matrix $A = (a_{ij})_{i,j=1}^n$ and real numbers $c_1, \dots, c_n \geq 0$. The system of inequalities
	\begin{gather*}
		\left| \sum_{j=1}^n a_{1j} x_j \right| \leq c_1, \\
		\left| \sum_{j=1}^n a_{ij} x_j \right| < c_i, \ \forall i=2, \dots, n.
	\end{gather*}
	admits a non-zero integer solution if $c_1 \cdots c_n \geq | \det(A) |.$
}
{}

\exercise{
	The following exercise shows Lagrange's four square theorem. 
	That is, all integers $n \in \N$ may be written as the sum of four squares.
		
	\begin{enumerate}[i)]
		\item Show that it suffices to show the theorem for odd primes $p$.

		\item Show that there are exactly $\frac{p-1}{2}$ perfect squares in $\F_p^\times$.

		\item Deduce, by considering the tuples $(0, p-1), \dots, (\frac{p-1}2, \frac{p-1}2)$, that there exist non-negative integers $a, b \in \N_0$ such that 
			\[ a^2 + b^2 \equiv  1 \mod {p}. \]

		\item Define the hypersphere
			\[ \mathcal{C} \df \{ (x,y,z,w) \in \R^4 \ | \ x^2 + y^2 + z^2 + w^2 < 2p \}, \]
		and the lattice
			\[ \Lambda \df \begin{pmatrix} p & 0 & a & b \\
										0 & p & -b & a \\
										0 & 0& 1 & 0 \\
										0 & 0 & 0 & 1 \end{pmatrix}. \]
		Conclude by showing that the shortest nonzero vector $v \in \Lambda \setminus \{ 0 \}$  verifies
			\[ \| v \|_2^2 < 2p, \qquad \text { and } \qquad  \| v \|_2^2  \equiv 0 \mod{p}. \]
	\end{enumerate}

}

% COMMENTED: would have fit in an earlier sheet (set 3, 4).
%\exercise{
%	Consider the expansion $\alpha = [a_0; a_1, \dots]$ of some $\alpha \in \R\setminus\Q$, with convergents $\{p_{k}/q_{k}\}_k$ given by the truncated expansion.
%
%	Define $x_k := [a_k; a_{k+1} \dots]$, and
%	recall that one showed in set 3 that, for all $k\geq0$, 
%		\[ \alpha = \frac{x_{k+1}p_k + p_{k-1}}{x_{k+1}q_k + q_{k-1}}. \]
%	 We say that $\alpha$ is \emph{badly approximable} if there exists a constant $c > 0$ such that
%		\[ | q \alpha - p | ~ > \frac{c}{q}, \]
%	for all integers $q \geq 1, p \in \Z$.
%
%	Show that $\alpha$ is \emph{badly approximable} if and only if the coefficients $\{ a_n \}_n$ are bounded.
%}
%{}

\exercise{
  Let $b_1,\dots,b_n$ be an orthonormal basis of $ℝ^n$, $λ_1,\dots,λ_n ∈ ℝ_{>0}$  and consider
  \begin{displaymath}
    \mathcal{E} = \left\{ ∑_{i=1}^n x_i b_i : \left\| ∑_{i=1}^n x_i \frac{b_i}{λ_i} \right\| \leq 1 \right\}. 
  \end{displaymath}
  Show that $\mathrm{vol}(\mathcal{E}) = \left( ∏_{i=1}^n λ_i \right) {\rm vol}(B(1,0))$, where
  $B(1,0) = \{ x∈ ℝ^n :\|x\|≤1\}$. 
}
{}

\exercise{
	Let $\alpha = (\alpha_1, \dots, \alpha_n) \in \R^n$ be \emph{badly approximable}, that is there exists a constant $c > 0$ such that
		\[ \| q \alpha - p \|_\infty > \frac{c}{q^{1/n}}, \]
	for all integers $q \geq 1, p \in \Z^n$.

	The goal of this exercise is to evaluate from below, the signed sum $| y^T \alpha |$, where $y \in \{ \pm 1 \}^n$.

	Consider the lattice $\Lambda$, dependent on a real quantity $Q > 0$,
		\[ \Lambda \df \begin{pmatrix} Q^{n+1} & 0^T \\
								\alpha & I_n \end{pmatrix} \cdot \Z^{n+1}, \]
	with successive minima $\lambda_1, \dots, \lambda_{n+1}$.
	
	\begin{enumerate}[i)]
		\item 
		Show that, for all $\epsilon > 0$, one has
			\[ \frac{1}{Q^{1 + \epsilon}} \leq \lambda_1 \leq \frac{1}{Q}, \]
		for all $Q \geq c^{-1/\epsilon}$ large enough. Crude bounds are enough here.
		
		\item 
		Let $\epsilon = \frac1{2n}$. Deduce from Minkowski's second theorem that $\lambda_{n+1}$ is vanishing in $Q$, and that one may find $q \geq 1$, $p \in \Z^n$ such that
			\[ \| q \alpha - p \| \leq \frac{1}{2n}, \qquad \text{ and } \qquad y^T p \neq 0, \qquad \text{ and } \qquad q \leq \frac{(n+1)^{n^2 + 3n}}{c^{2n^2}}. \]
		
		\item
		Conclude that 
			\[ | y^T \alpha | \geq \frac{1}{2q} \geq \frac{c^{2n^2}}{2n^{n^2 + 3n}} . \]
	\end{enumerate}

}
{}



\end{document}

%%% Local Variables:
%%% mode: latex
%%% TeX-master: t
%%% End:
