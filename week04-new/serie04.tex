\documentclass[12pt,a4paper]{article}

\usepackage{amsmath}
\usepackage{amssymb}
\usepackage{amsfonts}
\usepackage{amsthm}
\usepackage{graphics}
\usepackage{fullpage}
\usepackage{graphicx}
\usepackage{caption}
\usepackage{subcaption} 
\usepackage{enumerate}
\usepackage{polynom}
\usepackage[utf8]{inputenc} 
\usepackage{utf8math}
\usepackage{xfrac}

\date{}

\usepackage[boldsans]{concmath}

\theoremstyle{plain}
\newtheorem{theorem}{Th\'eor\`eme}
\newtheorem{Sol}{Solution}
\newtheorem*{Sol*}{Solution}
\theoremstyle{definition}
\newtheorem{Ex}{Exercice}


\def \N {\mathbb N}
\def \Q {\mathbb Q}
\def \R {\mathbb R}
\def \Z {\mathbb Z}
\def \K {\mathbb K}
\def \C {\mathbb C}
\def \id {{\rm id}\,}
\def \Ker {{\rm Ker}\,}
\def \Im {{\rm Im}\,}
\def \Vect {{\rm Vect}\,}

\newcommand{\df}{\mathrel{\mathop:}=}
\newcommand{\dx}{ \ \text{d} \, x}

\newcommand{\pscal}[1]{\langle {#1} \rangle}
\DeclareMathOperator{\spa}{span}


%%%%%%%%%%%%%%%%%%%%%%%%%%%%%%%%%%%%%%%%%%%%%%%%%%%%%%%%%%%%%%%%%%%%%%%%%%%%%%%%%%
%%%%%%%%%%%%%%%%%%%%%%%%%%%%%%%%%%%%%%%%%%%%%%%%%%%%%%%%%%%%%%%%%%%%%%%%%%%%%%%%%%
%
%       ENABLE or DISABLE dislpay of solutions
% 
%%%%%%%%%%%%%%%%%%%%%%%%%%%%%%%%%%%%%%%%%%%%%%%%%%%%%%%%%%%%%%%%%%%%%%%%%%%%%%%%%%
%%%%%%%%%%%%%%%%%%%%%%%%%%%%%%%%%%%%%%%%%%%%%%%%%%%%%%%%%%%%%%%%%%%%%%%%%%%%%%%%%%
		\newif\ifsolutions
		
		% ENABLE or DISABLE display of solutions
		%\solutionstrue
		\solutionsfalse


%%%%%%%%%%%%%%%%%%%%%%%%%%%%%%%%%%%%%%%%%%%%%%%%%%%%%%%%%%%%%%%%%%%%%%%%%%%%%%%%%%
%		Dont touch much, just change the correct number and date :). Based on the setup above, the solutions will be automaticelly displayed or hidden.
%%%%%%%%%%%%%%%%%%%%%%%%%%%%%%%%%%%%%%%%%%%%%%%%%%%%%%%%%%%%%%%%%%%%%%%%%%%%%%%%%%
		\newcommand{\exercise}[2]{
			\begin{Ex} #1 \end{Ex}
			\ifsolutions  \begin{Sol*} #2 \end{Sol*} \bigskip \else \bigskip  \fi
		}

		

%%%%%%%%%%%%%%%%%%%%%%%%%%%%%%%%%%%%%%%%%%%%%%%%%%%%%%%%%%%%%%%%%%%%%%%%%%%%%%%%%%
%   Beginning of the document. Make sure to add the correct dates and numbers everywhere.
%%%%%%%%%%%%%%%%%%%%%%%%%%%%%%%%%%%%%%%%%%%%%%%%%%%%%%%%%%%%%%%%%%%%%%%%%%%%%%%%%%
\begin{document}

\begin{center}
{Prof. Friedrich Eisenbrand \hfill October 20th 2023}
\end{center}
	
\hrule\vspace{\baselineskip}

\begin{center}
\textbf{Diophantine approximation}

Fall 2023

\bigskip

\textbf{Set 4}
\ifsolutions{\textbf{- Solutions}} \else{} \fi
\end{center}

\hrule\vspace{\baselineskip}


%%%%%%%%%%%%%%%%%%%%%%%%%%%%%%%%%%%%%%%%%%%%%%%%%%%%%%%%%%%%%%%%%%%%%%%%%%%%%%%%%%
%   Beginning of the exercises
%%%%%%%%%%%%%%%%%%%%%%%%%%%%%%%%%%%%%%%%%%%%%%%%%%%%%%%%%%%%%%%%%%%%%%%%%%%%%%%%%%

%%%%%%%%%%%%%%%%%%%%%%%%%%%%%%%%%%%%%%%%%%%%%%%%%%%%%%%%%%%%%%%%%%%%%%%%%%%%%%%%%%
%    Each exercise should look like 
% 		\exercise{ Exercise }{ Solution }
%%%%%%%%%%%%%%%%%%%%%%%%%%%%%%%%%%%%%%%%%%%%%%%%%%%%%%%%%%%%%%%%%%%%%%%%%%%%%%%%%%
\exercise{ Let $α = \sqrt{d}$ where $d ∈ ℕ_+$ is not a perfect square. We consider the field $ℚ(α) = \{ x + y α : x, y ∈ℚ \}$.  The \emph{norm} of an element $x +  y α ∈ ℚ(α)$ is defined as
  \begin{displaymath}
    N(x +  y α) = (x +  y α)(x -  y α). 
  \end{displaymath}
  Show $\Q(\alpha)$ is indeed a field, and that the norm is multiplicative, i.e. that one has
  \begin{displaymath}
    N(u⋅v) = N(u)⋅N(v) \text{ for each } u,v ∈ℚ(α). 
  \end{displaymath}

{\bigskip \noindent  \small \emph{ Hint: Show that the map $φ: ℚ(α) → ℚ(α)$, $φ( x + y α ) =  x - y α $ is an isomorphism (endomorphism) of $ℚ(α)$.} }
}
{
	The norm of an element of the field may be understood as the determinant of its multiplication matrix (in any basis).
	Indeed, take $\alpha = x + y \sqrt{d}$, and write the multiplication by $\alpha$ matrix in the basis $\{ 1, \sqrt{d} \}$
	\[
		[ \times \alpha ] = \begin{pmatrix} x & d y \\ y & x \end{pmatrix},
	\]
	which has determinant $x^2 - d y^2$.

	By commutativity of the field, the multiplication maps commute, which yields
		\[
		N(u \cdot v) = \det([ \times (u \cdot v) ]) = \det( [\times u] \cdot [\times v] ) = N(u) \cdot N(v), \]
	as required.

}

\exercise{Find a nontrivial  solution to the Pell equation $x^2 - 5 y^2 =1$, $x,y ∈ ℤ$.}{}

\exercise{Suppose $d ∈ℕ_+$ is not a perfect square such that $d ≡ 3 \pmod{4}$. Show that $ℤ[\sqrt{d}]$ has no units of norm $-1$. }
{
	If an element $x+y\sqrt{d} \in \Z[\sqrt{d}]$ has norm $-1$, then one has
		\[ x^2 + y^2 \equiv 3 \pmod{4}. \]
	Since squares have residue 0 or 1 mod 4, this equation admits no solution.
}


\exercise{Let $d ∈ ℕ_+$ be not a perfect square. The \emph{fundamental unit} of $ℤ[\sqrt{d}]$ is the smallest (as number in $ℝ$) unit $x + y \sqrt{d}$ of $ℤ[\sqrt{d}]$ with $x,y ∈ℕ_+$.

\bigskip 

\noindent
\begin{enumerate}[i)]
\item 
  Show that the fundamental unit exists and that it is uniquely defined.
\item Show that the fundamental unit is the smallest $ε>1$ in $ℤ[\sqrt{d}]$ such that $|N(ε)| =  1$. 
\end{enumerate}
}{
	\begin{enumerate}[i)]
		\item
		By exercise 1 of set 3, there exist infinitely many solutions of the equation $N(\alpha) = k, \alpha \in \Z[\sqrt{d}]$ for some integer $k < 1 + 2\sqrt{d}$.
			As such, there must exist two elements $\alpha, \beta \in \Z[\sqrt{d}]$ with coordinates in the basis $\{1, \sqrt{d}\}$ congruent mod $k$.

		One then verifies that $\frac1k \alpha \bar{\beta} \in \Z[\sqrt{d}]$, which concludes, since
			\[  N\left(\frac1k \alpha \bar{\beta}\right) \in \Z[\sqrt{d}] = \frac1{k^2} N(u) N(v) = 1. \]

		The uniqueness is then given by the fact that 1 and $\sqrt{d}$ are linearly independent over $\Q$.
		\item
		Let $\alpha = x + y\sqrt{d}$ be the fundamental unit of $\Z[\sqrt{d}]$. Clearly, $\alpha > 1$, since both $x$ and $y$ are positive.
		To conclude, it suffices to show that $\alpha$ is a unit if and only if $| N(\alpha) | = 1$.

		On one hand, if $\alpha$ is a unit, then $N(\alpha) N(\alpha^{-1}) = N(1) = 1$, which means that $N(\alpha)$ is a unit of $\Z$, ie. $\pm1$.

		On the other hand, if $N(\alpha)=1$, then $\alpha \bar{\alpha} = 1$, where $\bar{\alpha} = x - y \sqrt{d} \in \Z[\sqrt{d}]$.
	\end{enumerate}
}

\exercise{ Let $d ∈ ℕ_+$ be not a perfect square and let $ε$ be the fundamental unit of $ℤ[\sqrt{d}]$. Show that $G = \{ \pm ε ^z : z ∈ℤ\}$  is precisely the group of units of $ℤ[\sqrt{d}]$.

  \bigskip
  
  \noindent
  Describe all solutions to the Pell equation. 
}
{
	Let $\alpha = x + y\sqrt{d}$ be a unit of $Z[\sqrt{d}]$. We show that $\alpha$ belongs to $G$.

	We may write $x - y\sqrt{d}, -x + y\sqrt{d}$, and $-x - y\sqrt{d}$ as $\pm \alpha^{-1}$ and $-\alpha$, so that we may suppose $x, y \geq 0$.

	The case $x=1, y=0$ corresponds to $n=0$ in $G$, which lets us assume that $\alpha > 1$.

	There therefore exists an $n \geq 1$ such that
		\[ \epsilon^n \leq \alpha < \epsilon^{n+1}, \]
	by minimality of $\epsilon$.

	This means that $1 \leq \alpha \epsilon^{-n} < \alpha$. 
	By exercise 4ii), one must have $\alpha = \epsilon^n$, which concludes.
}



\exercise{
  Show the following variant of Liouville's Theorem:

  \bigskip 
  
  \noindent  
    Suppose that $α∈ ℝ$ is an non-zero algebraic number of degree $\mathbf{d ≥1}$. There exists a positive constant $c_α>0$ such that for each $(p,q) ∈ ℤ×ℕ_+$ with $\gcd(p,q) = 1$ one has
  \begin{displaymath}
    |α - p/q| > c_α/q^d. 
  \end{displaymath}
\smallskip 
\noindent 
\emph{Hint: Only the case when $α$ is rational needs to be considered, because of Liouville's original theorem.}
}{}


\exercise{Show that the \emph{Liouville number} $∑_{n=0}^{∞} 2^{-n!}$ is transcendental.

  \bigskip

\emph{Hint:   Consider integers  $p_j = 2^{j!} ∑_{n=0}^j 2^{-n!}$ and $q_j = 2^{j!}$ for each $j \geq 0$. }
  
}{}
  
\end{document}

%%% Local Variables:
%%% mode: latex
%%% TeX-master: t
%%% End:
