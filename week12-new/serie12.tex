\documentclass[12pt,a4paper]{article}

\usepackage{amsmath}
\usepackage{amssymb}
\usepackage{amsfonts}
\usepackage{amsthm}
\usepackage{graphics}
\usepackage{fullpage}
\usepackage{graphicx}
\usepackage{caption}
\usepackage{subcaption} 
\usepackage{enumerate}
\usepackage{polynom}
\usepackage[utf8]{inputenc} 
\usepackage{utf8math}
\usepackage{xfrac}

\date{}

\usepackage[boldsans]{concmath}

\theoremstyle{plain}
\newtheorem{theorem}{Theorem}
\newtheorem{Sol}{Solution}
\newtheorem*{Sol*}{Solution}
\theoremstyle{definition}
\newtheorem{Ex}{Exercise}
\newtheorem{definition}{Definition}
\newtheorem{lemma}[theorem]{Lemma}


\def \N {\mathbb N}
\def \Q {\mathbb Q}
\def \R {\mathbb R}
\def \Z {\mathbb Z}
\def \K {\mathbb K}
\def \C {\mathbb C}
\def \F {\mathbb F}
\def \id {{\rm id}\,}
\def \Ker {{\rm Ker}\,}
\def \Im {{\rm Im}\,}
\def \Vect {{\rm Vect}\,}

\newcommand{\df}{\mathrel{\mathop:}=}
\newcommand{\dx}{ \ \text{d} \, x}

\newcommand{\pscal}[1]{\langle {#1} \rangle}
\DeclareMathOperator{\spa}{span}
\newcommand{\nint}[1]{\ensuremath{\lfloor#1\rceil}}


\newcommand{\qp}{\begin{pmatrix} q \\ p \end{pmatrix}}
\newcommand{\onealpha}{\begin{pmatrix}1 \\ \alpha \end{pmatrix}}

%%%%%%%%%%%%%%%%%%%%%%%%%%%%%%%%%%%%%%%%%%%%%%%%%%%%%%%%%%%%%%%%%%%%%%%%%%%%%%%%%%
%%%%%%%%%%%%%%%%%%%%%%%%%%%%%%%%%%%%%%%%%%%%%%%%%%%%%%%%%%%%%%%%%%%%%%%%%%%%%%%%%%
%
%       ENABLE or DISABLE dislpay of solutions
% 
%%%%%%%%%%%%%%%%%%%%%%%%%%%%%%%%%%%%%%%%%%%%%%%%%%%%%%%%%%%%%%%%%%%%%%%%%%%%%%%%%%
%%%%%%%%%%%%%%%%%%%%%%%%%%%%%%%%%%%%%%%%%%%%%%%%%%%%%%%%%%%%%%%%%%%%%%%%%%%%%%%%%%
		\newif\ifsolutions
		
		% ENABLE or DISABLE display of solutions
		%\solutionstrue
		\solutionsfalse


%%%%%%%%%%%%%%%%%%%%%%%%%%%%%%%%%%%%%%%%%%%%%%%%%%%%%%%%%%%%%%%%%%%%%%%%%%%%%%%%%%
%		Dont touch much, just change the correct number and date :). Based on the setup above, the solutions will be automaticelly displayed or hidden.
%%%%%%%%%%%%%%%%%%%%%%%%%%%%%%%%%%%%%%%%%%%%%%%%%%%%%%%%%%%%%%%%%%%%%%%%%%%%%%%%%%
		\newcommand{\exercise}[2]{
			\begin{Ex} #1 \end{Ex}
			\ifsolutions  \begin{Sol*} #2 \end{Sol*} \bigskip \else \bigskip  \fi
		}

		

%%%%%%%%%%%%%%%%%%%%%%%%%%%%%%%%%%%%%%%%%%%%%%%%%%%%%%%%%%%%%%%%%%%%%%%%%%%%%%%%%%
%   Beginning of the document. Make sure to add the correct dates and numbers everywhere.
%%%%%%%%%%%%%%%%%%%%%%%%%%%%%%%%%%%%%%%%%%%%%%%%%%%%%%%%%%%%%%%%%%%%%%%%%%%%%%%%%%
\begin{document}

\begin{center}
{Prof. Friedrich Eisenbrand \hfill December 15th 2023}
\end{center}
	
\hrule\vspace{\baselineskip}

\begin{center}
\textbf{Diophantine approximation}

Fall 2023

\bigskip

\textbf{Set 12}
\ifsolutions{\textbf{- Solutions}} \else{} \fi
\end{center}

\hrule\vspace{\baselineskip}


%%%%%%%%%%%%%%%%%%%%%%%%%%%%%%%%%%%%%%%%%%%%%%%%%%%%%%%%%%%%%%%%%%%%%%%%%%%%%%%%%%
%   Beginning of the exercises
%%%%%%%%%%%%%%%%%%%%%%%%%%%%%%%%%%%%%%%%%%%%%%%%%%%%%%%%%%%%%%%%%%%%%%%%%%%%%%%%%%

%%%%%%%%%%%%%%%%%%%%%%%%%%%%%%%%%%%%%%%%%%%%%%%%%%%%%%%%%%%%%%%%%%%%%%%%%%%%%%%%%%
%    Each exercise should look like 
% 		\exercise{ Exercise }{ Solution }
%%%%%%%%%%%%%%%%%%%%%%%%%%%%%%%%%%%%%%%%%%%%%%%%%%%%%%%%%%%%%%%%%%%%%%%%%%%%%%%%%%


\exercise{\label{ex:1}

	Let $n\geq1$, and $a_1, \dots, a_n \in \N$ such that no nonemtpy subset of the $\sqrt{a_i}$'s has rational product.

	The goal of this exercise is to prove by induction that all $2^n$ products of elements of $\{ \sqrt{a_1}, \dots, \sqrt{a_n} \}$ are linearly independent over $\Q$.

	\begin{enumerate}[i)]

		\item 
		Show the result for $n=1$.

		\item Assume that there exists some $n \geq 2$ such that the result holds for $n-1$ but does not hold for $n$.
		Show that there exist $\alpha, \beta$ $\Q$-linear combinations of the products of $\{ \sqrt{a_1}, \dots, \sqrt{a_{n-2}} \}$ (the set is empty in the case $n=2$) such that
			\begin{align}\label{eq:1}
				 \sqrt{a_n} = \alpha + \beta \sqrt{a_{n-1}},
			\end{align}
		and hence that
			\begin{align}\label{eq:2}
				a_n = \alpha^2 + \beta^2 a_{n-1} + 2 \alpha \beta \sqrt{a_{n-1}}.
			\end{align}

		\item Show that \eqref{eq:1} implies $\beta \neq 0$, and that equation \eqref{eq:2} implies $\alpha\beta = 0$.

		\item Show that $\sqrt{a_n \cdot a_{n-1}} = a_{n-1} \beta$ and conclude by contradiction.
	\end{enumerate}

}{}

\exercise{
	Let $a_1, \dots, a_n \in \N$ be square-free distinct integers. Show that $\sqrt{a_1}, \dots, \sqrt{a_n}$ are linearly independent over $\Q$ using exercise \ref{ex:1}.
}{}

\exercise{
	TODO: Show that the conjugates of $x_i \sqrt{a_i}$ are all $\pm x_i \sqrt{a_i}$, and that their product is integer (nonzero in the case of distinct squarefree integers).

}{}

\exercise{
	TODO. Case $a_1 = 1$ in paper.

}


\end{document}

%%% Local Variables:
%%% mode: latex
%%% TeX-master: t
%%% End:
